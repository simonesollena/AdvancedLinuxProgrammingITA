% !TEX encoding = ISO-8859-1
% !TEX root = alp.tex
% !TEX program = pdflatex
% !TEX spellcheck = it-IT

\documentclass[11pt,a4paper,draft,openright]{book}
% bozza: evidenzia con una linea vert. il testo che straripa
%\documentclass[11pt,a4paper,final,openright]{book} % versione finale


%\usepackage{latexsym}
%\usepackage[applemac]{inputenc}
\usepackage[latin1]{inputenc}		% permette di scrivere le lettere accentate senza \' o \�
\usepackage[italian]{babel}
\usepackage{hyperref}			% per creare l'indice con hyperlink
\usepackage[T1]{fontenc}			% per ottenere una composizione con font che contenga
							% lettere accentate e simboli pi� usati in Europa
\usepackage{geometry}			% per impostazioni di margini, larghezza testo, ecc.
\usepackage[final]{listings}

\usepackage{alp}								
%\setcounter{secnumdepth}{3}			% numera anche \subsubsection
%\setcounter{tocdepth}{3}				% fa apparire nell'indice anche le \subsubsection

% Impostazioni larghezza note a margine e larghezza testo
\marginparwidth= 60pt
\textwidth= 440pt



% Togliere il segno di commento dalle seguenti due righe
% per la compilazione con latex2html
%\renewcommand{\numnameref}[1]{\ref{#1}}
%\renewcommand{\qnameref}[1]{\ref{#1}}


\author{Simone Sollena}
\title{\Huge{\textbf{Programmazione avanzata su Linux}\\ Traduzione italiana dell'opera\\
\emph{Advanced Unix Programming}\\di\\Mark Mitchell, Jeffrey Oldham, Alex Samuel}}
\frenchspacing

\newcommand{\firstsentence}[2]
{\Huge\textsf{\textbf{#1}}\normalsize\textsc{#2}}


\begin{document} 
\maketitle
\tableofcontents

\chapter{Programmazione UNIX avanzata con Linux}\label{cap:1}

\firstsentence{Q}{uesto capitolo ci mostra come effettuare i passi di base} richiesti per creare un programma
Linux in C o C++. In particolare, questo capitolo ci mostra come creare e modificare del codice
sorgente C e C++, compilare questo codice, ed effettuare il debug del risultato. Se si � gi�
abituati a programmare su Linux, si pu� andare avanti al \numnameref{cap:2}; si presti molta attenzione alla \numnameref{sec:2.3}, in
particolare riguardo al linking statico Vs. Dinamico che potresti non ancora conoscere.

Nel corso di questo libro, assumeremo che tu abbia familiarit\`a con i linguaggi di programmazione
C e C++. Assumeremo anche che tu sappia come eseguire semplici operazioni da riga di comando
shell di Linux, come creare directory e copiare files. Poich� molti programmatori linux hanno
cominciato a programmare in ambiente Windows, mostreremo occasionalmente similitudini e
contrasti tra Windows e Linux.

\section{Editare con \emph{Emacs}}\label{sec:1.1}

Un editor � il programma che si usa per editare il codice sorgente. Molti dei vari editor sono
disponibili per Linux, ma l'editor pi� popolare e maggiormente pieno di funzionalit� �
probabilmente  {\tt Gnu Emacs}.
\begin{quote}
{\large\textbf{ Riguardo Emacs}}\\
Emacs � molto pi� che un editor. � un programma incredibilmente potente, tanto che al
CodeSourcery, � amichevolmente conosciuto come Un Vero Programma (One True Program)
o giusto OTP per abbreviare. Da Emacs � possibile leggere ed inviare email ed � possibile
personalizzarlo ed estenderlo in numerosissimi modi, troppi per poterne parlare qui. Da Emacs
� possibile persino navigare su web!
\end{quote}

Se hai familiarit� con un altro editor, puoi certamente usare quello. Nulla nel resto di questo libro
dipende dall'utilizzo di Emacs. Se invece non hai ancora un editor preferito su linux, dovresti
andare avanti con il mini-tutorial fornito qui di seguito.

Se ti piace Emacs e vuoi imparare qulcosa in pi� riguardo le sue funzionalit� avanzate, dovresti
tenere in considerazione il fatto di leggere uno dei libri su Emacs disponibili. Un eccellente Tutorial
� {\emph{Learning GNU Emacs}}, scritto da Debra Cameron, Bill Rosenbaltt, e Eric S. Raymond (O'Reilly, 1996).

\subsection{Aprire un file sorgente C o C++}\label{subsec:1.1.1} % 1.1.1

� possibile avviare Emacs scrivendo {\tt emacs} nella propria finestra di terminale e premendo il
tasto invio. Quando Emacs � avviato, � possibile usare il men� in alto per creare un nuovo file
sorgente. Clicca sul men� Files, scegli Apri Files, e quindi scrivi il nome del file che vuoi aprire
nel ``minibuffer'' in basso allo schermo.\footnote{Se non stai lavorando su un sistema X Window,
devi premere F10 per accedere ai men�} Se vuoi creare un file sorgente C, usa un nome di file
che finisca in {\tt.c} o in {\tt.h}. Se vuoi creare un file sorgente C++, usa un nome di file che finisca
in {\tt.cpp}, {\tt.hpp}, {\tt.cxx}, {\tt.hxx}, {\tt.C} o {\tt.H}. Quando il file � aperto, puoi scrivere cos� come
faresti in un programma di word-processing. Per salvare il file, scegli la voce Save Buffer nel
menu Files. Quando avrai finito di usare Emacs, potrai scegliere l'opzione Exit Emacs nel men� Files.

Se non ti piace puntare e cliccare, puoi usare le scorciatoie da tastiera per aprire
automaticamente i files, salvare i files, ed uscire da Emacs. Per aprire un file digita {\tt{C-x C-f}}
({\tt{C-x}} significa premere il tasto Control e quindi premere il tasto x). Per salvare un file digita {\tt{C-x
C-s}}. Per uscire da Emacs digita {\tt{C-x C-c}}. Se vuoi prendere un po' pi� confidenza con
Emacs scegli la voce ``Emacs Tutorial'' nel men� Help. Il tutoial ti fornisce alcuni consigli su
come usare Emacs efficientemente.

\subsection{Formattazione automatica}\label{subsec:1.1.2} % 1.1.2

Se sei abituato alla programmazione in un Ambiente di Sviluppo Integrato (Integrated
Development Environment \--- IDE), sarai anche abituato ad avere un editor che ti aiuti a
formattare il codice. Emacs pu� fornire lo stesso tipo di funzionalit�. Se apri un file sorgente C
o C++, Emacs automaticamente mette in risalto il fatto che il file contiene del codice sorgente, e
non del semplice testo. Se premi il tasto Tab in una riga vuota, Emacs muove il cursore in un
punto di indentazione appropriato. Se premi il tasto di indentazione in una riga che contiene
gi� del testo, Emacs indenta il testo. Cos�, per esempio, supponi di aver scritto quanto segue:
\begin{listcodeC}
int main ()
{
printf ("Hello, world\n");
}
\end{listcodeC}

Se premi il tasto Tab sulla linea con la chiamata a {\tt{printf}}, Emacs riformatter� il tuo codice
cosicch� sar� simile a questo:
\begin{listcodeC}
int main ()
{
  printf ("Hello, world\n");
}
\end{listcodeC}
Nota come la linea � stata appropriatamente indentata.

Non appena userai Emacs ancora per un po', vedrai come ti torner� utile nell'esecuzione di
tutti i tipi di compiti di formattazione complicati. Se sei ambizioso, puoi programmare Emacs per
eseguire letteralmente ogni tipo di formattazione automatica immaginabile. Tanta gente ha
usato questo tipo di funzionalit� per implementare modelli per Emacs per editare proprio ogni
tipo di documento, per implementare giochi\footnote{Prova ad eseguire il comando {\tt{M-x dunnet}} se
vuoi giocare ad un vecchio affascinante gioco di avventura su testo} e per implementare front
ends di database.

\subsection{Evidenziazione della sintassi}\label{subsec:1.1.3} % 1.1.3

Oltre alla formattazione del codice, Emacs pu� rendere facile leggere codice C e C++
colorando diversi elementi sintattici. Per esempio, Emacs pu� far apparire parole chiave di un
colore, tipi di dati ``built-in'', come \texttt{int}, in un altro colore, e commenti ancora di un altro colore.
L'utilizzo dei colori rende un po' pi� facile scoprire i comuni errori di sintassi.

Il modo pi� facile per far apparire la colorazione � editare il file \texttt{$\sim$/.emacs} ed inserire la
seguente stringa:
\begin{listcodeBash}
(global-font-lock-mode t)
\end{listcodeBash}

Salva il file, esci da Emacs, e riavvia. Adesso apri un file sorgente C o C++ e divertiti!

Potresti aver notato che la stringa che hai inserito nel tuo \texttt{.emacs} somiglia al codice di
linguaggio di programmazione LISP. Ci� avviene perch� questo \textit{�} codice LISP! Molto di Emacs
� attualmente scritto in LISP. � possibile aggiungere funzionalit� a Emacs scrivendo altro
codice LISP.

\section{Compilare con GCC}\label{sec:1.2} % 1.2

Un compilatore trasforma codice sorgente leggibile dall'uomo in codice oggetto leggibile
da macchina che pu� girare automaticamente. I compilatori di scelta sui sistemi Linux sono tutti
parti del GNU Compiler Collection, solitamente conosciuti come GCC.\footnote{Per ulteriori
informazioni su GCC, visita http://gcc.gnu.org} GCC include anche compilatori per C, C++,
Java, C ad oggetti, Fortran, e Chill. Questo libro � principalmente focalizzato sulla
programmazione C e C++.

Supponi di avere un progetto come quello del listato 1.2 con un file sorgente C++
(\texttt{reciprocal.cpp}) e un file sorgente C (\texttt{main.c}) come
nel listato \ref{list:1.1}. Di questi due file si
supponga che vengano compilati e quindi linkati assieme per generare un programma chiamato
\texttt{recipocal}.\footnote{In Windows gli eseguibili solitamente hanno nomi che finisono in \texttt{.exe}. I
programmi di Linux, d'altro canto, solitamente non hanno estensioni. Cos�, l'equivalente di
Windows di questo programma sarebbe chiamato probabilmente \texttt{reciprocal.exe}; la
versione Linux � semplicemente \texttt{reciprocal}}
Questo programma calcoler� il reciproco di un intero.\\


\listfromfile{main.c}{file sorgente C}{list:1.1}
	{ALP-listings/chapter-1/main.c}

\listfromfile{reciprocal.cpp}{File sorgente C++}{list:1.2}
	{ALP-listings/chapter-1/reciprocal.cpp}

C'� anche un file header chiamato \texttt{reciprocal.hpp} (vedi listato 1.3).\\

\listfromfile{reciprocal.hpp}{file header}{list:1.3}
	{ALP-listings/chapter-1/reciprocal.hpp}

Il primo passo � trasformare il codice sorgente C e C++ in codice oggetto.

\subsection{Compilare un singolo file sorgente}\label{subsec:1.2.1} % 1.2.1

Il nome del compilatore C � gcc. Per compilare un file sorgente C, usa l'opzione \texttt{-c}. Cos�,
per esempio, l'inserimento di quanto segue, al prompt dei comandi, avvia la compilazione
del file sorgente \texttt{main.c}

\begin{listcodeBash}
% gcc -c main.c
\end{listcodeBash}

Il file oggetto risultante � chiamato \texttt{main.o}.

Il compilatore C++ � chiamato \texttt{g++}. Le sue operazioni sono molto simili a \texttt{gcc}; � possibile compilare
\texttt{reciprocal.cpp} digitando quanto segue:

\begin{listcodeBash}
% g++ -c reciprocal.cpp
\end{listcodeBash}

L'opzione \texttt{-c} dice a \texttt{g++} di compilare il programma solo in un file oggetto; senza questa,
\texttt{g++} si aspetterebbe di dover fare il link del programma per produrre un eseguibile. Dopo aver digitato
questo comando, otterrai un file oggetto chiamato \texttt{reciprocal.o}.

Probabilmente avrai bisogno di un altro paio di opzioni per compilare ogni programma
ragionevolmente grande. L'opzione \texttt{-I} � usata per dire a GCC dove cercare i files header. Di
default GCC cerca nella directory corrente e nelle directory dove sono installati gli headers per
le librerie standard. Quindi se hai bisogno di includere dei files header che si trovano da qualche
altra parte servir� l'opzione \texttt{-I}. Supponiamo ad esempio che il tuo progetto abbia una
directory per i files sorgenti chiamata \texttt{src} ed un'altra chiamata \texttt{include}. Dovresti compilare
\texttt{reciprocal.cpp} come in questo esempio per indicare che \texttt{g++} deve usare, in
aggiunta, la directory \texttt{../include} per trovare \texttt{reciprocal.hpp}:

\begin{listcodeBash}
% g++ -c -I ../include reciprocal.cpp
\end{listcodeBash}

A volte si vogliono definire delle macro nella riga di comando. Per esempio, nel codice
di produzione, non si vuole il sovraccarico dovuto al controllo delle asserzioni, presente in \texttt{reciprocal.cpp};
questo serve solo a fare il debug del programma. \`E possibile disattivare tale controllo
definendo la macro \texttt{NDEBUG}. Dovresti aggiungere un \texttt{\#define} esplicito per
\texttt{reciprocal.cpp}, ma ci� richiederebbe modifiche allo stesso codice sorgente. � pi� semplice
definire piuttosto
\texttt{NDEBUG} da riga di comando, come in questo caso:

\begin{listcodeBash}
% g++ -c -D NDEBUG recipocal.cpp
\end{listcodeBash}

Se hai voluto definire \texttt{NDEBUG} con alcuni valori particolari, potresti aver fatto qualcosa del
genere:

\begin{listcodeBash}
% g++ -c -D NDEBUG=3 reciprocal.cpp
\end{listcodeBash}

Se si sta compilando il codice per la produzione finale, probabilmente si vuole che GCC ottimizzi il codice
cosicch� giri il pi� veloce possibile. Ci� lo si pu� fare utilizzando l'opzione \texttt{-O2} da riga di
comando. (GCC ha molti livelli di ottimizzazione; il secondo livello � il pi� appropriato per
la maggior parte dei programmi.) Per esempio, il seguente, compila \texttt{reciprocal.cpp} con l'ottimizzazione
attivata:

\begin{listcodeBash}
% g++ -c -O2 reciprocal.cpp
\end{listcodeBash}

Nota che compilare con l'ottimizzazione pu� rendere il programma molto difficile da
debuggare con un debugger (vedi \numnameref{sec:1.4}). Inoltre, in certi casi,
compilare con l'ottimizzazione pu� far venire fuori dei bugs che il programma non ha
manifestato precedentemente.

Si possono passare molte altre opzioni a \texttt{gcc} e \texttt{g++}. Il miglior modo per ottenere una lista
completa � vedere la documentazione online. Lo si pu� fare digitando quanto segue al prompt dei
comandi:

\begin{listcodeBash}
% info gcc
\end{listcodeBash}

\subsection{Fare il link di file oggetto}\label{subsec:1.2.2} % 1.2.2

Adesso che hai compilato \texttt{main.c} e \texttt{utilities.cpp}, vorrai fare il link. Per fare
il link di un programma che contiene codice C++ si dovrebbe usare sempre \texttt{g++},
anche se contiene pure
del codice C. Se il programma contiene solo codice C, si dovrebbe usare invece \texttt{gcc}. Poich�
questo programma contiene sia codice C che C++, si dovrebbe usare \texttt{g++} in questo modo:

\begin{listcodeBash}
% g++ -o reciprocal main.o reciprocal.o
\end{listcodeBash}

L'opzione \texttt{-o} fornisce il nome del file da generare come output del passaggio di link.
Adesso puoi far girare \texttt{reciprocal} come in questo esempio:

\begin{listcodeBash}
% ./reciprocal 7
The reciprocal of 7 is 0.142857
\end{listcodeBash}

Come puoi notare, \texttt{g++} ha automaticamente fatto il link con la libreria standard C di runtime
contenente l'implementazione di printf. Se avessi avuto bisogno di fare il link con un'altra libreria
(come una graphical user interface toolkit), avresti dovuto specificare la libreria tramite
l'opzione \texttt{-l}. In Linux, i nomi di librerie quasi sempre cominciano con \texttt{lib}. Per esempio, la
libreria Pluggable Authentication Module (PAM) � chiamata \texttt{libpam.a}. Per fare il link in
\texttt{libpam.a}, usa un comando come questo:

\begin{listcodeBash}
% g++ -o reciprocal main.o reciprocal.o -lpam
\end{listcodeBash}

Il compilatore aggiunge automaticamente il prefisso \texttt{lib} ed il suffissto \texttt{.a}.

Come per i files header, il linker cerca le librerie in alcune locazioni standard, includendo le
directory \texttt{/lib} e \texttt{/usr/lib} che contengono le librerie standard di sistema. Se vuoi che il
linker cerchi in altre directory, dovresti usare l'opzione \texttt{-L}, che � la ``parallela'' dell'opzione \texttt{-I}
appena discussa. Puoi usare questa riga per dire al linker di cercare le librerie nella directory
\texttt{/usr/local/lib/pam} prima di cercare nei soliti posti:

\begin{listcodeBash}
% g++ -o reciprocal main.o reciprocal.o
                -L/usr/local/lib/pam -lpam
\end{listcodeBash}

Bench� non ci sia bisogno di usare l'opzione \texttt{-I} per dire al preprocessore di cercare nella
directory corrente, � necessario usare l'opzione \texttt{-L} per dire al linker di cercare nella directory
corrente. In particolare, per dire al linker dove trovare la
libreria test nella directory corrente, dovresti usare la seguente istruzione:

\begin{listcodeBash}
% gcc -o app app.o -L. -ltest
\end{listcodeBash}

\section{Automatizzare i processi con GNU Make}\label{sec:1.3} % 1.3

Se sei abituato a programmare per il sistema operativo Windows, sei probabilmente abituato
a lavorare con un Integrated Developmente Environment (IDE). Aggiungi i files sorgenti al
progetto e quindi l'IDE compila il tuo progetto automaticamente. Bench� ci siano degli IDE
disponibili per Linux, in questo libro non ne parleremo. Invece, questo libro ti mostrer� come
usare GNU Make per automatizzare la ricompilazione del codice, che � ci� che fanno molti dei
programmatori Linux.

L'idea di base che sta dietro a \texttt{make} � semplice. Dici a \texttt{make} quali \textit{obiettivi} vuoi compilare,
quindi fornisci delle \textit{regole} spiegando come compilarli. Specifichi inoltre le dipendenze che
indicano quando un particolare obiettivo dovrebbe essere ricompilato.

Nel nostro progetto di esempio \texttt{reciprocal} ci sono tre obiettivi ovvi: \texttt{reciprocal.o},
\texttt{main.o} ed il \texttt{reciprocal} stesso. Hai gi� le regole in mente per compilare questi obiettivi
sotto forma di linea di comando, dati precedentemente. Le dipendenze richiedono qualche
piccolo ragionamento in pi�. Chiaramente, \texttt{reciprocal} dipende da \texttt{reciprocal.o} e
\texttt{main.o} poich� non � possibile fare il link del programma completo fino a che non hai
compilato ognuno di questi file oggetto. I files oggetto vanno ricompilati ogni qualvolta il
corrispondente file sorgente cambia. C'� un altro lato da tenere in considerazione: un
cambiamento a \texttt{reciprocal.hpp} pu� causare il dover ricompilare entrambi i files oggetto
poich� entrambi i files sorgenti includono quel file header.
\begin{sloppypar}
In aggiunta agli obiettivi ovvi, dovrebbe esserci sempre un obiettivo \texttt{clean}. Quest'obiettivo
rimuove tutti i files oggetto e programmi generati cosicch� alla fine si abbia tutto pulito. La
regola per questo obiettivo usa il comando rm per rimuovere i files.
\end{sloppypar}
Puoi dare tutte queste informazioni a make mettendo le informazioni in un file chiamato
\texttt{Makefile}. Ecco ci� che � contenuto nel \texttt{Makefile}

\begin{listcodeBash}
reciprocal: main.o reciprocal.o
    g++ $(CFLAGS) -o reciprocal main.o reciprocal.o

main.o: main.c reciprocal.hpp
    gcc $(CFLAGS) -c main.c

reciprocal.o: reciprocal.cpp reciprocal.hpp
    g++ $(CFLAGS) -c reciprocal.cpp

clean:
    rm -f *.o reciprocal
\end{listcodeBash}

Puoi vedere che gli obiettivi sono elencati sulla sinistra, seguiti da due punti e quindi ogni
dipendenza. La regola per compilare quell'obiettivo sta sulla linea successiva. (Per il momento
ignora il \texttt{\$(CFLAGS)}.) La riga contenente la regola deve iniziare con un
carattere di tabulazione, altrimenti make far� confusione. Se editi il \texttt{Makefile} in Emacs,
Emacs ti aiuter� con la formattazione.

Se rimuovi i files oggetto che hai appena compilato, e digiti

\begin{listcodeBash}
% make
\end{listcodeBash}

su riga di comando, vedrai quanto segue:

\begin{listcodeBash}
% make
gcc -c main.c
g++ -c reciprocal.cpp
g++ -o reciprocal main.o reciprocal.o
\end{listcodeBash}

Puoi notare che \texttt{make} ha compilato automaticamente i file oggetto e fatto il link tra loro.
Se adesso cambi \texttt{main.c} in qualche modo, e digiti di nuovo \texttt{make}, vedrai
quanto segue:

\begin{listcodeBash}
% make
gcc -c main.c
g++ -o reciprocal main.o reciprocal.o
\end{listcodeBash}

Puoi notare che \texttt{make} ha saputo compilare \texttt{main.o} e rifare il link del programma, ma non
si � preoccupato di ricompilare \texttt{reciprocal.cpp} poich� nessuna delle dipendenze per
\texttt{reciprocal.o} � stata cambiata.

Il \texttt{\$(CFLAGS)} � una variabile di make. E' possibile definire questa variabile sia nello stesso
\texttt{Makefile} che su riga di comando. \texttt{GNU make} sostituir� il valore della variabile durante
l'esecuzione della regola. Cos�, per esempio, per ricompilare con l'ottimizzazione attivata,
dovresti fare ci�:

\begin{listcodeBash}
% make clean
rm -f *.o reciprocal
% make CFLAGS=-O2
gcc -O2 -c main.c
g++ -O2 -c reciprocal.cpp
g++ -O2 -o reciprocal main.o reciprocal.o
\end{listcodeBash}

Nota che il flag \texttt{-O2} � stato inserito nelle regole al posto di \texttt{\$(CFLAGS)}.

In questa sezione hai visto solo le capacit� pi� semplici di make. Puoi trovare altro
digitando ci�:

\begin{listcodeBash}
% info make
\end{listcodeBash}

Nel manale, troverai informazioni su come \texttt{make} riesca a mantenere un \texttt{Makefile} semplice,
come ridurre il numero di regole necessarie, e come calcolare automaticamente
le dipendenze. Inoltre puoi trovare altre informazioni in \textit{GNU, Autoconf, Automake, and Libtool}
di Gary V.Vaughan, Ben Elliston,Tom Tromey, e Ian Lance Taylor (New Riders Publishing,
2000).

\section{Debuggare con GNU Debugger (GDB)}\label{sec:1.4} % 1.4

Il debugger � il programma che si usa per scoprire perch� il proprio programma non si sta
comportando nel modo in cui pensi dovrebbe comportarsi. Lo farai molto.\footnote{...finch� il tuo
porgramma non funzioner� per la prima volta} GNU Debugger (\texttt{GDB}) � il debugger usato da
molti programmatori Linux. puoi usare \texttt{GDB} per vedere il tuo codice passo passo, impostare dei
breakpoints ed esaminare il valore di variabili locali.

\subsection{Compilare con Informazioni di Debugging}\label{subsec:1.4.1} % 1.4.1

Per usare \texttt{GDB}, � necessario compilare con le informazioni di debugging abilitate. Per fare
questo si aggiunge lo swtich \texttt{-g} sulla riga di comando di compilazione. Se stai usando un
\texttt{Makefile} come descritto precedentemente, puoi giusto settare il \texttt{CFLAGS} uguale a \texttt{-g}
quando esegui \texttt{make}, come mostrato qui:

\begin{listcodeBash}
% make CFLAGS=-g
gcc -g -c main.c
g++ -g -c reciprocal.cpp
g++ -g -o reciprocal main.o reciprocal.o
\end{listcodeBash}

Quando compili con \texttt{-g}, il compilatore include informazioni extra nei files oggetto e negli
eseguibili. Il debugger usa queste informazioni per mostrare quali indirizzi corriposndono a
quali linee nel codice sorgente, come stampare variabili locali, e cos� via

\subsection{Eseguire GDB}\label{subsec:1.4.2} % 1.4.2

Puoi avviare \texttt{gdb} digitando:

\begin{listcodeBash}
% gdb reciprocal
\end{listcodeBash}

Quando \texttt{gdb} si avvia, dovresti vedere il prompt di \texttt{GDB}:

\begin{listcodeBash}
(gdb)
\end{listcodeBash}

Il primo passo � quello di eseguire il tuo programma all'interno del debugger, Inserisci giusto
il comando run ed ogni argomento del programma. Prova ad eseguire il programma senza
nessun argomento, come questo:

\begin{listcodeBash}
(gdb) run
Starting program: reciprocal
Program received signal SIGSEGV, Segmentation fault.
__strtol_internal (nptr=0x0, endptr=0x0, base=10, group=0)
at strtol.c:287
287 strtol.c: No such file or directory.
(gdb)
\end{listcodeBash}

Il problema � che non c'� controllo di errore nel codice \texttt{main}. Il programma si aspetta un
argomento, ma in questo caso il programma � stato eseguito senza argomento. Il messaggio
\texttt{SEGSEV} indica un crash del programma. \texttt{GDB} sa che il crash attuale � avvenuto in una funzione
chiamata \texttt{\_\_strtol\_interna}l. Questa funzione � nella libreria standard, e il sorgente non
� installato, come spiega il messaggio ``No such file or directory''. E' possibile vedere lo stack
usando il comando where:

\begin{listcodeBash}
(gdb) where
#0 __strtol_internal (nptr=0x0, endptr=0x0, base=10, group=0)
at strtol.c:287
#1 0x40096fb6 in atoi (nptr=0x0) at ../stdlib/stdlib.h:251
#2 0x804863e in main (argc=1, argv=0xbffff5e4) at main.c:8
\end{listcodeBash}

Puoi vedere da questa schermata che il main ha chiamato la funzione \texttt{atoi} con un puntatore
\texttt{NULL}, che � ci� che fa nascere i problemi.

Usando il comando
\texttt{up} puoi salire di due livelli nello stack per raggiungere il \texttt{main}:

\begin{listcodeBash}
(gdb) up 2
#2 0x804863e in main (argc=1, argv=0xbffff5e4) at main.c:8
8 i = atoi (argv[1]);
\end{listcodeBash}

Nota che \texttt{gdb} � capace di trovare il sorgente per \texttt{main.c} e mostra la riga dove � avvenuta
la chiamata alla funzione dell'errore. E' possibile anche vedere il valore di variabili usando il
comando \texttt{print}:

\begin{listcodeBash}
(gdb) print argv[1]
$2 = 0x0
\end{listcodeBash}

Che conferma che il problema � certamente dovuto ad un puntatore \texttt{NULL} passato ad \texttt{atoi}.

Puoi impostare un breakpoint usando il comando \texttt{break}:

\begin{listcodeBash}
(gdb) break main
Breakpoint 1 at 0x804862e: file main.c, line 8.
\end{listcodeBash}

Questo comando imposta un punto di interruzione (breakpoint) sulla prima linea del \texttt{main}.\footnote{Certa gente ha
commentato che dire di interrompere il \texttt{main} � un po' bizzarro perch� solitamente questo
lo si vuol fare solo quando il \texttt{main} � gi� interrotto}
Adesso prova ad
eseguire il programma con un argomento, come questo:

\begin{listcodeBash}
(gdb) run 7
Starting program: reciprocal 7
Breakpoint 1, main (argc=2, argv=0xbffff5e4) at main.c:8
8 i = atoi (argv[1]);
\end{listcodeBash}

Puoi notare che il debugger si � fermato al breakpoint.

Puoi andare avanti dopo la chiamata ad \texttt{atoi} usando il comando \texttt{next}:

\begin{listcodeBash}
(gdb) next
9 printf (?The reciprocal of %d is %g\n?, i, reciprocal (i));
\end{listcodeBash}

Se vuoi vedere cosa sta accadendo dentro \texttt{reciprocal}, usa il comando \texttt{step} come
questo:

\begin{listcodeBash}
(gdb) step
reciprocal (i=7) at reciprocal.cpp:6
6 assert (i != 0);
\end{listcodeBash}

Adesso sei nel corpo della funzione \texttt{reciprocal}.

Potresti trovare molto conveniente eseguire \texttt{gdb} dall'interno di Emacs piuttosto che usare
\texttt{gdb} da riga di comando. Usa il comando \texttt{M-x gdb} per avviare \texttt{gdb} in una finestra di Emacs.
Se vieni fermato ad un breakpoint, Emacs tira fuori automaticamente il file appropriato. � facile
capire cosa sta accadendo quando dai uno sguardo all'intero file piuttosto che giusto una riga di
testo.

\section{Trovare altre informazioni}\label{sec:1.5} % 1.5

Praticamente ogni distribuzione di Linux � provvista di una grande quantit� di
documentazione utile. Puoi imparare molto di ci� di cui parleremo in questo libro leggendo la
documentazione della tua distribuzione Linux (anche se ti occuper� molto tempo). La
documentazione non � sempre ben organizzata, per cui, il compito pi� difficile � quello di
trovare ci� di cui hai bisogno. La documentazione � anche spesso datata, quindi prendi tutto ci�
che leggi \textit{``con le pinze''}. Se per esempio il sistema non si comporta esattamente come una \textit{man
page} (pagine di manuale) dice che dovrebbe comportarsi pu� darsi che la pagina
di manuale sia obsoleta.

Per aiutarti a navigare, qui ci sono le pi� utili fonti di informazioni riguardo la
programmazione Linux avanzata.

\subsection{Pagine Man}\label{subsec:1.5.1}

Le distribuzioni Linux includono pagine di manuale per la maggior parte dei comandi
standard, chiamate di sistema, e funzioni di librerie standard. Le pagine di manuale sono divise
in sezioni numerate; per i programmatori, le pi� importanti sono queste:

\begin{itemize}
\item (1) Comandi utente
\item (2) Chiamate di sistema
\item (3) Funzioni di librerie standard
\item (8) comandi di sistema/amministrativi
\end{itemize}

I numeri indicano sezioni di pagine di manuale. Le pagine di manuale di Linux sono
installate nel tuo sistema; usa il comando \texttt{man} per accedere ad esse. Per trovare una pagina di
manuale digita semplicemente \texttt{man nome}, dove \texttt{nome} � il nome di un comando o di una
funzione. In alcuni casi lo stesso nome si trova in pi� di una sezione; puoi specificare
esplicitamente la sezione anteponendo il numero della sezione al nome. Per esempio, se digiti quanto segue, otterrai la pagina di manuale per il comando \texttt{sleep} (nella sezione 1 delle pagine di
manuale di Linux):

\begin{listcodeBash}
% man sleep
\end{listcodeBash}

Per vedere la pagina di manuale per la funzione di libreria sleep, usa questo comando:

\begin{listcodeBash}
% man 3 sleep
\end{listcodeBash}

Ogni pagina di manuale include un sommario on-line del comando o funzione. Il comando
\texttt{whatis nome} mostra tutte le pagine di manuale (in tutte le sezioni) per un comando o una
funzione corrispondente a quel \texttt{nome}. Se non sei sicuro di quale comando o funzione vuoi, puoi
effettuare una ricerca per parola chiave sulle linee di sommario delle pagine di manuale, usando
\texttt{man -k parolachiave}.

Le pagine di manuale includono una grande quantit� di informazioni molto utili e
dovrebbero essere il primo posto nel quale cercare aiuto. La pagina di manuale per un comando
descrive opzioni per la linea di comando e argomenti, input e output, codici di errori,
configurazioni, e altre cose simili. La pagina di manuale per una chiamata di sistema o una
funzione di libreria descrive parametri e valori di ritorno, lista di codici di errori ed effetti
collaterali, e specifica quali files include usare per chiamare la funzione.

\subsection{Info}\label{subsec:1.5.2} % 1.5.2

Il sistema di documentazione Info contiene documentazione pi� dettagliata per molti
componenti del cuore del sistema GNU/Linux, pi� molti altri programmi. Le pagine Info sono
documenti ipertestuali, simili a pagine web. Per lanciare il browser Info basato su testo, digita
\texttt{info} in una nuova finestra di shell. Ti verr� presentato un men� di documenti Info installati
nel sistema. (Premi Control+H per visualizzare i tasti per navigare all'interno di un documento
Info).

Tra i pi� utili documenti Info ci sono questi:

\begin{itemize}
\item \texttt{gcc} \--- Il compilatore gcc
\item \texttt{libc} \--- La libreria GNU C, che include molte chiamate di sistema
\item \texttt{gdb} \--- Il debugger GNU
\item \texttt{emacs} \--- L'editor di testo Emacs
\item \texttt{info} \--- Lo stesso sistema Info
\end{itemize}

Praticamente tutti gli strumenti standard di programmazione di Linux (inclusi \texttt{ld}, il linker;
\texttt{as}, l'assemblatore, e \texttt{gprof}, il profiler) sono corredati di utili pagine Info. Puoi saltare
direttamente ad un particolare documento Info specificando il nome della pagina da riga di
comando:

\begin{listcodeBash}
% info libc
\end{listcodeBash}

Se fai la maggior parte della tua programmazione in Emacs, puoi accedere direttamente al
broswer interno Info digitando \texttt{M-x info} o \texttt{C-h i}.

\subsection{File Header}\label{subsec:1.5.3} % 1.5.3

� possibile imparare molto sulle funzioni di sistema disponibili e su come usarle guardando i
files header del sistema. Questi risiedono in \texttt{/usr/include} e in \texttt{/usr/include/sys}. Se,
per esempio, ottieni degli errori di compilazione dall'utilizzo di una chiamata di sistema, dai
un'occhiata al file header corrispondente per verificare che la signature della funzione sia la
stessa a quello che � descritto nella pagina di manuale.

Sui sistemi linux, gran parte dei dettagli fondamenteli su come lavorano le chiamate di sistema si
riflette nei files header nelle directory \texttt{/usr/include/bits}, \texttt{/usr/include/asm}, e
\texttt{/usr/include/linux}. Per esempio, i valori numerici dei segnali (descritti nella
\numnameref{sec:3.3}, nel \numnameref{cap:3}) sono definiti in
\texttt{/usr/include/bits/signum.h}. Questi file header facilitano la lettura alle menti
curiose. Tuttavia, non includerli nel programa; usa sempre i file header in \texttt{/usr/include}
come menzionato nella pagina di manuale per la funzione che stai usando.

\subsection{Codice sorgente}\label{subsec:1.5.4} % 1.5.4

Siamo in ambito open source, no? Ci� che regola il modo in cui il sistema lavora alla fine � lo
stesso codice sorgente del sistema e fortunatamente per i programmatori Linux, quel codice
sorgente � gratuitamente disponibile. C'� la possibilit� che la tua distribuzione Linux includa
tutto il codice sorgente dell'intero sistema e di tutti i programmi inclusi in essa; se non � cos�,
sei autorizzato, secondo i termini della GNU General Public License, a richiederlo al
distributore. (Comunque, il codice sorgente potrebbe non essere stato installato sull'hard disk.
Guarda la documentazione della tua distribuzione per le istruzioni su come installarlo).

Il codice sorgente per lo stesso kernel di linux � di solito memorizzato in
\texttt{/usr/src/linux}. Se questo libro ti incuriosisce sui dettagli di come lavorano i processi, la
memoria condivisa e le periferiche di sistema, puoi sempre imparare direttamente dal codice
sorgente. Molte delle funzioni descritte in questo libro sono implementate con librerie GNU C;
controlla la documentazione della tua distribuzione per la posizione del codice sorgente delle
librerie C.

% Fine del capitolo 1
\chapter{Scrivere del buon software GNU/Linux}\label{cap:2}

QUESTO CAPITOLO COPRE ALCUNE TECNICHE DI BASE CHE USANO
MOLTI PROGRAMMATORI GNU/Linux.
Seguendo le linee guida presentate, sarai in grado di scrivere programmi che lavorano bene
nell'ambiente GNU/Linux ed andare incontro alle aspettative degli utente GNU/Linux su come
dovrebbero funzionare i programmi.

\section{Interazione con l'ambiente di esecuzione}\label{sec:2.1} % 2.1

Quando hai cominciato a studiare C o C++, avrai appreso che la funzione speciale \texttt{main} � il
punto di inizio primario di un programma. Quando il sistema operativo esegue il tuo
programma, esso fornisce automaticamente certe funzionalit� che aiutano il programma a
comunicare con il sistema operativo e con l'utente. Probabilmente avrai anche appreso riguardo
ai due parametri del \texttt{main}, solitamente chiamati \texttt{argc} e \texttt{argv}, che ricevono l'input per il tuo
programma. Hai appreso anche a riguardo dello \texttt{stdout} e \texttt{stdin} (o gli stream \texttt{cout} e \texttt{cin}
in C++) che forniscono funzioni di input e output per la console. Queste funzionalit� sono
fornite dai linguaggi C e C++, ed essi interagiscono in certi modi con il sistema operativo
GNU/Linux. GNU/Linux fornisce anche altri modi per interagire con l'ambiente operativo.

\subsection{La lista degli argomenti}\label{subsec:2.1.1}

Esegui un programma dal prompt di shell digitando il nome del programma. Opzionalmente,
puoi fornire al programma informazioni aggiuntive digitando una o pi� parole dopo il nome del
programma, separate da spazi. Queste sono chiamate \textit{argomenti da riga di comando} (\textit{command-line
arguments}). (Puoi anche includere un argomento che contiene spazi, racchiudendolo tra
virgolette). Pi� generalmente, si fa riferimento a questi anche come la \textit{lista degli argomenti} del
programma perch� non � necessario che essi abbiano origine dalla riga di comando di shell. Nel
\numnameref{cap:3}, vedrai un altro modo di chiamare un programma, nel quale un
programma pu� specificare direttamente la lista degli argomenti di un altro programma.

Quando un programma � chiamato da shell, la lista degli argomenti contiene l'intera linea di
comando, includendo il nome del programma ed ogni argomento della riga di comando che pu�
essere stato fornito. Supponi, per esempio, di chiamare il comando \texttt{ls} nella shell per mostrare il
contenuto della directory root e le corrispondenti dimensioni dei files con questa riga:

\begin{listcodeBash}
% ls -s /
\end{listcodeBash}

La lista di argomenti che il programma riceve ha tre elementi. Il primo � il nome del
programma stesso, come specificato dalla riga di comando, chiamato \texttt{ls}. Il secondo ed il terzo
elemento della lista degli argomenti sono due argomenti da riga di comando, \texttt{-s} e \texttt{/}.

La funzione \texttt{main} del tuo programma pu� accedere alla lista degli argomenti tramite i
parametri \texttt{argc} e \texttt{argv} del main (se non li usi, li puoi semplicemente omettere). Il primo
parametro, \texttt{argc}, � un intero che � settato al numero di elementi nella lista degli argomenti. Il
secondo parametro, \texttt{argv}, � un \texttt{array} di puntatori a caratteri. La dimensione dell'array �
argc e gli elementi dell'array puntano agli elementi della lista degli argomenti, come stringhe
di caratteri ``NUL-terminated''.

Usare gli argomenti da riga di comando � tanto facile quanto esaminare il contenuto di \texttt{argc}
e \texttt{argv}. Se non sei interessato al nome sesso del programma, non dimenticare di saltare il
primo elemento.

Il listato 2.1 dimostra come usare \texttt{argc} e \texttt{argv}.\\

\listfromfile{arglist.c}{Usare \texttt{argc} e \texttt{argv}}{list:2.1}
{ALP-listings/chapter-2/arglist.c}

\subsection{Convenzioni GNU/Linux della riga di comando}\label{subsec:2.1.2} % 2.1.2

Quasi tutti i programmi GNU/Linux obbediscono ad alcune convenzioni su come sono
interpretati gli argomenti della riga di comando. Gli argomenti che i programmi si aspettano
rientrano in due categorie: opzioni (o flag) e altri argomenti. Le opzioni modificano il modo in
cui si comporta il programa, mentre gli altri argomenti forniscono gli input (per esempio, i nomi
dei file di input).

Le opzioni sono di due forme:
\begin{itemize}
\item{opzioni brevi consistono in un singolo trattino ed un singolo carattere (di solito una
lettere minuscola o una lettera maiuscola). Le opzioni brevi sono veloci da digitare.}
\item{Opzioni lunghe consistono in due trattini, seguiti da un nome fatto di lettere
minuscole e maiuscole e trattini. Le opzioni lunghe sono facili da ricordare e facili da
leggere (ad esempio, negli script di shell).}
\end{itemize}

Di solito, un programma le fornisce entrambe; una forma breve ed una forma lunga per
molte delle opzioni che supporta, una per brevit� e l'altra per chiarezza. Per esempio, molti
programmi capiscono le opzioni \texttt{-h} e \texttt{-{}-{}help}, e le trattano in modo identico. Normalmente,
quando un programma � invocato dalla shell, ogni opzione che si vuole segue immediatamente
il nome del programma. Alcune opzioni si aspettano di essere seguite immediatamente da un
argomento. Molti programmi, per esempio, interpretano l'opzione \texttt{-{}-{}output foo} per
specificare che l'output del programma dovrebbe essere memorizzato in un file chiamato foo.
Dopo le opzioni, possono seguire altri argomenti di riga di comando, tipicamente files di input o
input di dati.

Per esempio, il comando \texttt{ls -s /} mostra il contenuto della directory root. L'opzione \texttt{-s}
modifica il comportamento di default di \texttt{ls} dicendogli di mostrare la dimensione (in kilobytes)
di ogni voce. L'argomento \texttt{/} dice a \texttt{ls} di quale directory fare il listato. L'opzione \texttt{-{}-size} �
sinonima di \texttt{-s}, cos� lo stesso comando pu� essere invocato come \texttt{ls -{}-size /}.

Gli standard di codifica GNU (\textit{GNU Coding Standards}) elencano i nomi di alcune opzioni di
riga di comando comunemente usati. Se prevedi di fornire ogni opzione simile a queste, sarebbe
una buona idea usare i nomi specificati negli standard di codifica. Puoi vedere le linee guida
degli standard di codifica GNU per le opzioni di riga di comando invocando il seguente da un
prompt di shell in molti sistemi GNU/Linux:

\begin{listcodeBash}
% info "(standards)User Interfaces"
\end{listcodeBash}

\subsection{Usare \texttt{getopt\_long}}\label{subsec:2.1.3} % 2.1.3

Analizzare le opzioni di riga di comando � un compito fastidioso. Fortunatamente, le librerie
GNU C forniscono una funzione che puoi usare nei programmi C e C++ per rendere questo
lavoro un po' pi� facile (anche se resta sempre un po' noioso). Questa funzione,
\texttt{getopt\_long}, capisce entrambe le opzioni, brevi e lunghe. Se usi questa funzione, includi il
file header \texttt{<getopt.h>}. Supponi, per esempio, di stare scrivendo un programma che accetta
le tre opzioni mostrate nella tabella 2.1.

\begin{table}[htdp]
\caption{Esempi di opzioni di programmi}
\begin{center}
\begin{tabular}{|l|l|l|}
\hline
{Forma breve}&{Forma lunga}&{Scopo}\\
\hline
{\texttt{-h}}&{\texttt{-{}-help}}&{Mostra un sommario d'uso ed esce}\\
{\texttt{-o nomefile}}&{\texttt{-{}-output nomefile}}&{Specifica il nome del file di output}\\
{\texttt{-v}}&{\texttt{-{}-verbose}}&{Stampa messaggi dettagliati}\\
\hline
\end{tabular}
\end{center}
\label{default}
\end{table}%

In pi�, il programma accetta zero o pi� argomenti da linea di comando aggiuntivi, che sono i
nomi dei files di input.

Per specificare le opzioni lunghe disponibili, puoi costruire un array di elementi \texttt{struct
option}

Per usare \texttt{getopt\_long}, devi fornire due strutture dati, La prima � una stringa di caratteri
che contiene le opzioni brevi valide, ognuna una singola lettera. Un'opzione che richiede un
argomento � seguita da due punti, Per il tuo programma, la stringa \texttt{ho:v} indica che le opzioni
valide sono \texttt{-h}, \texttt{-o} e \texttt{-v}, con la seconda di queste opzioni seguita da un argomento.

Per specificare le opzioni lunghe disponibili, puoi costruire un array di elementi \texttt{struct
option}. Ogni elemento corrisponde a una opzione lunga ed ha quattro campi. In circostanze
normali, il primo campo � il nome dell'opzione lunga (come una stringa di caratteri, senza i due
trattini); il secondo � 1 se l'opzione prende un argomento o 0 altrimenti; Il terzo � NULL; e il
quarto � un carattere costante che specifica l'opzione breve sinonima per quella lunga. L'ultimo
elemento dell'array dovrebbe essere tutto zeri. Puoi costruire un array come questo:

\begin{listcodeC}
const struct option long_options[] = {
  { "hel"?, 0, NULL, 'h' },
  { "output", 1, NULL, 'o' },
  { "verbose", 0, NULL, 'v' },
  { NULL, 0, NULL, 0 }
};
\end{listcodeC}

Invochi la funzione \texttt{getopt\_long}, passantole gli argomenti \texttt{argc} e \texttt{argv} dal main, la
stringa di caratteri che descrive le opzioni brevi e l'array di elementi \texttt{struct option}
descrive le opzioni lunghe.
\begin{itemize}
\item {Ogni volta che chiami \texttt{getopt\_long}, esso verifica una singola opzione,
restituendo la lettera dell'opzione breve per quell'opzione o \texttt{-1} se non sono trovate altre
opzioni.}
\item {Tipicamente, chiamerai \texttt{getopt\_long} in un \texttt{loop}, per processare tutte le
opzioni che l'utente ha specificato e tratterai le opzioni specifiche all'interno di uno
switch.}
\item {Se \texttt{getopt\_long} incontra una opzione non valida (un'opzione che non hai
specificato come una opzione breve o lunga valida), stampa un messaggio di errore e
restituisce il carattere \texttt{?} (un marcatore di domanda). Molti programmi escono come
risposta a questo, possibilmente dopo aver mostrato un messaggio di informazioni
sull'utilizzo.}
\item {Quando usi un'opzione che prende un argomento, la variabile globale \texttt{optarg} punta
al testo di quell'argomento.}
\item {Dopo che \texttt{getopt\_long} ha finito di elaborare tutte le opzioni, la variabile globale
\texttt{optind} contiene l'indice (in \texttt{argv}) del primo argomento che non sia un'opzione.}
\end{itemize}

Il listato 2.2 mostra un esempio di come dovresti usare \texttt{getopt\_long} per processare i tuoi
argomenti.\\

\listfromfile{getopt\_long.c}{Usare \texttt{getopt\_long}}{list:2.2}
	{ALP-listings/chapter-2/getopt_long.c}

Usare \texttt{getopt\_long} pu� sembrare un po' laborioso, ma scrivere del codice da soli per
analizzare le opzioni della riga di comando potrebbe richiedere anche pi� tempo. La funzione
\texttt{getopt\_long} � molto sofisticata e permette una grande flessibilit� nello specificare che tipo di
opzioni accettare. Comunque, � una buona idea starsene lontani dalle funzionalit� molto
avanzate e lavorare con le opzioni della struttura di base descritte.

\subsection{I/O Standard}\label{subsec:2.1.4} % 2.1.4

Le librerie C standard forniscono stream di input e output (rispettivamente \texttt{stdin} e
\texttt{stdout}). Queste sono usate da \texttt{scanf}, \texttt{printf} e altre funzioni di libreria. Nella tradizione
UNIX, l'uso di standard input ed output � d'abitudine per i programmi GNU/Linux. Ci�
permette di concatenare programmi multipli usando le pipes di shell e la ridirezione di input ed
output. (Vedi la pagina di manuale della tua shell per impararne la sintassi).

La libreria C fornisce anche lo \texttt{stderr}, lo stream standard error. I programmi dovrebbero
stampare warning e messaggi di errore sullo standard error invece dello standard output. Questo
permette agli utenti di separare output normali e messaggi di errore, per esempio,
redirezionando lo standard output ad un file mentre permettendo allo standard error di stampare
sulla console. La funzione \texttt{fprintf} pu� essere usata per stampare sullo \texttt{stderr}, per
esempio:

\begin{listcodeC}
fprintf (stderr, ("Error: ..."));
\end{listcodeC}

Questi tre stream sono anche accessibili con i comandi di base di UNIX I/O (read, write, e
cos� via) tramite i descrittori di file o pipe. La sintassi per fare ci� varia a seconda delle shell;
per le shell stile Bourne (inclusa bash, la shell di default in molte distribuzioni GNU/Linux), la
sintassi � questa:

\begin{listcodeBash}
% program > output_file.txt 2>&1
% program 2>&1 | filter
\end{listcodeBash}

La sintassi \texttt{2>\&1} indica che il descrittore di file 2 (\texttt{stderr}) dovrebbe essere unita al
descrittore di file 1 (\texttt{stdout}). Nota che \texttt{2>\&1} deve seguire una redirezione di file (il primo
esempio) ma deve precedere una ridirezione di pipe (il secondo esempio).

Nota che \texttt{stdout} � bufferizzato. I dati scritti sullo \texttt{stdout} non sono inviati alla console (o
altre periferiche, se � rediretto) fino a che il buffer non si riempie, finch� il programma esce
normalmente, o finch� stdout � chiuso. Puoi azzerare il buffer chiamando la seguente:

\begin{listcodeC}
fflush (stdout);
\end{listcodeC}

Di contro, \texttt{stderr} non � bufferizzato; i dati che sono scritti sullo \texttt{stderr} vanno
direttamente alla conole.\footnote{In C++ la stessa distinzione persiste, rispettivamente per \texttt{cout} e
\texttt{cerr}. Nota che il token \texttt{endl} svuota uno stream ed in pi� stampa un carattere di nuova riga;
se non vuoi svuotare lo stream (per esempio, per ragioni di performance), usa la costante di
nuova linea \texttt{'\textbackslash n'}}

Questo pu� produrre qualche risultato sorprendente. Per esempio, questo loop non stampa
una frase al secondo; invece, le frasi sono bufferizzate ed un po' di queste vengono stampate
assieme quando il buffer � pieno.

\begin{listcodeC}
while (1) {
    printf (".");
    sleep (1);
}
\end{listcodeC}

In questo loop, comunque, il punto compare una volta ogni secondo:

\begin{listcodeC}
while (1) {
    fprintf (stderr, ".");
    sleep (1);
}
\end{listcodeC}

\subsection{Codici di uscita del programma}\label{subsec:2.1.5} % 2.1.5

Quando un programma termina, indica il proprio stato con un codice di uscita. Il codice di
uscita � un piccolo numero intero; per convenzione, un codice di uscita pari a zero indica
un'esecuzione senza problemi, mentre un codice di uscita diverso da zero indica che c'� stato un
errore. Alcuni programmi usano differenti valori diversi da zero per i codici di uscita per
distinguere errori specifici.

Con molte shell � possibile ottenere i codici di uscita del programma eseguito pi�
recentemente con la variabile speciale \texttt{\$?}. Qui c'� un esempio in cui il comando \texttt{ls} viene
chiamato due volte ed i suoi codici di uscita sono stampati dopo ogni chiamata. Nel primo caso,
\texttt{ls} viene eseguito correttamente e restituisce il codice di uscita zero. Nel secondo caso, \texttt{ls}
restituisce un errore (perch� il nome del file specificato sulla riga di comando non esiste) e
quindi restituisce un codice di errore diverso da zero.

\begin{listcodeBash}
% ls /
bin coda etc lib misc nfs proc sbin usr
boot dev home lost+found mnt opt root tmp var
% echo $?
0
% ls bogusfile
ls: bogusfile: No such file or directory
% echo $?
1
\end{listcodeBash}

Un programma C o C++ specifica il proprio codice di uscita restituendo quel valore dalla
funzione \texttt{main}. Ci sono altri metodi usati per fornire codici di uscita, e codici di uscita speciali
sono assegnati al programma che non termina in modo normale (per un segnale). Di questi
si parler� pi� avanti nel \autoref{cap:3}.

\subsection{L'ambiente}\label{subsec:2.1.6} % 2.1.6

GNU/Linux fornisce ogni programma in esecuzione con un \textit{ambiente}. L'ambiente � una
raccolta di coppie variabile/valore. Entrambi, nomi delle variabili d'ambiente ed i loro valori
sono stringhe di caratteri. Per convenzione, i nomi delle variabili d'ambiente sono scritti con
tutte le lettere maiuscole.

Probabilmente avrai gi� familiarit� con molte delle comuni variabili d'ambiente. Per
esempio:

\begin{itemize}
\item USER Contiene il tuo nome utente.
\item HOME Coniene il percorso della tua home directory.
\item{PATH Contiene una lista di directory separate dai due punti nelle quali Linux cerca i comandi che invochi}
\item{DISPLAY Contiene il nome ed il numero di display del server X Window nel quale
appariranno le finestre dell'interfaccia grafica X Window.}
\end{itemize}

La tua shell, come altri programmi, ha un ambiente. Le Shell forniscono metodi per
esaminare e modificare direttamente l'ambiente. Per stampare l'ambiente corrente nella tua shell
invoca il programma \texttt{printenv}. Diverse shell hanno diverse sintassi per l'utilizzo delle
variabili d'ambiente; la seguente � la sintassi per le shell stile Bourne.

\begin{itemize}
\item{La shell crea automaticamente una variabile di shell per ogni variabile d'ambiente
che trova, cos� puoi accedere ai valori delle variabili d'ambiente usando la sintassi
\texttt{\$varname}. Per esempio:
\begin{listcodeBash}
% echo $USER
samuel
% echo $HOME
/home/samuel
\end{listcodeBash}}
\item{Puoi usare il comando \texttt{export} per esportare una variabile di shell nell'ambiente.
Per esempio, per settare la variabile d'ambiente \texttt{EDITOR}, dovresti usare questo:
\begin{listcodeBash}
% EDITOR=emacs
% export EDITOR
\end{listcodeBash}
O pi� brevemente
\begin{listcodeBash}
% export EDITOR=emacs
\end{listcodeBash}}
\end{itemize}

In un programma, puoi accedere ad una variabile d'ambiente con la funzione \texttt{getenv} in
\texttt{<stdlib.h>}. Questa funzione prende il nome di una variabile e ne restituisce il valore
corrispondente come stringa di caratteri, o NULL se la variabile non � definita nell'ambiente.
Per settare o cancellare variabili d'ambiente, usa rispettivamente le funzioni \texttt{setenv} e
\texttt{unsetenv}.

Enumerare tutte le variabili d'ambiente � un po' difficile. Per farlo puoi accedere ad una
speciale variabile globale chiamata \texttt{environ}, definita nella libreria GNU C. Questa variabile,
di tipo \texttt{char**}, � un array di puntatori a stringhe di caratteri che terminano con NULL. Ogni stringa
contiene una variabile d'ambiente, nella forma \texttt{VARIABILE=valore}.

Il programma del listato 2.3, per esempio, stampa semplicemente l'intero ambiente facendo
un loop nell'\texttt{array environ}.\\

\listfromfile{print-env.c}{Stampa l'ambiente di esecuzione}{list:2.3}
	{ALP-listings/chapter-2/print-env.c}

Non modificare \texttt{environ} tu stesso, ma piuttosto usa le funzioni \texttt{setevn} e \texttt{unsetenv}.

Di solito, quando viene avviato un nuovo programma, esso eredita una copia dell'ambiente
del programma che lo ha invocato (il programma shell, se � stato invocato interattivamente).
Cos�, per esempio, i programmi che esegui dalla shell, possono esaminare i valori delle variabili
d'ambiente che hai settato nella shell.

Le variabili d'ambiente sono comunemente usate per comunicare informazioni di
configurazione ai programmi. Supponi, per esempio, di star scrivendo un programma che si
collega ad un server internet per ottenere alcune informazioni. Potresti scrivere il programma in
modo tale che il nome del server sia specificato da riga di comando. Comunque, supponi che il
nome del server non sia qualcosa che l'utente cambia molto spesso. Puoi usare in questo caso
una speciale variabile d'ambiente \--- detta \texttt{SERVER\_NAME} \--- per specificare il nome del server;
se questa variabile non esiste, viene usato un valore di default. Una parte del tuo programma
potrebbe somigliare a quello mostrato nel listato 2.4.

\listfromfile{client.c}{Parte di un programma client di rete}{list:2.4}
	{ALP-listings/chapter-2/client.c}

Sopponi che il programma sia chiamato \texttt{client}. Assumendo che non hai settato la
variabile \texttt{SERVER\_NAME}, viene utilizzato il valore di default per il nome del server:

\begin{listcodeBash}
% client
accessing server server.my-company.com
\end{listcodeBash}

Ma � facile specificare un server diverso:

\begin{listcodeBash}
% export SERVER_NAME=backup-server.elsewhere.net
% client
accessing server backup-server.elsewhere.net
\end{listcodeBash}

\subsection{Usare files temporanei}\label{subsec:2.1.7} % 2.1.7

A volte un programma necessita di creare un file temporaneo, per memorizzare dati per un
po' o passare dati ad un altro programma. Sui sistemi GNU/Linux, i file temporanei sono
memorizzati nella directory /tmp. Quando usi i file temporanei, dovresti essere gi� informato
delle seguente "insidie":
\begin{itemize}
\item{Pi� di una istanza del tuo programma pu� essere eseguita simultaneamente (dallo
stesso utente o diversi utenti). Le istanze dovrebbero usare diversi nomi dei files
temporanei cosicch� non vadano in collisione.}
\item{I permessi dei file temporanei dovrebbero essere settati in modo che utenti non
autorizzati non possano alterare l'esecuzione del programma modificando o sostituendo
un file temporaneo.}
\item{I nomi dei file temporanei dovrebbero essere generati in modo tale che non possano
essere previsti esternamente; altrimenti un attaccante potrebbe sfruttare il ritardo nel
testing se un nome dato � gi� in uso ed aprire un nuovo file temporaneo.}
\end{itemize}

GNU/Linux fornisce delle funzioni, \texttt{mkstemp} e \texttt{tmpfile} che si curano da sole di questi
problemi al posto tuo (in aggiunta a molte funzioni che non lo fanno). Quale userai dipende dal
fatto che tu voglia passare il file temporaneo ad un altro programma e se vuoi usare le funzioni
I/O di UNIX (\texttt{open}, \texttt{write}, e cos� via) o le funzioni di stream I/O delle librerie C (\texttt{fopen},
\texttt{fprintf}, e cos� via).

\paragraph{Usare \textit{mkstemp}}

La funzione \texttt{mkstemp} crea un file temporaneo il cui nome � univoco da un modello di nomi
di file, crea il file con gli opportuni permessi cosicch� solo l'utente corrente vi possa accedere e
aprire il file per lettura/scrittura. Il modello del nome del file � una stringa di caratteri che
finisce con ``XXXXXX'' (sei X maiuscole); \texttt{mkstemp} sostituisce le X con dei caratteri in modo
tale che il nome del file sia univoco. Il valore di ritorno � un descrittore di file; a questo punto
usa le funzioni della famiglia \texttt{write} per scrivere sul file temporaneo.

I file temporanei creati con \texttt{mkstemp} non vengono cancellati automaticamente. E' compito tuo
rimuovere il file temporaneo quando non � pi� usato. (I programmatori dovrebbero fare
attenzione a cancellare i file temporanei; altrimenti il file system /tmp si riempir�, rendendo il
sistema non funzionante). Se il file temporaneo serve solo per l'utilizzo interno e non verr�
passato ad un altro programma, sarebbe una buona idea chiamare \texttt{unlink} nel file temporaneo
immediatamente. La funzione \texttt{unlink} rimuove la voce della directory corrispondente al file,
ma poich� i file in un file system hanno dei riferimenti, il file stesso non verr� rimosso fino a
che non ci saranno pi� dei descrittori di file per quel file. Per questi motivi il tuo programma
potr� continuare ad usare il file temporaneo, ed il file andr� via automaticamente non appena
chiuderai il descrittore del file. Poich� Linux chiude i descrittori di file quando un programma
termina, il file temporaneo verr� rimosso anche se il programma dovesse terminare in modo
imprevisto.

La coppia di funzioni nel listato 2.5 dimostra \texttt{mkstemp}. Usate insieme, queste funzioni
facilitano la scrittura di un buffer di memoria su un file temporaneo (in modo tale che la
memoria possa essere liberata o riutilizzata) e rileggerlo successivamente.\\

\listfromfile{temp\_file.c}{utilizzo di mkstemp}{list:2.5}
	{ALP-listings/chapter-2/temp_file.c}

\paragraph{Usare \textit{tmpfile}}

Se stai usando le funzioni di I/O della libreria C e non hai necessit� di passare il file
temporaneo ad un altro programma, puoi usare la funzione tmpfile. Questa crea ed apre un file
temporaneo, e restituisce un puntatore di file che punta a questo. Il file temporaneo � gi�
\textit{``unlinked''}, come nell'esempio precedente, cos� viene cancellato automaticamente quando il
puntatore al file viene chiuso (con \texttt{fclose}) o quando il programma termina.

GNU/Linux fornisce molte altre funzioni per generare file temporanei e nomi di file
temporanei, inclusi \texttt{mktemp}, \texttt{tmpnam} e \texttt{tempnam}. Comunque,
non usare queste funzioni poich�
sono soggette ai problemi di sicurezza gi� descritti.

\section{Codifica difensiva}\label{sec:2.2} % 2.2

Scrivere programmi che girano correttamente durante l'utilizzo ``normale'' � difficile;
scrivere programmi che si comportino decentemente in situazioni errori � ancora pi� difficile.
Questa sezione dimostra alcune tecniche di codifica per scovare velocemente i bug e per trovare
e riprendersi dai problemi in un programma in esecuzione.

Gli esempi di codice presentati successivamente in questo libro saltano volutamente il codice
di controllo degli errori e di recupero poich� ci� potrebbe oscurare il funzionamento di base che
viene presentato. Comunque, l'esempio finale nel \numnameref{cap:11}\marginpar{capitolo 11, ``Un'applicazione di esempio
GNU/Linux''}, ritorna a dimostrare come usare queste tecniche per scrivere programmi robusti.

\subsection{Usare assert}\label{subsec:2.2.1} % 2.2.1

Una buona cosa da tenere in mente quando si codificano programmi applicativi � che i bug o
gli errori inaspettati potrebbero causare il fallimento dei programmi drammaticamente, il pi�
presto possibile. Ci� ti aiuter� a trovare bug al pi� presto durante i cicli di sviluppo e testing. I
fallimenti che non vengono fuori da soli sono spesso dimenticati e non si mostrano fino a che
l'applicazione non � nelle mani dell'utente.

Uno dei metodi pi� semplici per scovare condizioni inaspettate � la macro \texttt{assert} dello
standard C. L'argomento da passare a questa macro � un'espressione Booleana. Il programma
viene terminato se ci� che viene valutato dall'espressione � falso, dopo aver stampato un
messaggio d'errore contenente il file sorgente, il numero di riga ed il testo dell'espressione. La
macro \texttt{assert} � molto utile per una grande variet� di controlli di consistenza interni al
programma. Per esempio, usa \texttt{assert} per testare la validit� degli argomenti di una funzione,
per testare le precondizioni e postcondizioni delle chiamate di funzioni (e chiamate di metodi in
C++), e per testare dei valori di ritorno inaspettati.

Ogni utilizzo di \texttt{assert} lavora non solo come un controllo di una condizione a runtime, ma
anche come documentazione sul funzionamento del programma all'interno del codice sorgente.
Se il tuo programma contiene un \texttt{assert (condizione)} ci� dice a qualcuno che sta
leggendo il tuo codice sorgente che quella \texttt{condizione} dovrebbe essere sempre vera in quel
punto del programma, e se la \texttt{condizione} non � vera, probabilmente c'� un bug nel
programma.

Per il codice che deve avere delle prestazioni ottimali, le verifiche a runtime come l'uso di
\texttt{assert} pu� imporre significanti penalit� in termini di prestazioni. In questo caso, puoi
compilare il codice definendo la macro \texttt{NDEBUG}, tramite l'utilizzo del flag \texttt{-NDEBUG} da riga di
comando del compilatore. Con \texttt{NDEBUG} settato, le occorrenze della macro \texttt{assert} verranno
preprocessate via. E' una buona idea farlo solo quando necessario per ragioni di prestazioni e
solo in casi critici di file sorgenti che devono avere certe prestazioni.

Poich� � possibilie preprocessare via le macro assert, stai molto attento al fatto che ogni
espressione utilizzata con asert non abbia effetti collaterali. Specialmente, non andrebbero
chiamate delle funzioni all'interno dell'espressione assert, assegnare variabili o usare operatori
di modifica come ++. Supponi, per esempio, di chiamare una funzione \texttt{do\_something}
ripetutamente in un loop. La funzione \texttt{do\_something} restituisce zero in caso di successo e
un valore diverso da zero in caso di fallimento, ma ti aspetti che la funzione non fallisca mai nel
tuo programma. Potresti quindi essere tentato di scrivere:

\begin{listcodeC}
for (i = 0; i < 100; ++i)
    assert (do_something () == 0);
\end{listcodeC}

Comunque, potresti scoprire che questo controllo a runtime impone al programma troppe
penalit� di prestazioni e decidere successivamente di compilare con \texttt{NDEBUG} definito. Ci�
rimuover� interamente la chiamata \texttt{assert}, cos� l'espressione non sar� mai valutata e
\texttt{do\_something} non verr� mai chiamata. Piuttosto potresti scrivere questo:
\begin{listcodeC}
for (i = 0; i < 100; ++i) {
int status = do_something ();
assert (status == 0);
}
\end{listcodeC}

Un'altra cosa da tenere in mente � che non si dovrebbe usare assert per testare l'input non
valido da parte dell'utente. Agli utenti non piace quando l'applicazione va in crash
semplicemente con un messaggio di errore critico in riposta all'input non corretto. Dovresti
sempre controllare l'input non corretto e produrre dei messaggi d'errore adeguati in risposta
all'input. Utilizza \texttt{assert} solo per controlli interni a runtime.

Alcuni buoni punti per l'utilizzo di \texttt{assert} sono questi:

\begin{itemize}
\item{controlla che non ci siano puntatori nulli, per esempio, un argomento di funzione non
valido. Il messaggio di errore generato da \texttt{\{assert (pointer != NULL) \}},
\begin{verbatim}
Assertion 'pointer != ((void *)0)' failed.
\end{verbatim}
Fornisce molte pi� informazioni rispetto al messaggio di errore che ne risulterebbe se il tuo
programma facesse riferimento ad un puntatore nullo:
\begin{verbatim}
Segmentation fault (core dumped)
\end{verbatim}
}
\item{Controlla le condizioni nei valori dei parametri delle funzioni. Per esempio, se una
funzione dovrebbe essere chiamata solo con un valore positivo per il parametro foo, usa
questo all'inizio del corpo della funzione:
\begin{verbatim}
assert (foo > 0);
\end{verbatim}
Ci� ti aiuter� a trovare utilizzi scorretti della funzione, e rende anche molto chiaro a chi
sta leggendo il codice sorgente della funzione che c'� una restrizione sul valore del
parametro.
}
\end{itemize}

Non fermari solo a questo; usa \texttt{assert} liberamente nei tuoi programmi.

\subsection{Fallimenti delle chiamate di sistema}\label{subsec:2.2.2} % 2.2.2

Molti di noi hanno imparato originariamente come scrivere programmi che vanno
dall'esecuzione fino al completamento lungo un percorso ben definito. Dividiamo il programma
in compiti (task) e sotto-compiti (subtask), ed ogni funzione finisce un compito invocando altre
funzioni per compiere i corrispondenti sotto-compiti. Dando gli input appropriati, ci aspettiamo
che una funzione produca il corretto output e relativi effetti collaterali.

La realt� dell'hardware e del software dei computer si intromette in questo sogno idealizzato.
I computer hanno risorse limitate; l'hardware sbaglia; molti programmi vengono eseguiti allo stesso
tempo; sia utenti che programmatori commettono degli errori. � spesso al confine tra
applicazione e sistema operativo che si manifestano queste realt�. Quindi, quando si utilizzano
le chiamate di sistema per accedere alle risorse di sistema, per eseguire operazioni di I/O, o per
altri scopi, � importante capire non solo cosa accade quando le chiamate di sistema hanno
successo, ma anche come e perch� le chiamate possono fallire.

Le chiamate di sistema possono fallire in molti modi. Per esempio:

\begin{itemize}
\item{Il sistema pu� girare al di fuori delle risorse (o il programma pu� eccedere i limiti
delle risorse assegnate dal sistema per un singolo programma). Per esempio, il
programma potrebbe tentare di allocare troppa memoria, per scrivere troppo sull'hard
disk, o aprire troppi file allo stesso tempo.}
\item{Linux pu� bloccare certe chiamate di sistema quando un programma tenta di eseguire
un'operazione per la quale non ha i permessi. Per esempio, un programma pu� tentare di
scrivere su un file marcato per la sola lettura, accedere alla memoria di un altro
processo, o uccidere il programma di un altro utente.}
\item{Gli argomenti per una chiamata di sistema potrebbero non essere validi, sia perch�
l'utente potrebbe aver fornito un input non valido o per un bug del programma. Per
esempio, il programma potrebbe passare un indirizzo di memoria non valido o un
descrittore di file non valido ad una chiamata di sistema. O un programma potrebbe
tentare di aprire una direcotory come fosse un file ordinario, o potrebbe passare il nome
di un file ordinario ad una chiamata di sistema che si aspetta una directory.}
\item{Una chiamata di sistema pu� fallire per motivi esterni al programma. Ci� accade
molto spesso quando una chiamata di sistema accede ad una periferica hardware. La
periferica potrebbe essere difettosa o potrebbe non supportare una particolare
operazione, o forse un disco non � inserito nel drive.}
\item{Una chiamata di sistema pu� qualche volta essere interrotta da un evento esterno,
come l'invio di un segnale. Questo pu� non indicare un totale fallimento, ma � la causa
del fatto che il programma chiamante riavvii la chiamata di sistema, se lo si desidera.}
\end{itemize}

In un programma ben scritto, che fa ampio uso delle chiamate di sistema, si verifica spesso il
caso in cui la maggior parte del codice ha il compito di scovare e gestire gli errori ed altre
circostanze critiche piuttosto che occuparsi del lavoro principale del programma.

\subsection{Codici di errore dalle chiamate di sistema}\label{subsec:2.2.3} % 2.2.3

La maggior parte delle chiamate di sistema restituiscono zero se l'operazione ha avuto
successo o un valore diverso da zero se l'operazione � fallita. (Molti, comunque, usano diverse
convenzioni per i valori di ritorno; per esempio, \texttt{malloc} restituisce un puntatore nullo per
indicare un fallimento. Leggi sempre le pagine di manuale con attenzione quando usi una
chiamata di sistema). Sebbene quest'informazione possa essere sufficiente per determinare
quando il programma dovrebbe continuare l'esecuzione come di solito, probabilmente non
fornisce abbastanza informazioni per un ripristino sensibile dagli errori.

Molte chiamate di sistema usano una variabile speciale chiamata \texttt{errno} per memorizzare
ulteriori informazioni in caso di fallimento.\footnote{Attualmente, per ragioni di sicurezza dei
thread, \texttt{errno} � implementata come macro, ma usata come una variabile globale.} Quando
una chiamta di sistema fallisce, il sistema setta la variabile \texttt{errno} ad un valore che indica cosa
� andato storto. Poich� tutte le chiamate di sistema usano la stessa variabile \texttt{errno} per
memorizzare le informazioni, dovresti copiare immediatamente il valore in un'altra variabile
dopo il fallimento della chiamata di sistema. Il valore di \texttt{errno} verr� sovrascritto la prossima
volta che farai una chiamata di sistema.

I valori di errore sono degli interi; valori possibili sono assegnati dalle macro di
preprocessore, per convenzione con tutte lettere maiuscole ed inizianti con ``E'' \--- per esempio,
\texttt{EACCES} e \texttt{EINVAL}. Usa sempre queste macro per fare riferimento ai valori di \texttt{errno}
piuttosto che valori interi. Se usi i valori di \texttt{errno} includi l'header \texttt{<errno.h>}.

GNU/Linux fornisce una conveniente funzione, \texttt{strerror}, che restituisce una descrizione
in stringa di caratteri di un codice di errore \texttt{errno}, adatto per l'utilizzo nei messaggi di errore.
Se usi \texttt{strerror} includi \texttt{<string.h>}.

GNU/Linux fornisce anche \texttt{perror}, che stampa la descrizione dell'errore direttamente sullo
stream \texttt{stderr}. Passa a perror un prefisso come stringa di caratteri da stampare prima della
descrizione dell'errore, che solitamente dovrebbe includere il nome della funzione che � fallita.
Se usi \texttt{perror} includi \texttt{<stdio.h>}.

Questo frammento di codice cerca di aprire un file; se l'apertura fallisce, esso stampa un
messaggio di errore ed esce dal programma. Nota che la chiamata \texttt{open} restituisce un
descrittore di file aperto se l'operazione di apertura ha avuto successo, o \texttt{-1} se l'operazione
fallisce.

\begin{listcodeC}
fd = open ("inputfile.txt", O_RDONLY);
if (fd == -1) {
/* The open failed. Print an error message and exit. */
fprintf (stderr, ?error opening file: %s\n?, strerror (errno));
exit (1);
}
\end{listcodeC}

Dipendentemente dal tuo programma e dalla natura della chiamata di sistema, l'azione
appropriata in caso di fallimento pu� essere quella di stampare un messaggio di errore, di cancellare
l'operazione, di far chiudere il programma, di ritentare oppure di ignorare l'errore. � importante
comunque inserire in un modo o nell'altro una logica che gestisca tutte le possibili modalit� di
fallimento.

Un possibile codice di errore al quale dovresti prestare attenzione, specialmente con le
funzioni di I/O, � \texttt{EINTR}. Alcune funzioni, come \texttt{read}, \texttt{select} e \texttt{sleep}, possono
impiegare molto tempo per essere eseguite. Queste sono considerate funzioni \textit{bloccanti} perch�
l'esecuzione del programma � bloccata fino a che la chiamata non � completata. Comunque, se
il programma riceve un segnale mentre � bloccato in una di queste chiamate, la chiamata finir�
senza aver completato l'operazione. In questo caso, \texttt{errno} � settato a \texttt{EINTR}. Di solito, in
questi casi, vorrai ritentare la chiamata di sistema.

Ecco un frammento di codice che usa la chiamata \texttt{chown} per cambiare il proprietario di un
file dato da \texttt{path} all'utente da \texttt{user\_id}. Se la chiamata fillisce, il programma esegue l'azione
in base al valore di \texttt{errno}. Nota che quando troviamo un possibile bug nel programma,
usciamo usando \texttt{abort} o \texttt{assert}, che causa la generazione di un file \textit{core}. Ci� pu� esser
utile per un debug post-mortem. Per altri errori non recuperabili,
come la condizione di out-of-memory,
usciamo usando \texttt{exit} ed un valore diverso da zero perch� un file core non sarebbe
molto utile.

\begin{listcodeC}
rval = chown (path, user_id, -1);
if (rval != 0) {
  /* Save errno because it's clobbered
     by the next system call. */
  int error_code = errno;
  /* The operation didn?t succeed; chown
     should return -1 on error. */
  assert (rval == -1);
  /* Check the value of errno, and take appropriate action. */
  switch (error_code) {
  case EPERM: /* Permission denied. */
  case EROFS: /* PATH is on a read-only file system. */
  case ENAMETOOLONG: /* PATH is too long. */
  case ENOENT: /* PATH does not exit. */
  case ENOTDIR: /* A component of PATH is not a directory. */
  case EACCES: /* A component of PATH is not accessible. */
      /* Something?s wrong with the file.
       Print an error message. */
    fprintf (stderr, "error changing ownership of %s: %s\n",
        path, strerror (error_code));
    /* Don?t end the program; perhaps give the user a chance to
      choose another file... */
    break;
  case EFAULT:
    /* PATH contains an invalid memory address.
       This is probably a bug. */
  abort ();
  case ENOMEM:
    /* Ran out of kernel memory. */
       fprintf (stderr, ?%s\n?, strerror (error_code));
    exit (1);
  default:
    /* Some other, unexpected, error code.
       We?ve tried to handle all possible error codes;
       if we've missed one, that?s a bug! */
    abort ();
  };
}
\end{listcodeC}

Potresti semplicemente aver usato questo codice, che si comporta allo stesso modo se la
chiamata ha successo:

\begin{listcodeC}
rval = chown (path, user_id, -1);
assert (rval == 0);
\end{listcodeC}

Ma se la chiamata fallisce, quest'alternativa non fornisce nessun aiuto per il riportare,
gestire, o riprendersi dagli errori.

Se usi la prima forma, la seconda forma o qualche altra cosa intermedia dipende dai requisiti
di trovare gli errori e gestirli del tuo programma.

\subsection{Errori ed allocazione delle risorse}\label{subsec:2.2.4} % 2.2.4

Spesso, quando una chiamata di sistema fallisce, sarebbe opportuno annullare l'operazione
corrente ma non terminare il programma poich� potrebbe essere possibile un ripristino
dall'errore. Un modo per farlo � quello di ritornare dalla funzione corrente, restituendo un
codice alla chiamante che indichi l'errore.

Se decidi di ritornare mentre l'esecuzione � nel mezzo della funzione, � importante assicurarsi che ogni risorsa
nella funzione precedentemente allocata con successo sia prima deallocata. Le risorse possono
includere memoria, descrittori di file, puntatori a file, file temporanei, sincronizzazione di
oggetti, e cos� via. Altrimenti, se il tuo programma continua a girare, le risorse allocate prima
dell'errore verrebbero perse.

Considera, per esempio, una funzione che legge da un file in un buffer. La funzione potrebbe
seguire questi passi:

\begin{enumerate}
\item Allocare il buffer.
\item Aprire il file.
\item Leggere dal file nel buffer.
\item Chiudere il file.
\item Restituire il buffer.
\end{enumerate}

Se il file non esiste, il passo 2 fallir�. Un'azione appropriata potrebbe essere quella di
restituire NULL dalla funzione. Comunque, se il buffer � stato gi� allocato al passo 1, c'� il
rischio di perdere quella memoria. Devi ricordarti di deallocare il buffer da qualche parte lungo
ogni flusso di controllo dal quale non ritorni. Se il passo 3 fallise, non devi solo deallocare il
buffer prima di ritornare, ma devi anche chiudere il file.

Il listato 2.6 mostra un esempio di come dovresti scrivere questa funzione.\\

\listfromfile{readfile.c}{Liberare le risorse in condizioni non normali}{list:2.6}
	{ALP-listings/chapter-2/readfile.c}

Linux libera la memoria allocata, file aperti, e molte altre risorse quando un programma
termina, cos� non � necessario dellocare buffer e chiudere file prima di chiamare \texttt{exit}. Potresti
aver bisogno di liberare manualmente altre risorse condivise, comunque, come file temporanei e
memoria condivisa, che potrebbero potenzialmente persistere nel programma.

\section{Scrivere ed usare librerie}\label{sec:2.3} % 2.3

Virtualmente tutti i programmi sono linkati con una o pi� librerie. Ogni programma che usa
una funzione C (come \texttt{printf} o \texttt{malloc}) sar� linkato con la libreria di runtime C. Se il tuo
programma ha un'interfaccia utente grafica (GUI), esso sar� linkato con le librerie delle finestre.
Se il tuo programma usa un database, il fornitore del database ti dar� le librerire che potrai usare
per accedere al database.

In ognuno di questi casi, devi decidere come linkare la libreria, statica o dinamica. Se scegli
di fare il link statico, i tuoi programmi saranno grandi e difficili da aggiornare, ma
probabilmente facili da sviluppare. Se fai il link dinamico, i tuoi programmi saranno piccoli,
facili da aggiornare, ma difficili da sviluppare. Questa sezione spiega come fare i link, sia statici
che dinamici, esamina le differenze pi� dettagliatamente, e fornisce alcune ``regole a naso'' per
decidere che tipo di link � meglio per te.

\subsection{Archivi}\label{subsec:2.3.1} % 2.3.1

Un \textit{archivio} (o libreria statica) � semplicemente una collezione di file oggetto memorizzati in
un singolo file. (Un archivio approssimativamente � l'equivalente di un file \texttt{.LIB} di Windows).
Quando fornisci un archivio al linker, il linker cerca nell'archivio i file oggetto che gli servono,
li estrae, e quindi ne fa il link nel tuo programma come se tu gli avessi fornito quei file
direttamente.

Puoi creare un archivio usando il comando \texttt{ar}. I file di archivio tradizionalmente usano
l'estensione .a piuttosto che l'estensione .o usata comunemente dai file oggetto. Ecco come puoi
combinare \texttt{test1.o} e \texttt{test2.o} in un singolo archivio \texttt{libtest.a}:
\begin{listcodeBash}
% ar cr libtest.a test1.o test2.o
\end{listcodeBash}

Il flag \texttt{cr} dice ad \texttt{ar} di creare un archivio.\footnote{� possibile usare altri flag per rimuovere file da
una archivio o per eseguire altre operazione nell'archivio. Queste operazioni sono usate
raramente ma sono documentate nella pagina di manuale di \texttt{ar}.} Adesso puoi fare il link con
questo archivio usando l'opzione \texttt{-ltest} con gcc o g++, come descritto nella \numnameref{subsec:1.2.2}, nel \numnameref{cap:1}.

Quando il linker incontra un archivio sulla riga di comando, cerca nell'archivio tutte le
definizioni di simboli (funzioni o variabili) che sono referenziate dai file oggetto, che sono gi�
stati processati ma non ancora definiti. I file oggetto che definiscono questi simboli sono estratti
dall'archivio ed inclusi nell'eseguibile finale. Poich� il linker cerca l'archivio solo quando lo
incontra sulla riga di comando, di solito ha senso mettere l'archivio alla fine della riga di
comando. Per esempio, supponiamo che \texttt{test.c} contenta il codice nel listato 2.7 e \texttt{app.c}
contenga il codice nel listato 2.8.\\

\listfromfile{test.c}{Contenuto della libreria}{list:2.7}
	{ALP-listings/chapter-2/test.c}

\listfromfile{app.c}{Un programma che usa le funzioni della libreria}{list:2.8}
	{ALP-listings/chapter-2/app.c}

Adesso supponi che \texttt{test.o} sia combinato con alcuni altri file oggetto per creare l'archivio
libtest.a. Il seguente comando non funzioner�:
\begin{listcodeBash}
% gcc -o app -L. -ltest app.o
app.o: In function 'main':
app.o(.text+0x4): undefined reference to 'f'
collect2: ld returned 1 exit status
\end{listcodeBash}

Il messaggio di errore indica che anche se \texttt{libtest.a} contiene una definizione di \texttt{f}, il
linker non la trova. Ci� accade perch� \texttt{libtest.a} � stato cercato quando � stato incontrato
all'inizio, ed in quel punto il linker non ha visto alcun riferimento a \texttt{f}.

D'altro canto, se usiamo questa linea, non avremo nessun errore:
\begin{listcodeBash}
%gcc -o app app.o -L. -ltest
\end{listcodeBash}

Il motivo � che il riferimento ad \texttt{f} in app.o causa che il linker includa il file oggetto \texttt{test.o}
dall'archivio \texttt{libtest.a}.

\subsection{Librerie condivise}\label{subsec:2.3.2} % 2.3.2

Una libreria condivisa (conosciuta anche come oggetto condiviso o libreria linkata
dinamicamente) � simile a un archivio nel quale c'� un gruppo di file oggetto. Comunque, ci
sono molte differenze importanti. La differenza principale � che quando una libreria condivisa �
linkata in un programma, l'eseguibile finale non contiene il codice che � presente nella libreria
condivisa. L'eseguibile invece contiene un riferimento alla libreria condivisa. Se nel sistema
molti programmi sono linkati alla stessa libreria condivisa, essi faranno tutti riferimento alla
stessa libreria, ma nessuno sar� incluso. Cos�, la libreria � ``condivisa'' da tutti i programmi che
sono linkati ad essa.

Una seconda importante differenza � che la libreria condivisa non � semplicemente una
collezione di file oggetto nel quale il linker sceglie quelli che servono a soddisfare i riferimenti
indefiniti. Piuttosto, i file oggetto che compongono la libreria condivisa sono ombinati in un
singolo file oggetto cosicch� un programma che linka assieme una libreria condivisa include
sempre tutto il codice della libreria piuttosto che solo quelle porzioni di cui ha bisogno.

Per creare una libreria condivisa, devi compilare il file oggetto che former� la libreria usando
l'opzione \texttt{-fPIC} per il compilatore, come questa:
\begin{listcodeBash}
% gcc -c -fPIC test1.c
\end{listcodeBash}
L'opzione \texttt{-fPIC} dice al compilatore che stai per usare \texttt{test.o} come parte di un oggetto
condiviso.

\begin{quote}
{\large\textbf{Position-Independent Code (PIC)}}\\
PIC sta per codice a posizione indipendente (position-independen code). Le
funzioni in una libreria condivisa possono essere caricate ad indirizzi diversi
in diversi programmi, cos� il codice nell'oggetto condiviso non deve
dipendere dall'indirizzo (o posizione) al quale � stato caricato. Questa
considerazione non ha nessun impatto su di te, come programmatore, eccetto che
devi ricordare di usare il flag -fPIC quando compili del codice che userai in una
libreria condivisa.
\end{quote}

Quindi combini il file oggetto in una libreria condivisa, come questa:
\begin{listcodeBash}
% gcc -shared -fPIC -o libtest.so test1.o test2.o
\end{listcodeBash}

L'opzione \texttt{-shared} dice al linker di prourre una libreria piuttosto che un eseguibile come
di solito. Le librerie condivise usano l'estensione \texttt{.so}, che sta per shared object (oggetto
condiviso). Come gli archivi statici, il nome inizia sempre con \texttt{lib} per indicare che il file � una
libreria.

Fare il link con una libreria condivisa � come fare il link con un archivio statico. Per
esempio, la seguente riga far� il link con libtest.so se esso � nella directory corrente
o in una delle directory nei percorsi di ricerca standard del sistema:
\begin{listcodeBash}
% gcc -o app app.o -L. -ltest
\end{listcodeBash}
Supponiamo che siano disponibili entrambi \texttt{libtest.a} e \texttt{libtest.so}. A questo punto
il linker dovr� scegliere una delle libreria e scartare l'altra. Il linker cerca in ogni directory
(prima quelle specificate dall'opzione \texttt{-L}, e poi quelle nelle directory standard). Quando il linker
trova una directory che contiene o \texttt{libtest.a} o \texttt{libtest.so}, smette la ricerca nelle
directory. Se solo una delle due forme � presente nella directory, il linker sceglier� quella
forma. Altrimenti, il linker sceglier� la versione in libreria condivisa, a meno che tu non gli dica
di fare altrimenti. Puoi usare l'opzione \texttt{-static} per far utilizzare gli archivi statici. Per
esempio, la seguente linea user� l'archivio \texttt{libtest.a}, anche se la libreria \texttt{libtest.so} �
pure disponibile.
\begin{listcodeBash}
% gcc -static -o app app.o -L. -ltest
\end{listcodeBash}

Il comando \texttt{ldd} mostra le librerie condivise che sono linkate nell'eseguibile. � necessario
che queste librerie siano disponibili quando l'eseguibile � in esecuzione. Nota che \texttt{ldd} elencher�
una libreria aggiuntiva chiamata \texttt{ld-linux.so}, che � parte del meccanismo di linking
dinamico di GNU/Linux.

\paragraph{Usare \textit{LD\_LIBRARY\_PATH}}

Quando fai il link di un programma con una libreria condivisa, il linker non mette il percorso
completo per la libreria condivisa nell'eseguibile finale. Piuttosto, mette solo il nome della
libreria condivisa. Quando il programma � in esecuzione, il sistema cerca la libreria condivisa e
la carica. Il sistema, di default, cerca solo in \texttt{/lib} e \texttt{/usr/lib}. Se la libreria condivisa
linkata al tuo programma � installata fuori da queste directory, non verr� trovata, ed il sistema si
rifiuter� di eseguire il programma.

\begin{sloppypar}
Una soluzione a questo problema sta nell'utilizzo dell'opzione \texttt{-Wl,} \texttt{-rpath} quando si fa il link
del programa. Supponi di usare la seguente:
\end{sloppypar}
\begin{listcodeBash}
% gcc -o app app.o -L. -ltest -Wl,-rpath,/usr/local/lib
\end{listcodeBash}

In questo modo, quando il programma � in esecuzione, il sistema cercher� in
\texttt{/usr/local/lib} per ogni libreria condivisa richiesta.

\begin{sloppypar}
Un'altra soluzione al problema � impostare la variabile d'ambiente \texttt{LD\_LIBRARY\_PATH}
quando si esegue il programma. Come la variabile d'ambiente \texttt{PATH}, \texttt{LD\_LIBRARY\_PATH} �
una lista di directory separate dai due punti. Per esempio, se \texttt{LD\_LIBRARY\_PATH} �
\texttt{/usr/local/lib:/opt/lib}, allora la ricerca verr� fatta in \texttt{/usr/local/lib} e in
\texttt{/opt/lib} prima delle directory standard \texttt{/lib} e \texttt{/usr/lib}. Dovresti notare anche che se
hai \texttt{LD\_LIBRARY\_PATH}, il linker nel creare l'eseguibile cercher� nelle directory fornite qui
oltre che nelle directory fornite dall'opzione \texttt{-L}.\footnote{Potresti trovare dei riferimenti a
\texttt{LD\_RUN\_PATH} in alcune documentazioni online. Non credere a quello che leggi; questa
variabile non fa nulla sotto GNU/Linux.}
\end{sloppypar}

\subsection{Librerie standard}\label{subsec:2.3.3} % 2.3.3

Anche se non non hai specificato nessuna libreria quando hai fatto il link del tuo programma,
quasi certamente questo user� una libreria condivisa. Ci� accade perch� GCC fa il link
automaticamente nelle librerie standard C, \texttt{libc}, per te. Le funzioni matematiche delle librerie
standard C non sono incluse in \texttt{libc}; invece, stanno in una libreria separata, \texttt{libm}, che hai
bisogno di specificare esplicitamente. Per esempio, per compilare e fare il link di un
programma \texttt{compute.c} che usa funzioni trigonometriche come \texttt{sin} e \texttt{cos}, devi richiamare
questo codice:
\begin{listcodeBash}
% gcc -o compute compute.c -lm
\end{listcodeBash}

Se scrivi un programma C++ e fai il link usando il comando \texttt{c++} o \texttt{g++}, prenderai
automaticamente anche la libreria standard C++, \texttt{libstdc++}.

\subsection{Dipendenze delle librerie}\label{subsec:2.3.4} % 2.3.4

Una libreria spesso dipender� da un'altra libreria. Per esempio, molti sistemi GNU/Linux
includono \texttt{libtiff}, una libreria contenente funzioni per leggere e scrivere file di immagini in
formato TIFF. Questa libreria, a sua volta, usa le librerie \texttt{libjpeg} (procedure per immagini
JPEG) e \texttt{libz} (procedure di compressione)

Il listato 2.9 mostra un programma molto piccolo che usa libtiff per aprire un file di
immagine TIFF.\\

\listfromfile{tifftest.c}{Utilizzo di \textit{libtiff}}{list:2.9}
	{ALP-listings/chapter-2/tifftest.c}

Salva questo file sorgente come \texttt{tifftest.c} Per compilare questo programma e fare il
link con \texttt{libtiff}, specifica \texttt{-ltiff} sulla tua riga di comando del linker:
\begin{listcodeBash}
% gcc -o tifftest tifftest.c -ltiff
\end{listcodeBash}

Di default, questo si collegher� alla versione della libreria condivisa di \texttt{libtiff}, trovata in
\texttt{/usr/lib/libtiff.so}. Poich� \texttt{libtiff} usa \texttt{libjpeg} e \texttt{libz}, le versioni delle librerie condivise
di queste due sono anche qui (una libreria condivisa pu� puntare ad altre librerie condivise dalle
quali dipende). Per verificare ci�, usa il comando \texttt{ldd}:
\begin{listcodeBash}
% ldd tifftest
    libtiff.so.3 => /usr/lib/libtiff.so.3 (0x4001d000)
    libc.so.6 => /lib/libc.so.6 (0x40060000)
    libjpeg.so.62 => /usr/lib/libjpeg.so.62 (0x40155000)
    libz.so.1 => /usr/lib/libz.so.1 (0x40174000)
    /lib/ld-linux.so.2 => /lib/ld-linux.so.2 (0x40000000)
\end{listcodeBash}

Le librerie statiche, d'altro canto, non possono puntare ad altre librerie. Se decidi di fare il
link con la versione statica di \texttt{libtiff} specificando \texttt{-static} sulla riga di comando, otterrai
dei simboli non risolti:
\begin{listcodeBash}
% gcc -static -o tifftest tifftest.c -ltiff
/usr/bin/../lib/libtiff.a(tif_jpeg.o): In function
	'TIFFjpeg_error_exit':
tif_jpeg.o(.text+0x2a): undefined reference to 'jpeg_abort'
/usr/bin/../lib/libtiff.a(tif_jpeg.o): In function
	'TIFFjpeg_create_compress':
tif_jpeg.o(.text+0x8d): undefined reference to 'jpeg_std_error'
tif_jpeg.o(.text+0xcf): undefined reference to 'jpeg_CreateCompress'
...
\end{listcodeBash}
Per fare il link di questo programma devi specificare le altre due librerie tu stesso:
\begin{listcodeBash}
% gcc -static -o tifftest tifftest.c -ltiff -ljpeg -lz
\end{listcodeBash}
Occasionalmente, due librerie saranno mutuamente dipendenti. In altre parole, il primo
archivio far� riferimento a simboli definiti nel secondo archivio e vice versa. La situazione
generalmente mostra la scarsit� di design, ma lo fa occasionalmente.
In questo caso, nella riga di comando puoi indicare pi� volte una singola libreria.
Il linker cercher�
la libreria ogni volta che ne ha bisogno. Per esempio, questa linea causa la ricerca in
\texttt{libfoo.a} pi� volte:
\begin{listcodeBash}
% gcc -o app app.o -lfoo -lbar -lfoo
\end{listcodeBash}

Cos�, anche se libfoo.a fa riferimento a simboli in \texttt{libbar.a} e vice versa, il link del
programma verr� fatto con successo.

\subsection{Pro e contro}\label{subsec:2.3.5} % 2.3.5

Adesso che sai tutto riguardo gli archivi statici e le librerie condivise, probabilmente ti
chiederai come usarle. Ci sono poche altre importanti considerazioni da tenere a mente.

Il principale vantaggio di una libreria condivisa � quello di risparmiare spazio sul sistema nel
quale il programma � installato. Se stai installando 10 programmi, ed essi fanno tutti uso della
stessa libreria condivisa, risparmi molto spazio usando una libreria condivisa. Se invece hai
utilizzato un archivio statico, l'archivio � incluso in tutti e 10 i programmi. Cos�, usando una
libreria condivisa risparmi spazio su disco. Ci� riduce anche il tempo di download se il tuo
programma viene scaricato dal web.

Un vantaggio correlato alle librerie condivise � che gli utenti possono aggiornare le librerie
senza aggiornare tutti i programmi che dipendono da esse. Per esempio, supponi che hai creato
una libreria condivisa che gestisce le connessioni HTTP. Molti programmi potrebbero
dipendere da questa libreria. Se trovi un bug in questa libreria puoi aggiornare la libreria.
Istantaneamente tutti i programmi che dipendono da questa libreria saranno riparati; non dovrai
rifare il link di tutti i programmi come faresti per un archivio statico.

Questi vantaggi potrebbero farti pensare che dovresti sempre usare le librerie condivise.
Comunque, sostanzialmente esistono delle ragioni per le quali usare invece gli archivi statici. Il
fatto che un aggiornamento di una libreria condivisa ha effetto su tutti i programmi che
dipendono da essa pu� essere uno svantaggio. Per esempio, se stai sviluppando un software
mission-critical, farai piuttosto il link con un archivio statico in modo che un aggiornamento
delle librerie condivise nel sistema non produca effetti sul tuo programma. (Altrimenti, gli
utenti potrebbero aggiornare la libreria condivisa, danneggiando con ci� il tuo programma e
quindi chiamare l'assistenza clienti, attribuendoti le colpe!).

Se non riesci ad installare le tue librerie in \texttt{/lib} o \texttt{/usr/lib}, dovresti pensarci due volte
prima di usare una libreria condivisa. (Non sarai in grado di installare le librerie in quelle
directory se ti aspetti che l'utente installi il tuo software senza permessi di amministrazione). In
particolare, il trucco \texttt{-Wl, -rpath} non funzioner� se non sai dove andranno a finire le librerie.
E chiedere al tuo utente di settare \texttt{LD\_LIBRARY\_PATH} significa un passo in pi� per loro.
Poich� ogni utente deve farlo individualmente, ci� � sostanzialmente un peso.

Devi pesare questi vantaggi e svantaggi per ogni programma che distribuisci.

\subsection{Loading e Unloading dinamico}\label{subsec:2.3.6} % 2.3.6

A volte vuoi caricare del codice a tempo di esecuzione senza fare il link esplicito in quel
codice. Per esempio, considera un'applicazione che supporti moduli ``plug-in'' come un browser
web. Il browser permette a terze parti di creare plug-in per fornire funzionalit� aggiuntive. Gli
sviluppatori di terze parti creano librerie condivise e le mettono in posizioni conosciute. Il
browser web quindi carica automaticamente il codice di queste librerie.

Questa funzionalit� � disponibile su Linux usando la funzione \texttt{dlopen}. Puoi aprire una
libreria condivisa chiamata \texttt{libtest.so} chiamando \texttt{dlopen} come qui:
\begin{listcodeC}
dlopen ('libtest.so', RTLD_LAZY)
\end{listcodeC}
(Il secondo parametro � un flag che indica come collegare i simboli nella libreria condivisa. Se
vuoi altre informazioni puoi consultare le pagine di manuale online di \texttt{dlopen}, ma \texttt{RTLD\_LAZY}
� l'impostazione che ti serve di solito). Per usare le funzioni di caricamento dinamico, includi il
file header \texttt{<dlfcn.h>} e fai il link con l'opzione \texttt{-ldl} per avere il collegamento alla libreria
libdl.

Il valore di ritorno di questa funzione � un \texttt{void *} che � usato come aggancio per la
libreria condivisa. Puoi passare questo valore alla funzione \texttt{dlsym} per ottenere l'indirizzo di
una funzione che � stata caricata con la libreria condivisa. Per esempio, se \texttt{libtest.so}
definisce una funzione chiamata \texttt{my\_function}, puoi chiamarla come in questo modo:
\begin{listcodeC}
void* handle = dlopen ('libtest.so', RTLD_LAZY);
void (*test)() = dlsym (handle, 'my_function');
(*test)();
dlclose (handle);
\end{listcodeC}

La chiamata di sistema \texttt{dlsym} pu� anche essere usata per ottenere un puntatore ad una
variabile statica nella libreria condivisa.

Enrambi \texttt{dlopen} e \texttt{dlsym} restituiscono \texttt{NULL} se non hanno successo. In questo caso, puoi
chiamare \texttt{dlerror} (senza parametri) per ottenere un messaggio di errore comprensibile con la
descrizione del problema.

La funzione \texttt{dlcole} chiude la libreria condivisa. Tecnicamente, \texttt{dlopen} attualmente
carica la libreria solo se non � gi� stata caricata. Se la libreria � gi� stata caricata, \texttt{dlopen}
incrementa semplicemente il contatore di riferimento alla libreria. Similmente, \texttt{dlclose}
decrementa il contatore di riferimento e chiude la libreria solo se il contatore di riferimento ha
raggiunto zero.

Se stai scrivendo il codice nella tua libreria condivisa in C++, probabilmente vorrai
dichiarare quelle funzioni e variabili alle quali vorrai che si acceda altrove con lo specificatore
di link \texttt{extern "C"}. Per esempio, se la funzione C++ \texttt{my\_function} � in una libreria
condivisa e vuoi accedere ad essa con \texttt{dlsym}, la dovresti dichiarare come questa:
\begin{listcodeC}
extern "C" void foo ();
\end{listcodeC}

Ci� evita che il compilatore C++ ``distrugga'' il nome della funzione,
cambiando il nome da \texttt{foo} ad uno differente, per trasformarlo in uno
diverso che rappresenti codificate informazioni extra riguardo la funzione. Un compilatore C
non distrugger� i nomi; user� qualunque nome che darai alla tua funzione o variabile.

% Fine capitolo 2
% !TEX encoding = ISO-8859-1
% !TEX root = alp.tex
% !TEX program = pdflatex
% !TEX spellcheck = it-IT

\chapter{Processi}\label{cap:3}

UN'ISTANZA DI UN PROGRAMMA IN ESECUZIONE � CHIAMATA PROCESSO.
Se hai due finestre di
terminale visualizzate a schermo, con buone probabilit� stai eseguendo due volte lo stesso programma terminale
\--- hai due processi di terminale. Ogni finestra di terminale � probabilmente una shell
in esecuzione; ogni shell in esecuzione � un altro processo. Quando invochi un comando da una shell, il
programma corrispondente � eseguito in un nuovo processo; il processo schell si riprende
quando i processi sono finiti.

I programmatori avanzati spesso usano pi� processi cooperanti in una singola applicazione
per abilitare l'applicazione a fare pi� di una cosa per volta, per aumentare la robustezza
dell'applicazione e per fare uso di programmi gi� esistenti.

Molte delle funzioni di manipolazione dei processi che sono descritte in questo capitolo sono
simili a quelle in altri sistemi UNIX. Molti sono dichiarati nel file header \texttt{<unistd.h>};
controlla la pagina di manuale per ogni funzione per esserne sicuro.

\section{Uno sguardo ai processi}\label{sec:3.1}

Anche se sei seduto davanti al tuo computer, ci sono processi in esecuzione. Ogni
programma in esecuzione usa uno o pi� processi. Iniziamo dando un'occhiata ai processi gi� in
esecuzione sul tuo computer.

\subsection{ID dei processi}\label{subsec:3.1.1}

Ogni processo in un sistema Linux � identificato dal suo ID di processo univoco, al queale
spesso ci si riferisce come \textit{pid}. Gli ID dei processi sono numeri a 16 bit assegnati
sequenzialmente da Linux quando viene creato un nuovo processo.

Ogni processo inoltre ha un processo padre (ad eccezione del processo speciale \texttt{init},
descritto nella \numnameref{subsec:3.4.3}). Quini, puoi immaginare che i processi in un
sistema Linux siano impostati in un albero, con il processo \texttt{init} come radice. Il \textit{parent
process ID}, o ppid, � semplicemente l'ID del padre del processo.

Quando si fa riferimento agli ID dei processi in un programma C o C++, si usa sempre il tipo
di dato \texttt{pid\_t}, che � definito in \texttt{<sys/types.h>}. Un programma pu� ottenere l'ID di un
processo in esecuzione con la chiamata di sistema \texttt{getpid()} e ottenere l'ID del padre di
questo processo con la chiamata di sistema \texttt{getppid()}. Per esempio, il programma nel listato
3.1 stampa gli ID dei suoi processi e gli ID dei loro genitori.\\

\listfromfile{print-pid.c}{Stampa gli ID dei processi}{list:3.1}
	{ALP-listings/chapter-3/print-pid.c}

Nota che se invochi questo programma molte volte, viene riportato un ID di processo diverso
perch� ogni invocazione crea un nuovo processo. Comunque, se lo invochi ogni volta dalla
stessa shell, l'ID del processo padre (che � l'ID di processo del processo shell) � lo stesso.

\subsection{Vedere i processi attivi}\label{subsec:3.1.2} % 3.1.2

Il comando \texttt{ps} mostra i processi che stanno girando nel sistema. La versione GNU/Linux di
\texttt{ps} ha molte opzioni perch� cerca di essere compatibile con le versioni di \texttt{ps} in molte altre
varianti di UNIX. Queste opzioni controllano quali processi sono listati e le informazioni a
riguardo per ogni visualizzazione.

Per default, l'invocazione di \texttt{ps} mostra i processi mostrati dal terminale o dalla finestra di terminale
nella quale \texttt{ps} � invocato. Per esempio:
\begin{listcodeBash}
% ps
PID     TTY     TIME      CMD
21693   pts/8   00:00:00  bash
21694   pts/8   00:00:00  ps
\end{listcodeBash}

L'invocazione di \texttt{ps} mostra due processi. Il primo, \texttt{bash}, � la shell in esecuzione in
questo terminale. Il secondo � l'istanza in esecuzione del programma \texttt{ps} stesso. La prima
colonna, etichettata PID, mostra l'ID di processo di ognuno.

Per una vista pi� dettagliata e ci� che sta girando sul sistema GNU/Linux, invoca questo:
\begin{listcodeBash}
% ps -e -o pid,ppid,command
\end{listcodeBash}

L'opzione \texttt{-e} dice a \texttt{ps} di mostrare tutti i processi che stanno girando sul sistema. L'opzione
\texttt{-o pid,ppid,command} indica a \texttt{ps} quali informazioni mostrare a riguardo di ogni
processo \--- in questo caso, l'ID di processo, l'ID del processo padre, ed il comando in esecuzione
in questo processo.

\begin{quote}
\large{\textbf{Formati dell'output di ps}}\\
Con l'opzione \texttt{-o} al comando \texttt{ps}, specifichi le informazioni riguardanti i processi
che vuoi nell'output come lista separata da virgole. Per esempio,\\
\texttt{ps -o pid,user,start\_time,command}\\
mostra l'ID di processo, il nome del
proprietario del processo, l'orario al quale il processo � stato avviato ed il comando
eseguito in quel processo. Vedi la pagina di manuale di \texttt{ps} per la lista completa di
codici di campo. Puoi usare invece le opzioni -f (lista completa), -l (lista lunga) o -j
(lista dei lavori) per ottenere tre diversi formati di liste preimpostati.
\end{quote}

Ecco le prime righe ed ultime righe di output da questo comando nel mio sistema. Puoi
notare output differenti, in base a cosa c'� in esecuzione nel tuo sistema.
\begin{listcodeBash}
% ps -e -o pid,ppid,command
PID     PPID    COMMAND
1       0       init [5]
2       1       [kflushd]
3       1       [kupdate]
...
21725   21693   xterm
21727   21725   bash
21728   21727   ps -e -o pid,ppid,command
\end{listcodeBash}

Nota che l'ID del processo padre del comando ps, 21727, � l'ID di processo di bash, la shell
dalla quale ho invocato \texttt{ps}. L'ID del processo padre di bash a sua volta � 21725, l'ID di
processo del programma xterm nel quale la shell sta girando.

\subsection{Uccidere un processo}\label{subsec:3.1.3} % 3.1.3

Puoi uccidere un processo in esecuzione con il comando \texttt{kill}. Specifica semplicemente
sulla linea di comando l'ID di processo del processo da uccidere.

Il comando \texttt{kill} lavora inviando al processo un \texttt{SIGTERM}, o segnale di termine.\footnote{puoi
anche usare il comando \texttt{kill} per inviare altri segnali a un processo. Questo � descritto nella
\numnameref{sec:3.4}.} Ci� fa terminare il processo, a meno che il programma in
esecuzione non prenda esplicitamente o mascheri il senale SIGTERM. I segnali sono descritti
nella \numnameref{sec:3.3}.

\section{Creare processi}\label{sec:3.2} % 3.2

Vengono usate due tecniche comuni per creare un nuovo processo. La prima � relativamente
semplice ma dovrebbe essere usata con moderazione perch� non � efficiente ed ha considerevoli
rischi di sicurezza. La seconda tecnica � pi� complessa ma fornisce maggiore flessibilit�,
velocit� e sicurezza.

\subsection{usando system}\label{subsec:3.2.1} % 3.2.1

La funzione \texttt{system} nella libreria standard C fornisce una via facile per eseguire un
comando dall'interno di un programma, come se il comando sia stato digitato in una shell. Di
fatto, \texttt{system} crea un sottoprocesso eseguendo la Boune shell standard (/bin/sh) e cede il
comando a quella shell per l'esecuzione. Per esempio, questo programma nel listato 3.2 invoca
il comando \texttt{ls} per mostrare il contenuto della directory root come se avessi digitato \texttt{ls -l /} in
una shell.\\

\listfromfile{system.c}{Utilizzo della chiamata \textit{system}}{list:3.2}
	{ALP-listings/chapter-3/system.c}

La funzione \texttt{system} restituisce lo stato di uscita del comando shell. Se la shell stessa non
pu� essere eseguita, \texttt{system} restituisce 127; se si verifica un altro errore, \texttt{system} restituisce
\texttt{-1}.

Poich� la funzione \texttt{system} usa una shell per invocare il tuo comando, esso � soggetto alle
caratteristiche, limitazioni, e problemi di sicurezza della shell di sistema. Non puoi fare
riferimento all'abilit� di ogni particolare versione della shell Bourne. Su molti sistemi UNIX,
\texttt{/bin/sh} � un link simbolico ad un'altra shell. Per esempio, su molti sistemi GNU/Linux,
\texttt{/bin/sh} punta alla \texttt{bash} (Bourne-Again Shell), e diverse distribuzioni GNU/Linux usano
diverse versioni di bash. Invocare un programma con permessi di \texttt{root} con la funzione
\texttt{system}, per esempio, pu� avere diversi risultati in diversi sistemi GNU/Linux. Quindi, �
preferibile usare il metodo \texttt{fork} ed \texttt{exex} per creare processi.

\subsection{Usando \textit{fork} ed \textit{exec}}\label{subsec:3.2.2} % 3.2.2

Le API di DOS e Windows contengono la famiglia di funzioni \texttt{span}. Queste funzioni
prendono come argomento il nome di un programma da eseguire e creano una nuova istanza di
processo di quel programma. Linux non contiene una singola funzione che fa tutto ci� in un
solo passo. Piuttosto, Linux fornisce una funzione, \texttt{fork}, che crea un processo figlio che � una
copia esatta del suo processo padre. Linux fornisce un altro insieme di funzioni, la famiglia
\texttt{exec}, che fanno in modo che un particolare processo smetta di essere un'istanza di un
programma e diventare invece una copia del processo corrente. Quindi puoi usare \texttt{exec} per
trasformare uno di questi processi in un'istanza del programma che vuoi avviare.

\paragraph{Chiamare \textit{fork}}

Quando un programma chiama \texttt{fork}, viene creato un processo duplicato, chiamato processo
figlio. Il processo padre continua l'esecuzione del programma dal punto il cui � stato chiamato il
\texttt{fork}. Il processo figlio, inoltre, esegue lo stesso programma dallo stesso posto.

Cos�, in cosa differiscono i due processi? Primo, il processo figlio � un nuovo processo e
quindi ha un nuovo ID di processo, diverso dall'ID del suo processo padre. Un modo per un
programma per distinguere quando � nel processo padre o nel processo figlio � chiamare
\texttt{getpid}. Comunque, la funzione \texttt{fork} fornisce diversi valori di ritorno ai processi padre e
figlio \--- un processo ``va dentro'' la chiamata \texttt{fork} e due processi ``vengono fuori'',
con differenti valori di ritorno. Il valore di ritorno nel processo padre � l'ID di processo del figlio.
Il valore di ritorno nel processo figlio � zero. Poich� nessun processo ha un ID di
processo pari a zero, ci� rende facile per il programma che sta girando sapere se � nel processo
padre o figlio.

Il listato 3.3 � un esempio di utilizzo di \texttt{fork} per duplicare il processo di un programma.
Nota che il primo blocco della condizione if � eseguito solo nel processo padre, mentre la
clausola else � eseguita nel processo figlio.\\

\listfromfile{fork.c}{Usare \textit{fork} per duplicare il processo di un programma}{list:3.3}
	{ALP-listings/chapter-3/fork.c}

\paragraph{Usare la famiglia \textit{exec}}

Le funzioni \texttt{exec} sostituiscono il programma in esecuzione in un processo con un altro
programma. Quando un programma chiama la funzione \texttt{exec}, quel processo cessa
immediatamente l'esecuzione di quel programma ed inizia ad eseguire un nuovo programma
dall'inizio, assumendo che la chiamata \texttt{exec} non incontri nessun errore.

All'interno della famiglia \texttt{exec} ci sono funzioni che variano leggermente nelle loro capacit�
e su come sono chiamate.
\begin{itemize}
\item{Funzioni che contengono la lettera \textit{p} nel loro nome (\texttt{execvp} ed \texttt{execpl}) accettano
il nome di un programma e cercano un programma con quel nome nella directory della
corrente esecuzione del programma; Alle funzioni che non contengono la p bisogna dare
il percorso completo del programma che deve essere eseguito.}
\item{Funzioni che contengono la lettera \textit{v} nel loro nome (\texttt{execv}, \texttt{execvp} ed \texttt{execve})
accettano la lista di argomenti del nuovo programma come un array terminato da NULL
di puntatori a stringhe. Le funzioni che contengono la letara \textit{l} (\texttt{execl}, \texttt{execlp} ed
\texttt{execle}) accettano la lista di argomenti usando il meccanismo varargs del linguaggio
C.}
\item{Funzioni che contengono la lettera \textit{e} nel loro nome (\texttt{execve} ed \texttt{execle}) accettano
un ulteriore argomento, un array di variabili di ambiente. L'argomento dovrebbe essere
un array terminato da NULL di puntatori a stringhe di caratteri. Ogni stringa di caratteri
dovrebbe essere nella forma ``VARIABILE=valore''.}
\end{itemize}

Poich� \texttt{exec} sostituisce il programma chiamante con un altro, esso non ritorna mai finch�
non viene incontrato un errore.

La lista degli argomenti passata al programma � analoga agli argomenti da riga di comando
che dai ad un programma quando lo esegui dalla shell. Essi sono disponibili tramite i parametri
del main \texttt{argc} ed \texttt{argv}. Ricorda che quando un programma viene invocato dalla shell, la
shell setta il primo elemento della lista degli argomenti (\texttt{argv[1]}) al primo argomento della
riga di comando e cos� via. Quando usi una funzione \texttt{exec} nel tuo programma, anche tu,
dovresti passare il nome della funzione come primo elemento della lista degli argomenti.

\paragraph{Usare \textit{fork} ed \textit{exec} insieme}

Un comune modello per eseguire un sottoprogramma all'interno di un programma � prima
fare il fork del processo e quindi exec del sottoprogramma. Ci� permette al programma
chiamante di continuare la propria esecuzione nel processo padre mentre il programma
chiamante � sostituito dal sottoprogramma nel processo figlio.

Il programma nel listato 3.4, come il listato 3.2, lista il contenuto della directory radice
usando il comando \texttt{ls}. Diversamente dall'esempio precedente, comunque, esso invoca il
comando ls direttamente, passandogli gli argomenti della riga di comando \texttt{-l} e \texttt{/} piuttosto
che invocarlo tramite una shell.\\

\listfromfile{fork-exec.c}{Usare \textit{fork} ed \textit{exec} insieme}{list:3.4}
	{ALP-listings/chapter-3/fork-exec.c}

\subsection{Scheduling dei processi}\label{subsec:3.2.3} % 3.2.3

Linux schedula i processi padre e figlio indipendentemente; non c'� nessuna garanzia su
quale dei due girer� prima o per quanto tempo esso girer� prima che Linux lo interrompa e
faccia girare l'altro processo (o qualche altro processo nel sistema). In particolare, nessuna
parte, oppure tutte, del comando ls, potrebbero girere nel processo foglio prima che il processo
padre non sia stato completato.\footnote{Un metodo per serializzare i due processi � presentato nella
\numnameref{subsec:3.4.1}.} Linux promette che ogni processo verr�
eventualmente eseguito \--- nessun processo verr� completamente impoverito delle risorse di
esecuzione.

Devi specificare che un processo � meno importante \--- e gli dovrebbe essere data una priorit�
pi� bassa \--- assegnandogli un valore alto di \textit{niceness} (futilit�). Per default, ogni processo ha un
niceness pari a zero. Un alto valore di niceness indica che al processo viene data una priorit� di
esecuzione minore; viceversa, un processo con un basso (negativo) niceness ottiene maggior
tempo di esecuzione.

Per eseguire un programma con un niceness diverso da zero, usa il comando \texttt{nice},
specificando il valore niceness con l'opzione \texttt{-n}. Per esempio, questo � come dovresti invocare
il comando ``\texttt{sort input.txt > output.txt}'', una lunga operazione di ordinamento
con una priorit� ridotta in modo che non rallenti troppo il sistema:
\begin{listcodeBash}
% nice -n 10 sort input.txt > output.txt
\end{listcodeBash}

Puoi usare il comando \texttt{renice} da riga di comando per cambiare il niceness di un processo in
esecuzione.

Per cambiare il niceness di un processo in esecuzione programmaticamente, usa la funzione
\texttt{nice}. Il suo argomento � un valore di incremento, che � aggiunto al valore niceness del pocesso
che esso chiama. Ricorda che un valore positivo incrementa il valore niceness e quindi riduce la
priorit� di esecuzione del processo.

Nota che solo un processo con privilegi di root pu� eseguire un processo con un valore
niceness negativo o ridurre il valore niceness di un processo in esecuzione. Ci� significa che
puoi specificare valori negativi ai comandi nice e renice solo quando sei loggato come root, e
solo un processo che � in esecuzione come root pu� passare valori negativi alla funzione nice.
Ci� evita che gli utenti ordinari si accaparrino la priorit� di esecuzione togliendola ad altri che
stanno usando il sistema.

\section{Segnali}\label{sec:3.3} % 3.3

I segnali sono meccanismi per comunicare e manipolare i processi in Linux. L'argomento dei
segnali � molto ampio; qui discutiamo su alcuni dei pi� importanti segnali e tecniche che sono
usate per controllare i processi.

Un segnale � un messaggio speciale inviato ad un processo. I segnali sono asincroni; quando
un processo riceve un segnale, esso processa il segnale immediatamente, senza completare la
funzione corrente o anche la corrente riga di codice. Ci sono diverse dozzine di segnali divesi,
ognuno con un diverso significato. Ogni tipo di segnale � specificato dal suo numero di segnale,
ma nei programmi, di solito si fa riferimento ai segnali tramite il loro nome. Su Linux, questi
sono definiti in \texttt{/usr/include/bits/signum.h} (Non dovresti includere questo file
header direttamente nei tuoi programmi, piuttosto usa \texttt{<signal.h>}.)

Quando un processo riceve un segnale, pu� fare una delle diverse cose, dipendentemente
dalle disposizioni del segnale. Per ogni segnale ci sono disposizioni di default, che determinano
cosa accade al processo se il programma non specifica alcuni altri comportamenti. Per molti tipi
di segnali, un programma pu� specificare alcuni altri comportamenti \--- tra ignorare il segnale o
chiamare una funzione speciale \texttt{signal-handler} per rispondere al segnale. Se viene usato un
gestore di segnale, l'esecuzione corrente del programma � messa in pausa, viene eseguito
gestore del segnale e quando il gestore del segnale ritorna il programma riprende.

Il sistema Linux invia i segnali ai processi in risposta a condizioni specifiche. Per esempio,
\texttt{SIGBUS} (errore del bus), \texttt{SIGSEGV} (segmentation violation) e \texttt{SIGFPE} (floating point
exception) possono essere inviati ad un processo che tenta di eseguire un'operazione illegale. Le
disposizioni di default di questi segnali tentano di terminare il processo e produrre un file
riassuntivo.

Un processo pu� anche inviare un segnale ad un altro processo. Un uso comune di questo
meccanismo � quello di terminare un altro processo inviandogli un segnale \texttt{SITGERM} o
\texttt{SIGKILL}.\footnote{Qual'� la differenza? Il segnale \texttt{SIGTERM} chiede ad un processo di terminare; il
processo pu� ignorare la richiesta mascherando o ignorando il segnale. Il segnale \texttt{SIGKILL}
uccide sempre il processo immediatamente poich� il processo non pu� mascherare o ignorare il
\texttt{SIGKILL}.} Un altro uso comune � di inviare un comando ad un programma in esecuzione. Due
segnali ``userdefined'' (definiti dall'utente) sono riservati per questo scopo: \texttt{SIGUSR}1 e
\texttt{SIGUSR2}. Qualche volta viene usato per questo scopo anche il segnale \texttt{SIGHUP},
comunemente per svegliare un programma in pausa o causare ad un programma la rilettura del
suo file di configurazione.

La funzione \texttt{sigaction} pu� essere usata per settare una disposizione di un segnale. Il
primo parametro � il numero del segnale. I successivi due parametri sono puntatori a strutture
\texttt{sigaction}; il primo di essi contiene la disposizione desiderata per quel numero di segnale
mentre il secondo riceve la disposizione precedente. Il campo pi� importante nella prima o
seconda struttura \texttt{sigaction} � \texttt{sa\_handler}. Esso pu� prendere uno dei tre valori:
\begin{itemize}
\item{\texttt{SIG\_DFL}, che specifica la disposizione di default per il segnale.}
\item{\texttt{SIG\_IGN}, che specifica che il segnale dovrebbe essere ignorato.}
\item{Un puntatore ad una funzione che gestisce un segnale (signal-handler). La funzione
dovrebbe prendere un parametro, il numero di segnale e restituire un \texttt{void}.}
\end{itemize}

Poich� i segnali sono asincroni, il programma principale pu� trovarsi in uno stato molto fragile
nel momento in cui viene processato un segnale e quindi mentre viene eseguita una funzione che
gestisce il segnale.

Un gestore di segnale dovrebbe effettuare il minimo lavoro necessario per rispondere al
segnale e quindi restituire il controllo al programma principale (o terminare il programma). In
molti casi, esso consiste semplicemente nel registrare il fatto che si � ricevuto un segnale. Il
programma principale quindi verifica periodicamente se � stato ricevuto un segnale e reagisce di
conseguenza.

� possibile che un gestore di segnale venga interrotto dall'invio di un altro segnale.
Anche se pu� sembrare un'occorrenza rara, se accade, sar� molto difficile diagnosticare e fare il
debug del problema. (Questo � un esempio di una condizione rara, discussa nel \numnameref{cap:4}, \numnameref{sec:4.4}.
Quindi, dovresti stare molto
attento su cosa fa il tuo programma in un gestore di segnale.

Anche assegnare un valore ad una variabile globale pu� essere pericoloso poich�
l'assegnazione potrebbe essere effettuata in due o pi� istruzioni, e tra una e l'altra potrebbe
presentarsi un secondo segnale, lasciando la variabile in uno stato corrotto. Se usi una variabile
globale per memorizzare un segnale da una funzione gestore di segnali, essa dovrebbe essere
del tipo speciale \texttt{sig\_atomic\_t}. Linux garantisce che gli assegnamenti a variabili di questo
tipo sono effettuati in un'istruzione singola e quindi non possono essere interrotti a met� strada.
In linux, \texttt{sig\_atomic\_t} � un \texttt{int} ordinario; infatti, gli assegnamenti ad interi che
rappresentano la dimensione di un \texttt{int} o pi� piccolo, o a puntatori, sono atomici. Se vuoi
scrivere un programma che sia portabile verso ogni sistema UNIX standard, comunque, usa
\texttt{sig\_atomic\_t} per queste variabili globali.

Questo abbozzo di programma nel listato 3.5, per esempio, usa una funzione gestore di
segnali per contare il numero di volte che il programma riceve \texttt{SIGUSR1}, uno dei segnali
riservati per l'utilizzo nella applicazioni.\\

\listfromfile{sigusr1.c}{Usare un gestore di segnale}{list:3.5}
	{ALP-listings/chapter-3/sigusr1.c}

\section{Terminazione dei processi}\label{sec:3.4} % 3.4

Normalmente, un processo termina in uno di due modi. O il programma in esecuzione chiama
la funzione \texttt{exit}, o la funzione \texttt{main} del programma finisce.
Ogni processo ha un codice di uscita: un numero che il processo restituisce al suo genitore.
Il codice di uscita � l'argomento passato alla funzione \texttt{exit}, o il valore restituito dal \texttt{main}.

Un processo pu� anche terminare non normalmente, in risposta ad un segnale. Per esempio,
i segnali \texttt{SIGBUS}, \texttt{SIGSEGV} e \texttt{SIGFPE} menzionati precedentemente
causano la terminazione del processo. Altri segnali sono usati per terminare il processo
esplicitamente. Il segnale \texttt{SIGINT} � inviato
ad un processo quando l'utente cerca di terminarlo premendo Ctrl+C nel proprio terminale.
Il segnale \texttt{SIGTERM} � inviato dal comando \texttt{kill}. La direttiva di default per
entrambi � quella di teminare il processo. Chiamando la funzione \texttt{abort}, un processo
stesso si manda il senale \texttt{SIGABRT}, che emina il processo e produce un file core.
Il segnale di terminazione pi� potente � \texttt{SIGKILL}, che termina un processo
immediatamente e non pu� essere bloccato o gestito da un programma.

Ognuno di questi segnali pu� essere inviato usando il comando \texttt{kill} specificando
un argomento extra da riga di comando; per esempio, per terminare un processo che ha dei
problemi inviandogli un \texttt{SIGKILL}, invoca la seguente, dove \texttt{pid} � il suo ID di processo:
\begin{listcodeBash}
% kill -KILL pid
\end{listcodeBash}
Per inviare un segnale da un programma, usa la funzione \texttt{kill}. Il primo parametro
� l'ID del processo obiettivo. Il secondo parametro � il numero di segnale; usa \texttt{SIGTERM}
per simulare il comportamento di default del comando \texttt{kill}. Per sempio, dove
\texttt{child pid} contiene l'iID di processo del processo figlio, puoi usare la funzion
\texttt{kill} per terminare un processo figlio da un processo padre chiamandolo in un
modo simile a questo:
\begin{listcodeBash}
kill (child_pid, SIGTERM);
\end{listcodeBash}
Se usi la funzione \texttt{kill} includi gli header \texttt{<sys/types.h>} e \texttt{<signal.h>}.

Per convenzione, il codice di uscita � usato per indicare se un programma � stato eseguito
correttamente. Un codice di uscita pari a zero indica la corretta esecuzione, mentre un codice
di uscita diverso da zero indica che c'� stato un errore. Nell'ultimo caso, il valore particolare
restituito pu� dare alcune indicazione sulla natura dell'errore. Sarebbe una buona idea riportare
questa convenzione nei tuoi programmi poich� altri componenti del sistema GNU/Linux assumono
che ci sia questo comportamento. Per esempio, la shell assume la presenza di questa convenzione
quando colleghi pi� programmi con gli operatori \texttt{\&\&} (and logico) e \texttt{||} (or logico.
Quindi, dovresti restituire zero dal \texttt{main} a meno che non si verifichi un errore.

Con molte shell � possibile ottenere il codice di uscita del programma eseguito pi� recentemente
usando la variabile speciale \texttt{\$?}. Ecco un esempio nel quale il comando \texttt{ls} � invocato
due volte ed � mostrato il codice di uscita dopo ogni chiamata. Nel primo caso, \texttt{ls} viene
eseguito correttamente e restituisce il codice di uscita zero. Nel secondo caso, \texttt{ls} incontra
un errore (perch� il nome del file specificato sulla riga di comando non esiste) e quindi restituisce
un codice di uscita diverso da zero.
\begin{listcodeBash}
% ls /
bin   coda  etc   lib         misc  nfs proc  sbin  usr
boot  dev   home  lost+found  mnt   opt root  tmp   var
% echo $?
0
% ls bogusfile
ls: bogusfile: No such file or directory
% echo $?
1
\end{listcodeBash}
Nota che bench� il tipo di parametro della funzione \texttt{exit} sia un \texttt{int} e la funzione
\texttt{main} restituisca un \texttt{int}, Linux non riserva tutti i 32 bit del codice di ritorno.
In fatti, dovresti usare solo codici di uscita compresi tra zero e 127. I codici di uscita superiori
a 128 hanno uno speciale significato \--- quando un processo � terminato da un segnale, il
suo codice di uscita � 128 pi� il numero del segnale.

\subsection{Attendere che un processo termini}\label{subsec:3.4.1} % 3.4.1

Se hai scritto ed eseguito l'esempio \texttt{fork} ed \texttt{exec} nel listato 3.4, dovresti
aver notato che l'output del programma \texttt{ls} spesso appare dopo che il ``programma
principale'' � gi� stato completato. Ci� accade perch� il processo figlio, nel quale gira
\texttt{ls}, � schedulato indipendentemente dal processo padre. Poich� Linux � un sistema
operativo multitasking, entrambi i processi appaiono come eseguiti simultaneamente e non
puoi predire quando il programma \texttt{ls} avr� una possibilit� di essere eseguito, prima
o dopo che venga eseguito il processo padre.

In alcune situazioni, comunque, � preferibile che il processo padre aspetti fino a che uno o
pi� processi figli non siano stati completati. Ci� pu� essere fatto con la famiglia di chiamate
di sistema \texttt{wait}. Queste funzioni ti permettono di aspettare che un processo completi
la propria esecuzione ed abilitare il processo padre ad ottenere informazioni sulla terminazione
dei suoi processi figli. Ci sono quattro diverse chiamate di sistema nella famiglia \texttt{wait};
puoi scegliere di ottenere poche o molte informazioni sul processo che � terminato e puoi
scegliere se interessarti su quale processo figlio � stato terminato.

\subsection{Le chiamate di sistema \textit{wait}}\label{subsec:3.4.2} % 3.4.2

La funzione pi� semplice � chiamata semplicemente \texttt{wait}. Essa blocca il processo
chiamante fino a che uno dei suoi processi figli non esce (o non c'� un errore). Esso restituisce
un codice di stato tramite un argomento di tipo puntatore ad intero, dal quale puoi estrarre
informazioni su come il processo figlio � uscito. Per esempio, la macro \texttt{WEXITSTATUS}
estrae il codice di uscita del processo figlio.

Puoi usare la macro \texttt{WIFEXITED} per determinare dallo stato di uscita di un processo
figlio se il processo � terminato normalmente (tramite la funzione \texttt{exit} o uscendo dal
\texttt{main}) o � morto in seguito ad un segnale non gestito. Nell'ultimo caso, usa la macro
\texttt{WTERMSIG} per estrarre, dal suo stato di uscita, il numero di segnale per il quale � morto.

Qui c'� nuovamente la funzione \texttt{main} dall'esempio \texttt{fork} ed \texttt{exec}.
Questa volta, il processo padre chiama \texttt{wait} per attendere fino a che il processo
figlio, che � l'esecuzione del comando \texttt{ls}, non � finito.
\begin{listcodeC}
int main ()
{
  int child_status;

  /* The argument list to pass to the "ls" command. */
  char* arg_list[] = {
    "ls", /* argv[0], the name of the program. */
    "-l",
    "/",
    NULL /* The argument list must end with a NULL. */
};

/* Spawn a child process running the "ls" command. Ignore the
   returned child process ID. */
spawn ("ls", arg_list);
/* Wait for the child process to complete. */
wait (&child_status);
if (WIFEXITED (child_status))
  printf ("the child process exited normally, with exit code %d\n",
          WEXITSTATUS (child_status));
else
  printf ("the child process exited abnormally\n");

return 0;
}
\end{listcodeC}
In Linux sono disponibili molte chiamate di sistema simili, che sono pi� flessibili o
forniscono pi� informazioni sull'uscita del processo figlio. La funzione \texttt{waitpid}
pu� essere usata per attendere che finisca uno specifico processo figlio piuttosto che
ogni processo figlio. La funzione \texttt{wait3} restituisce statistiche di utilizzo della
CPU riguardo l'uscita del processo figlio e la funzione \texttt{wait4} ti permette di
specificare altre opzioni riguardo a quale processo attendere.

\subsection{Processi zombie}\label{subsec:3.4.3} % 3.4.3
Se un processo figlio termina mentre il suo genitore sta chiamando una funzione \texttt{wait},
il processo figlio scompare ed il suo stato di uscita � passato al suo padre tramite la chiamata \texttt{wait}.
Ma cosa accade quando un processo figlio termina ed il processo figlio non sta chiamando \texttt{wait}?
Esso semplicemente scopare? No, perch� l'informazione sulla sua terminazione \--- come quella se �
terminato normalmente e, se cos�, qual � il suo stato di uscita \--- andrebbe persa. Invece, quando un
processo figlio termina, diventa un processo zombie.

Un \textit{processo zombie} � un processo che � terminato ma non � stato ancora cancellato.
� responsabilit� del processo padre cancellare il proprio figlio zombie. Le funzioni \texttt{wait} 
anno anche questo, cos� non � necessario verificare se il tuo processo figlio sta ancora girando
prima di mettersi in attesa per questo. Supponi, per esempio, che un programma faccia il fork
in un processo figlio, esegua qualche altra operazione, e quindi chiami \texttt{wait}. Se il processo
figlio in quel momento non � ancora terminato, il processo padre rester� bloccato nella chiamata
\texttt{wait} fino a che il processo figlio non abbia finito. Se il processo figlio finisce prima che il
processo padre chiami \texttt{wait}, il processo figlio diventa uno zombie. Quando il processo
padre chiama \texttt{wait}, viene estratto lo stato di uscita del figlio zombie, il processo figlio �
cancellato e la chiamata \texttt{wait} ritorna immediatamente.

Cosa accade se il processo padre non cancella il proprio figlio? Essi stanno in giro per il sistema,
come processi zombie. Il programma nel listato 3.6 fa il fork in un processo figlio, che termina
immediatamente e quindi aspetta per un minuto senza mai cancellare il processo figlio.\\

\listfromfile{zombie.c}{Creare un processo zombie}{list:3.6}
	{ALP-listings/chapter-3/zombie.c}
Prova a compilare questo file in un eseguibile chiamato \texttt{make-zombie}. Eseguilo e,
mentre � ancora in esecuzione, visualizza l'elenco dei processi nel sistema invocando il
seguente comando in un'altra finestra:
\begin{listcodeBash}
% ps -e -o pid,ppid,stat,cmd
\end{listcodeBash}
Questo elenca gli ID di processo, ID di processo padre, stato del processo e riga di comando
del processo. Osserva che, in aggiunta al processo padre \texttt{make-zombie}, c'� un altro
processo visualizzato \texttt{make-zombie}. Questo � il processo figlio; nota che l'ID del suo
processo padre � l'ID del processo principale \texttt{make-zombie}. Il processo figlio � markato
come \texttt{<defunct>}, ed il suo codice di uscita � Z, che sta per zombie.

Cosa accade quando il programma principale \texttt{make-zombie} termina nel momento in cui
il processo padre esce senza aver mai chiamato \texttt{wait}? Il processo zombie sta ancora in giro?
No \--- prova ad eseguire nuovamente \texttt{ps} e nota che entrambi i processi
\texttt{make-zombie} sono andati via. Quando un programma termina, il suo figlio � ereditato
da un processo speciale, il programma \texttt{init}, che gira sempre con l'ID di processo pari
a 1 (Esso � il primo processo avviato quando Linux si avvia). Il processo \texttt{init} cancella
automaticamente ogni processo figlio zombie che esso eredita.

\subsection{Cancellare i figli in maniera asincrona}\label{subsec:3.4.4} % 3.4.4
Se stai usando un processo figlio, semplicemente per eseguire (\texttt{exec}) un altro programma,
� bene chiamare \texttt{wait} immediatamente nel processo padre, che si bloccher� fino a che il
processo figlio non � stato completato. Ma speso, vorrai che il processo padre continui la sua
esecuzione mentre uno o pi� figli vengono eseguiti in maniera asincrona. Come puoi essere
sicuro di cancellare i processi figli che hanno completato la loro esecuzione in modo da non
lasciare processi zombie, che consumano risorse di sistema, e stanno in giro?

Un approccio per il processo padre potrebbe essere chiamare \texttt{wait3} o \texttt{wait4}
periodicamente per cancellare figli zombie. Chiamare \texttt{wait} per questo scopo non
funziona bene perch�, se nessun figlio ha terminato, la chiamata bloccher� il processo padre
finch� uno dei figli non avr� terminato. Comunque, \texttt{wait3} e \texttt{wait4} prendono
un parametro flag aggiuntivo, al quale puoi passare il valore flag \texttt{WNOHANG}. Con questo
flag, la funzione gira in \textit{modalit� nonblocking} \--- esso canceller� un processo figlio
terminato, se c'�, o semplicemente dice se non ce ne sono. Il valore di ritorno della chiamata
� l'ID di processo del figlio terminato nel caso pi� comune, o zero nell'ultimo caso.

Una soluzione pi� elegante � notificare al processo padre quando un figlio termina. Ci sono molti
modi per farlo usando i metodi discussi nel \numnameref{cap:5}, ma fortunatamente Linux lo fa
al posto tuo, usando i segnali. Quando un processo figlio termina, Linux invia un segnale al processo
padre, il segnale \texttt{SIGCHILD}. La disposizione di default di questo segnale � quella di non fare
nulla, che � il motivo per il quale puoi non averlo notato prima.

Comunque, un modo semplice per cancellare i processi figli � catturando il segnale
\texttt{SIGCHILD}. Di certo, quando si cancella un processo figlio, � importante memorizzare
il suo stato di terminazione, se questa informazione � necessaria, perch� quando il processo
sar� stato cancellato usando \texttt{wait}, questa informazione non sar� pi� disponibile.
Il listato 3.7 mostra un esempio di un programma che usa il gestore \texttt{SIGCHILD} per
cancellare i suoi processi figli.\\

\listfromfile{sigchld.c}{Cancellare i figli gestendo SIGCHILD}{list:3.7}
	{ALP-listings/chapter-3/sigchld.c}
Nota come il gestore di segnale memorizza lo stato di uscita del processo figlio in una variabile
globale, per mezzo della quale il programma principale vi pu� accedere. Poich� la variabile �
assegnata in un gestore di segnale, il suo tipo � \texttt{sig\_atomic\_t}.











































% !TEX encoding = ISO-8859-1
% !TEX root = alp.tex
% !TEX program = pdflatex
% !TEX spellcheck = it_IT

\chapter{Threads}\label{cap:4}

\firstsentence{I}{thread, come i processi sono un meccanismo per permettere ad un programma}
di fare pi� di una cosa allo stesso tempo. Come con i processi, i thread appaiono
in esecuzione concorrente; Il kernel di Linux li schedula in maniera sincrona,
interrompendo ogni thread da un momento all'altro per dare agli altri la
possibilit� di andare in esecuzione.

Concettualmente, un thread esiste in un processo. I thread sono unit� di
esecuzione pi� piccole dei processi. Quando invochi un programma, Linux crea un
nuovo processo ed in quel processo crea un singolo thread, che esegue il
programma sequenzialmente. Questo thread pu� creare altri thread; tutti questi
thread eseguono lo stesso programma nello stesso processo, ma ogni thread pu�
essere l'esecuzione di una diversa parte del programma in qualsiasi momento.

Abbiamo visto come un programma pu� fare il fork in un processo figlio. Il
processo figlio inizialmente gira nel suo programma padre, con la memoria
virtuale del padre, descrittori di file e cos� via copiati. Il processo figlio
pu� modificare la propria memoria, chiudere descrittori di file, e simili, senza
effetti sul padre, e vice versa. Quando un programma crea un altro thread,
comunque, non viene copiato nulla. Il thread creatore ed il thread creato
condividono lo stesso spazio di memoria, descrittori di file ed altre risorse di
sistema con l'originale. Se un thread cambia il valore di una variabile, per
esempio, l'altro thread di conseguenza vedr� il valore modificato. Similmente,
se un thread chiude un descrittore di file, gli altri thread non potranno
leggere o scrivere in quel descrittore di file. Poich� un processo e tutti i
suoi thread possono eseguire solo un programma per volta, se ogni thread
all'interno di un processo chiama una delle funzioni \texttt{exec}, tutti gli
altri thread vengono terminati (il nuovo programma pu�, di certo, creare nuovi
thread).

GNU/Linux implementa le API di thread standard POSIX (conosciute come
\textit{pthreads}). Tutte le funzioni di thread e tipi di dati sono
dichiarati nel file header \texttt{<pthread.h>}. Le funzioni pthread non sono
incluse nella libreria C standard. Piuttosto, esse sono in \texttt{libpthread},
quindi devi aggiungere \texttt{-lpthread} alla riga di comando quando fai il
link del tuo programma.

\section{Creazione dei thread}\label{sec:4.1} % 4.1

Ogni thread in un processo � identificato da un \textit{thread ID}. Quando si fa
riferimento agli ID dei thread nei programmi C o C++, si usa il tipo
\texttt{pthread\_t}.

Dopo la creazione, ogni thread esegue una \textit{funzione thread}. Questa � una
funzione ordinaria e contiene il codice che il thread dovrebbe eseguire. Quando
la funzione ritorna, il thread esce. Su GNU/Linux, le funzioni thread prendono
un singolo parametro, di tipo \texttt{void*}, ed hanno un tipo di ritorno
\texttt{void*}. Il parametro � l'\textit{argomento del thread}: GNU/Linux passa
il valore nel thread senza guardarlo. Il tuo programma pu� usare questo
parametro per passare i dati ad un nuovo thread. Similmente, il tuo programma
pu� usare il valore di ritorno per passare dati da un thread esistente al suo
programma creatore.

La funzione \texttt{pthread\_create} crea un nuovo thread. La usi assieme ai
seguenti:
\begin{enumerate}
\item{Un puntatore alla variabile \texttt{pthread\_t}, nel quale � memorizzato
l'ID di thread del nuovo thread}
\item{Un puntatore ad un oggetto \textit{attributo thread}. Questo oggetto
controlla i dettagli di come i thread interagiscono con il resto del programma.
Se passi \texttt{NULL} come attributo di thread, verr� creato un thread con gli
attributi di thread di default. Gli attributi di thread sono discussi nella
\numnameref{subsec:4.1.5}.}
\item{Un puntatore alla funzione thread. Questo � una funzione puntatore
ordinaria, di questo tipo:
\begin{listcodeC}
void* (*) (void*)
\end{listcodeC}
\item{Un argomento di thread di tipo \texttt{void*}. Ogni cosa che passi �
semplicemente passata come argomento della funzione thread quando il thread
inizia la sua esecuzione.}
}
\end{enumerate}
Una chiamata a \texttt{pthread\_create} ritorna immediatamente ed il thread
originale continua ad eseguire le istruzioni seguenti la chiamata. Nel
frattempo, il nuovo thread inizia ad eseguire la funzione thread. Linux schedula
entrambi i thread in maniera asincrona e il tuo programma non deve fare
affidamento sull'ordine relativo nel quale sono eseguite le istruzioni nei due
threads.

Il programma nel listato 4.1 crea un thread che stampa continuamente \texttt{x} sullo
standard error. Dopo aver chiamato \texttt{pthread\_create}, il thread
principale stampa continuamente \texttt{o} sullo standard error.\\

\listfromfile{thread-create.c}{Creare un thread}{list:4.1}
	{ALP-listings/chapter-4/thread-create.c}

Compila e fai il link di questo programma usando il seguente codice:
\begin{listcodeBash}
% cc -o thread-create thread-create.c -lpthread
\end{listcodeBash}
Prova ad eseguirlo per vedere cosa accade. Nota il modello di \texttt{x} ed
\texttt{o} imprevedibile, come Linux schedula alternativamente i due thread.

In circostanze normali, un thread esce in uno di due modi. Un modo, come
illustrato precedentemente, � ritornando dalla funzione thread. Il valore di
ritorno dalla funzione thread � considerato essere il valore di ritorno del
thread. In alternativa, un thread pu� uscire esplicitamente chiamando
\texttt{pthread\_exit}. Questa funzione pu� essere chiamata dall'interno della
funzione thread. L'argomento di \texttt{pthread\_exit} � il valore di ritorno
del thread.

\subsection{Passare dati ai thread}\label{subsec:4.1.1} %4.1.1
L'argomento del thread fornisce un modo conveniente per passare i dati ai
thread. Poich� il tipo dell'argomento � \texttt{void*}, comunque, non puoi
passare molti dati direttamente tramite l'argomento. Piuttosto, usa l'argomento
del thread per passare un puntatore ad alcune strutture o array di dati. Una
tecnica comunemente usata � quella di definire una struttura per ogni funzione
thread, che contiene i ``parametri'' che la funzione thread si aspetta.

Usando l'argomento del thread, � facile riutilizzare la stessa funzione thread
per molti thread. Tutti questi thread eseguono lo stesso codice, ma con dati
diversi.

Il programma nel listato 4.2, � simile all'esempio precedente. Questa volta
vengono creati due nuovi thread, uno per stampare \texttt{x} e l'altro per stampare \texttt{o}.
Invece di stampare infinitamente, comunque, ogni thread stampa un numero fisso
di caratteri e quindi esce ritornando dalla funzione thread. La stessa funzione
thread, \texttt{char\_print}, � usata da entrambi i thread, ma ognuno �
configurato diversamente usando \texttt{struct} \texttt{char\_print\_parms}.\\

\listfromfile{thread-create2.c}{Creare due threads}{list:4.2}
	{ALP-listings/chapter-4/thread-create2-add1.c}

\textit{Ma aspetta!} Il programma nel listato 4.2 contiene un bug serio. Il
thread principale (che esegue la funzione \texttt{main}) crea le strutture del
parametro del thread (\texttt{thread1\_args} e \texttt{thread2\_args}) come
variabili loacali, e quindi passa i puntatori a queste strutture ai thread che
esso crea. Cosa fare per evitare che Linux scheduli i tre thread in modo che
\texttt{main} completi la sua esecuzione prima che uno degli altri due thread
sia completato? \textit{Nulla!} Ma se ci� accade, la memoria contenente le
strutture del parametro del thread verranno deallocate mentre gli altri due
thread staranno ancora accedendo ad esse. 

\subsection{Unire i thread}\label{subsec:4.1.2} % 4.1.2
Una soluzione � quella di forzare il \texttt{main} ad aspettare fino a che gli
altri due thread non siano compleatati. Ci� di cui abbiamo bisogno � una funzione
simile a \texttt{wait} che aspetti che finisca un thread piuttosto che un
processo. La funzione � \texttt{pthread\_join}, che prende due argomenti: l'ID
del thread da aspettare ed un puntatore ad una variabile \texttt{void*} che
ricever� il valore di ritorno del thread finito. Se non ti interessa il valore
di ritorno del thread, passa \texttt{NULL} come secondo argomento.

Il listato 4.3 mostra la funzione \texttt{main} corretta per l'esempio sbagliato
nel listato 4.2. In questa versione, il \texttt{main} non esce fino a che
entrambi i thread che stampano x e o non sono completati, cos� essi non useranno
pi� le strutture degli argomenti.\\

\listfromfile{thread-create2.c}
	{ Funzione \textit{Main} rivista per \textit{thread-create2.c}}{list:4.3}
	{ALP-listings/chapter-4/thread-create2-add2.c}

Morale della favola: Assicurati che ogni dato che passi a un thread per
riferimento non sia deallocato, \textit{anche da un diverso thread}, fino a che
non sei sicuro che il thread abbia finito con esso. Ci� � vero sia per variabili
locali, che sono deallocate quando esse vanno fuori dal proprio raggio di azione,
che per gruppi di variabili allocate in blocco che vengono deallocate chiamando
\texttt{free} (o usando \texttt{delete} in C++).

\subsection{Valori di ritorno dei thread}\label{subsec:4.1.3} % 4.1.3
Se il secondo argomento che passi a \texttt{pthread\_join} non � nullo, il
valore di ritorno del thread verr� messo nella locazione puntata da
quell'argomento. Il valore di ritorno del thread, come l'argomento del thread, �
di tipo \texttt{void*}. Se vuoi passare un singolo \texttt{int} o altri numeri
piccoli, puoi farlo facilmente tramite il cast del valore a \texttt{void*} e
quindi rifacendo il cast al tipo appropriato dopo aver chiamato
\texttt{pthread\_join}.\footnote{Nota che ci� non � portabile ed � compito tuo
assicurarti che per il valore pu� essere effettuato il cast a \texttt{void*} e
tornare indietro senza perdita di bit.}

Il programma nel listato 4.4 calcola l'ennesimo numero primo in un thread
separato. Quel thread restituisce il numero primo desiderato come suo valore di
ritorno di thread. Il thread principale, nel frattempo, � libero di eseguire
altro codice. Nota che l'agoritmo di divisione successivo usato in
\texttt{compute\_prime} � un po' inefficiente; se nel tuo programma hai bisogno
di calcolare molti numeri primi consulta un libro su algoritmi numerici.\\

\listfromfile{primes.c}{Calcola numeri primi in un thread}{list:4.4}
	{ALP-listings/chapter-4/primes.c}

\subsection{Altro sugli ID dei thread}\label{subsec:4.1.4} % 4.1.4
Occasionalmente, � utile per una sequenza di codice determinare quale thread lo
sta eseguendo. La funzione \texttt{pthread\_self} restituisce l'ID di thread del
thread nel quale � chiamata. Questo ID di thread pu� essere confrontato con
altri ID di thread usando la funzione \texttt{pthread\_equal}

Queste funzioni possono essere utili per determinare quando un particolare ID di
thread corrisponde al thread corrente. Per esempio, c'� un errore per un thread
nel chiamare \texttt{pthread\_join} per unirlo a se stesso. (In questo caso,
	\texttt{pthread\_join} restituirebbe il codice di errore
	\texttt{EDEADLK}.) Per verificare questa condizione in anticipo, puoi
usare del codice come questo:
\begin{listcodeC}
if (!pthread_equal (pthread_self (), other_thread))
  pthread_join (other_thread, NULL);
\end{listcodeC}

\subsection{Attributi dei thread}\label{subsec:4.1.5} % 4.1.5

Gli attributi dei thread forniscono un meccanismo per mettere a punto il
comportamento di thread individuali. Ricorda che \texttt{pthread\_create}
accetta un argomento che � un puntatore ad un oggetto attributo di thread. Se
passi un puntatore nullo, per configurare il nuovo thread vengono utilizzati
gli attributi di default dei thread. Comunque, puoi creare e personalizzare l'oggetto
attributo di un thread per specificare altri valori per gli attributi.

Per specificare gli attributi dei thread � necessario seguire i seguenti passi:
\begin{enumerate}
\item{Creare un oggetto \texttt{pthread\_attr\_t}. La maniera pi� facile �
dichiarare semplicemente una variabile automatica di questo tipo.}
\item{Chiamare \texttt{pthread\_attr\_init}, passando un puntatore a questo
oggetto. Ci� inizializza gli attributi ai loro valori di default.}
\item{Modificare l'oggetto attributo in modo che contenga i valori desiderati.}
\item{Passare un puntatore all'oggetto attributo quando si chiama
\texttt{pthread\_create}.}
\item{Chiamare \texttt{pthread\_attr\_destroy} per rilasciare l'oggetto
attributo. La variabile \texttt{pthread\_attr\_t} da sola non � deallocata; pu�
essere reinizializzata con \texttt{pthread\_attr\_init}.}
\end{enumerate}
Un singolo oggetto attributo di thread pu� essere usato per avviare diversi
thread. Non � necessario mantenere un oggetto attributo di thread in giro dopo
che i thread sono stati creati.

Per molti lavori di programmazione delle applicazioni in GNU/Linux,
� tipicamente di interesse un solo attributo di thread
(gli altri attributi disponibili servono principalmente per la programmazione
realtime specializzata). Questo
attributo � lo \textit{stato distaccato} (\textit{detached state}) del thread.
Un thread pu� essere creato
come un \textit{thread congiungibile} (di default) o come un \textit{thread
distaccato}.
Un thread congiungibile (\textit{joinable thread}), come un processo, non � cancellato
automaticamente da GNU/Linux quando termina. Invece, lo stato di uscita del
thread rimane in sospeso nel sistema (qualcosa di simile ad un processo zombie)
finch� un altro thread non chiama \texttt{pthread\_join} per ottenere il suo
valore di ritorno. Solo allora le sue risorse vengono rilasciate. Un thread
distaccato, di contro, viene cancellato automaticamente quando termina. Poich�
un thread distaccato � immediatamente cancellato, un altro thread pu� non
riuscire ad essere sincronizzato con il suo completamento usando
\texttt{pthread\_join} o ottenere il suo valore di ritorno.

Per impostare lo stato distaccato in un oggetto attributo di thread, usa
\texttt{pthread\_attr\_setdetachstate}. Il primo argomento � un puntatore
all'oggetto attributo del thread, e il secondo � lo stato distaccato desiderato.
Poich� lo stato joinable � quello di default, � necessario fare questa chiamata
solo per creare thread distaccati; passa \texttt{PTHREAD\_CREATE\_DETACHED} come
secondo argomento.

Il codice nel listato 4.5 crea un thread distaccato impostando l'attributo di
thread allo stato detach per il thread.\\

\listfromfile{detached.c}
	{Scheletro di un programma che crea un thread distaccato}{list:4.5}
	{ALP-listings/chapter-4/detached.c}

Anche se un thread � creato in uno stato joinable, pu� essere successivamente
cambiato in thread distaccato. Per fare ci�, chiama \texttt{pthread\_detach}.
Una volta che un thread diventa distaccato, non pu� pi� tornare ad essere
nuovamente congiungibile (\textit{joinable}).

\section{Cancellazione di thread}\label{sec:4.2} % 4.2
In circostanze normali, un thread finisce quando esce normalmente, sia
ritornando dalla sua funzione thread sia chiamando \texttt{pthread\_exit}.
Comunque, � possibile per un thread richiedere che un altro thread di termini.
Ci� � chiamato \textit{cancellazione} (\textit{canceling}) di un thread.

Per cancellare un thread, chiama \texttt{pthread\_cancel} passandogli l'ID del
thread che deve essere cancellato. Un thread cancellato pu� essere
successivamente congiunto (joined); infatti, dovresti collegarti ad un thread
cancellato per liberarne le risorse, fino a che il thread � disgiunto (vedi
\numnameref{subsec:4.1.5}). Il valore di ritorno di un thread cancellato � il
valore speciale dato da \texttt{PTHREAD\_CANCELED}.

Spesso un thread pu� trovarsi in del codice che pu� essere eseguito in maniera
completa o nulla, senza vie di mezzo. per esempio, il thread pu� allocare delle risorse,
usarle e quindi deallocarle. Se il thread viene cancellato nel mezzo di questo
codice, pu� non avere l'opportunit� di dellocare le risorse e quindi le risorse
resteranno bloccate. Per cercare di evitare che ci� accada � possibile per un
thread controllare se e quando esso pu� essere cancellato.

Un thread, per quanto riguarda la cancellazione, pu� trovarsi in uno di tre
stati.
\begin{itemize}
\item{Il thread pu� essere \textit{cancellabile in modo asincrono}
	(\textit{asynchronously cancelable}). Il thread pu� essere cancellato
	in ogni punto della sua esecuzione.}
\item{Il thread pu� essere \textit{cancellabile in modo sincrono}
	(\textit{synchronously cancelable}). Il thread pu� essere cancellato,
	ma non proprio ad ogni punto della sua esecuzione. Piuttosto, le
	richieste di cancellazione sono accodate e il thread � cancellato solo
	quando raggiunge punti specifici nella sua esecuzione.}
\item{un thread pu� essere \textit{incancellabile} (\textit{uncancelable}).
	I tentativi di cancellare il thread sono silenziosamente ignorati}
\end{itemize}
Quando viene inizialmente creato, un thread � cancellabile in modo sincrono.

\subsection{Thread sincroni ed asincroni}\label{subsec:4.2.1} % 4.2.1
Un thread cancellabile in modo asincrono pu� essere cancellato in ogni punto
della sua esecuzione. Un thread cancellabile in modo sincrono, di contro, pu�
essere cancellato solo in particolari punti della sua esecuzione. Questi punti
sono chiamati \textit{punti di cancellazione} (\textit{cancellation points}).
Nel thread verr� accodata una richiesta di cancellazione fino a che esso non
raggiunge il prossimo punto di cancellazione.

Per rendere un thread cancellabile in modo asincrono, usa
\texttt{pthread\_setcanceltype}. Ci� ha effetto sui thread che attualmente
chiamano le funzioni. Il primo argomento dovrebbe essere
\texttt{PTHREAD\_CANCEL\_ASYNCHRONOUS} per rendere il thread cancellabile in
modo asincrono, o \texttt{PTHREAD\_CANCEL\_DEFERRED} per riportarlo nello stato
di cancellabile in modo sincrono. Il secondo argomento, se diverso da null, � un
puntatore ad una variabile che riceve il tipo di cancellazione precedente per il
thread. Questa chiamata, per esempio, rende il thread chiamante cancellabile in
modo asincrono.
\begin{listcodeC}
pthread_setcanceltype (PTHREAD_CANCEL_ASYNCHRONOUS, NULL);
\end{listcodeC}
Cosa costituisce un punto di cancellazione e dove dovrebbero essere posti? La
via pi� diretta per creare un punto di cancellazione � chiamare
\texttt{pthread\_testcancel}. Questo non fa nulla ad eccetto il fatto di
processare una cancellazione in sospeso in un thread cancellabile in modo
sincrono. Dovresti chiamare \texttt{pthread\_testcancel} periodicamente durante
i calcoli pi� lunghi in una funzione thread, nei punti in cui il thread pu�
essere cancellato senza bloccare nessuna risorsa o produrre altri effetti
negativi.
Certe altre funzioni hanno implicitamente dei punti di cancellazione. Questi
sono elencati nella pagina di manuale di \texttt{pthread\_cancel}. Nota che
altre funzioni possono usare queste funzioni internamente e quindi ci saranno
indirettamente dei punti di cancellazione.

\subsection{Sezioni critiche non cancellabili}\label{subsec:4.2.2} % 4.2.2
Un thread pu� disabilitare la cancellazione di se stesso completamente con la
funzione \texttt{pthread\_setcancelstate}. Come per
\texttt{pthread\_setcanceltype}, questa ha effetto sul thread chiamante. Il
primo argomento � \texttt{PTHREAD\_CANCEL\_DISABLE} per disabilitare la
cancellazione, o \texttt{PTHREAD\_CANCEL\_ENABLE} per riabilitare la
cancellazione. Il secondo argomento, se diverso da null, punta ad una variabile
che ricever� lo stato precedente di cancellazione. Questa chiamata, per esempio,
disabilita la cancellazione del thread nel thread chiamante.
\begin{listcodeC}
pthread_setcancelstate (PTHREAD_CANCEL_DISABLE, NULL);
\end{listcodeC}
L'uso di \texttt{pthread\_setcancelstate} ti permette di implementare
\textit{sezioni critiche} (\textit{critical section}). Una sezione critica � una
sequenza di codice che pu� essere eseguito solo interamente o per nulla; in
altre parole, se un thread inizia ad eseguire la sezione critica, deve
continuare fino alla fine della sezione critica senza essere cancellato.

Per esempio, supponi di stare scrivendo una routine per una programma bancario
che trasferisce denaro da un conto ad un altro. Per farlo, devi aggiungere
valori sul bilancio di un conto e detrarre gli stessi valori dal bilancio di un
altro conto. Se succede che il thread che sta eseguendo la tua routine � stato
cancellato proprio nel momento sbagliato, tra le due operazioni, il programma
potrebbe avere incrementato falsamente il totale del deposito bancario facendo
fallire il completamento della transazione. Per prevenire questa possibilit�
poni le due operazioni in una sezione critica.

Puoi implementare il trasferimento con una funzione come
\texttt{process\_transaction}, mostrata nel listato 4.6. Questa funzione
disabilita la cancellazione del thread per avviare una sezione critica prima di
modificare uno dei bilanci del conto.\\

\listfromfile{critical-section.c}
	{Protegge una transazione bancaria con una sezione critica}{list:4.6}
	{ALP-listings/chapter-4/critical-section.c}

Nota che � importante ripristinare il vecchio stato di cancellato alla fine
della sezione critica piuttosto che impostarlo incondizionatamente a
\texttt{PTHREAD\_CANCEL\_ENABLE}. Ci� ti permette di chiamare la funzione
\texttt{process\_transaction} senza problemi dall'interno di un'altra sezione
critica \--- in questo caso, la tua funzione lascer� lo stato cancel allo stesso
modo di come � stato trovato.

\subsection{Quando usare la cancellazione di thread}\label{subsec:4.2.3} % 4.2.3
In genere, � una buona idea non usare la cancellazione di thread per terminare
l'esecuzione di un thread, tranne che in circostanze particolari. Durante una
normale operazione, la migliore strategia � quella di comunicare al thread che
dovrebbe uscire, e quindi aspettare che il thread esca da solo in maniera
ordinata. Discuteremo delle tecniche per comunicare con i thread pi� avanti in
questo capitolo e nel
\numnameref{cap:5}.

\section{Dati specifici dei thread}\label{sec:4.3} % 4.3
Diversamente dai processi, tutti i thread in un singolo programma condividono lo
stesso spazio di indirizzi. Ci� significa che se un thread modifica una
locazione di memoria (per esempio, una variabile globale), la modifica �
visibile a tutti gli altri thread. Ci� permette a diversi thread di operare
sugli stessi dati senza usare i meccanismi della comunicazione tra processi
(che sono descritti nel \autoref{cap:5}).

Ogni thread, comunque, ha la sua lista di chiamate. Ci� permette ad ogni thread
di eseguire diversi pezzi di codice e di chiamare e ritornare da subroutines nel
suo modo usuale. Come in un programma a thread singolo, ogni invocazione di una
subroutine in ogni thread ha il proprio insieme di variabili locali, che sono
memorizzate nella lista (stack) per quel trhead.

A volte, comunque, � desiderabile duplicare certe variabili in modo che ogni
thread abbia una copia separata. GNU/Linux lo permette fornendo ad ogni thread
un'area dei \textit{dati sepcifici dei thread}. Le variabili memorizzante in
quest'area sono duplicate per ogni thread, ed ogni thread pu� modificare la sua
copia di una variabile senza avere effetto sugli altri thread. Poich� tutti i
thread condividono lo stesso spazio di memoria, i dati specifici dei thread non
possono essere accessibili usando normali variabili di riferimento. GNU/Linux
fornisce funzioni speciali per assegnare e ritrovare valori dall'area di dati
specifica dei thread.

Puoi creare tanti elementi di dati specifici per i thread quanti ne vuoi, ognuno
di tipo \texttt{void*}. ogni elemento � referenziato da una chiave. Per creare
una nuova chiave, e quindi un nuovo elemento di dati per ogni thread, usa
\texttt{pthread\_key\_create}. Il primo argomento � un puntatore a una variabile
\texttt{pthrea\_key\_t}. Questo valore della chiave pu� essere usato da ogni
thread per accedere alla propria copia dell'elemento di dati corrispondente. Il
secondo argomento di \texttt{pthread\_key\_t} � una funzione di pulizia. Se qui
passi un puntatore a funzione, GNU/Linux chiama automaticamente quella funzione
quando ogni thread esce, passando il valore specifico di thread corrispondente
a quella chiave. Questo � particolarmente comodo perch� la funzione di pulizia �
chiamata anche se il thread � cancellato ad un punto arbitrario durante la sua
esecuzione. Se il valore specifico di thread � null, la funzione di pulizia dei
thread non viene chiamata. Se non hai bisogno di una funzione di pulizia, puoi
passare null piuttosto che un puntatore a funzione.

Dopo che hai creato una chiave, ogni thread pu� impostare il suo valore
specifico del thread corrispondente a quella chiave, chiamando
\texttt{pthread\_setspecific}. Il primo argomento � la chiave ed il secondo � il
valore specifico di thread \texttt{void*} da memorizzare. Per ritrovare un
elemento di dati specifico di thread, chiama \texttt{pthread\_getspecific},
passando la chiave come suo argomento.

Supponi, per esempio, che la tua applicazione divida un compito tra diversi
thread.
Per scopi di verifica, ogni thread deve avare un file di log separato, nel quale
vengono registrati messaggi di esecuzione per i compiti di quel thread. L'area
di dati specifica del thread � una zona conveniente per memorizzare il puntatore
di file per il file di log di ogni thread individuale.

Il listato 4.7 mostra come dovresti implementarlo. La funzione \texttt{main}
in questo programma di esempio crea una chiave per memorizzare il puntatore di
file specifico del thread e quindi lo memorizza in \texttt{thread\_log\_key}.
Poich� questa � una variabile globale, � condivisa da tutti i thread. Quando
ogni thread comincia ad eseguire la sua funzione thread, esso apre un file di log
e memorizza il puntatore a file in quella chiave. Successivamente, ognuno di
questi thread pu� chiamare \texttt{write\_to\_thread\_log} per scrivere un
messaggio nel file di log specifico del thread. Questa funzione ritrova il
puntatore di file per il file di log del thread dai dati specifici del thread e
scrive il messaggio.\\

\listfromfile{tsd.c}
	{Files di Log implementati per ogni thread con i dati specifici dei thread}
	{list:4.7}
	{ALP-listings/chapter-4/tsd.c}

Nota che \texttt{thread\_funcion} non ha bisogno di chiudere il file di log. Ci�
accade perch� quando la chiave file di log � stata creata, � stato specificato
\texttt{close\_thread\_log} come funzione di pulizia per quella chiave. Ogni
volta che un thread esce, GNU/Linux chiama quella funzione, passando il valore
specifico di thread per la chiave log del thread. Questa funzione si preoccupa
di chiudere il file di log.

\subsection{Gestori di pulizia}\label{subsec:4.3.1} % 4.3.1
Le funzioni di pulizia per le chiavi dei dati specifiche dei thread possono
risultare molto comodi per assicurare che le risorse non vengano disperse quando
un thread esce o � cancellato. A volte, comunque, pu� risultare utile
specificare delle funzioni di pulizia senza creare un nuovo elemento di dati
specifico di thread che � duplicato per ogni thread. GNU/Linux fornisce per
questo scopo dei \textit{gestori di pulizia} (\textit{cleanup handlers}).

Un gestore di pulizia � semplicemente una funzione che dovrebbe essere invocata
quando un thread esce. Il gestore prende un solo parametro \texttt{void*} ed il
valore del suo argomento � fornito quando il gestore � registrato \--- Ci� rende
facile usare la stessa funzione di gestione per deallocare pi� istanze di risorse.

Un gestore di pulizia � una misura temporanea, usata per deallocare una risorsa
solo se il thread esce o � cancellato senza aver terminato l'esecuzione di una
particolare porzione di codice.
In circostanze normali, quando il thread non esce e non � cancellato, le risorse
dovrebbero essere deallocate esplicitamente e il gestore di pulizia dovrebbe
venire rimosso.

Per registrare un gestore di pulizia, chiama \texttt{pthread\_cleanup\_push},
passando un puntatore alla funzione di pulizia ed il valore \texttt{void*} del
suo argomento. La chiamata a \texttt{pthread\_cleanup\_push} deve essere
bilanciata da una corrispondente chiamata a \texttt{pthread\_cleanup\_pop}, che
termina il gestore di pulizia. Per comodit�, \texttt{pthread\_cleanup\_pop}
prende un argomento \texttt{int}; se l'argomento � diverso da zero, l'azione di
pulizia � eseguita e quindi chiusa.

Il frammento di programma nel listato 4.8 mostra come dovresti usare un gestore
di pulizia per assicurarti che un buffer allocato dinamicamente venga ripulito se
il thread termina.\\

\listfromfile{cleanup.c}
	{Frammento di programma che dimostra un gestore di pulizia di thread}
	{list:4.8}
	{ALP-listings/chapter-4/cleanup.c}

Poich� l'argomento di \texttt{pthread\_cleanup\_pop} in questo caso non � zero,
la funzione di pulizia
\texttt{deallocate\_buffer} viene chiamata automaticamente qui e non ha bisogno
di essere chiamata
esplicitamente. In questo semplice caso, potremmo aver usato direttamente la
funzione di
libreria standard \texttt{free} come nostro gestore di pulizia al posto di
\texttt{deallocate\_buffer}.


\subsection{Pulizia del Thread in C++}\label{subsec:4.3.2} % 4.3.2
I programmatori C++ sono abituati ad ottenere la pulizia ``gratuitamente''
raggruppando
le azioni di pulizia in oggetti distruttori. Quando gli oggetti vanno al di
fuori del loro campo di operabilit�, o perch� un blocco � eseguito fino al
completamento o perch� viene lanciata un'eccezione, il C++ si assicura che i
distruttori siano chiamati per quelle variabili automatiche che loro hanno. Ci�
fornisce un meccanismo manuale per assicurarsi che il codice di pulizia venga
chiamato, senza curarsi di come il blocco � andato in uscita.

Se un thread chiama \texttt{pthread\_exit}, tuttavia, il C++ non garantisce che
i distruttori siano chiamati per tutte le variabili automatiche nello stack del
thread. Un modo ingegnoso per recuperare questa funzionalit� � quello di
invocare \texttt{pthread\_exit} al livello alto della funzione thread lanciando
una speciale eccezione.

Il programma nel listato 4.9 lo dimostra. Usando questa tecnica, una funzione
indica la sua intenzione di uscire dal thread lanciando una
\texttt{ThreadExitException} piutossto
che chiamare direttamente \texttt{pthread\_exit}. Poich� l'eccezione � catturata
nella funzione del thread ad alto livello, tutte le variabili locali nello stack
del trhread verranno distrutte appropriatamente quando l'eccezione si propaga.

\listfromfile{cxx-exit.cpp}
	{Implemente l'uscita sicura dal Thread con le eccezioni C++}{list:4.9}
	{ALP-listings/chapter-4/cxx-exit.cpp}

\section{Sincronizzazione e Sezioni Critiche}\label{sec:4.4} % 4.4
La programmazione con i thread � molto complessa perch� molti programmi con i
thread sono programmi concorrenti. In particolare, non c'� modo di sapere quando
il sistema scheduler� un thread da eseguire e quando ne eseguir� un altro. Un
thread pu� girare per un tempo molto lungo, o il sistema pu� commutare tra i
thread molto velocemente. In un sistema con pi� processori, il sistema pu� anche
schedulare thread multipli da eseguire letteralmente allo stesso tempo.

Debuggare un programma con thread � difficile perch� non puoi sempre e
facilmente riprodurre il comportamento che ha causato il problema. Puoi eseguire
il programma una volta ed avere ogni cosa che funziona bene; la prossima volta
che lo esegui pu� andare in crash. Non c'� verso di fare in modo che il sistema
scheduli i thread esattamente allo stesso modo in cui lo ha fatto prima.

La maggior causa di molti bug che riguardano i thread � che i thread cercano di
accedere agli stessi dati. Come detto precedentemente, questo � uno dei pi�
potenti aspetti dei thread, ma pu� anche essere pericoloso. Se un thread � solo
a met� strada dell'aggiornamento di una struttura dati mentre un altro thread
accede alla stessa struttura dati, � facile che ne venga fuori il caos. Spesso,
programmi con thread buggati contengono codice che funzioner� solo se un thread
viene schedulato pi� spesso \--- o pi� raramente \--- di un altro trhead. Questi
bug sono chiamati \textit{condizioni in competizione} (\textit{race conditions});
I thread stanno gareggiando con altri per modificare la stessa struttura dati.

\subsection{Condizioni in competizione}\label{subsec:4.4.1} % 4.4.1
Supponi che il tuo programma abbia usa serie di lavori in coda che vengono
processati da molti thread concorrenti. La coda di lavori � rappresentata da una
lista collegata di oggetti \texttt{struct job}.

Dopo che ogni thread termina un'operazione, esso controlla la coda per vedere se
� disponibile un altro lavoro. Se \texttt{job\_queue} � diverso da null, il
thread rimuove l'intestazione della lista collegata e setta \texttt{job\_queue}
al prossimo lavoro nella lista.

La funzione thread che processa il lavoro nella coda dovrebbe somigliare al
listato 4.10

\listfromfile{job-queue1.c}
	{Funzione Thread per processare un lavoro dalla coda}
	{list:4.10}
	{ALP-listings/chapter-4/job-queue1.c}

Adesso supponi che a due thread accada di finire un lavoro pi� o meno allo
stesso tempo, ma un solo lavoro rimanga in coda. Il primo thread verifica se
\texttt{job\_queue} � null; vedendo che non lo �, il thread entra nel loop e
memorizza il puntatore all'oggetto lavoro in \texttt{netx\_job}. A questo punto,
succede che Linux interrompe il primo thread e schedula il secondo. Anche il
secondo thread verifica \texttt{job\_queue} vedendo che non � null, assegna pure
lo stesso puntatore lavoro a \texttt{next\_job}. Per la sfortunata coincidenza,
adesso abbiamo due thread che eseguono lo stesso lavoro.

Per rendere la situazione peggiore, un thread scollegher� l'oggetto lavoro dalla
coda, lasciando \texttt{job\_queue} contenente \texttt{null}. Quando l'altro
thread valuta \texttt{job\_queue->next}, ne risulta un errore di segmentazione.

Questo � un esempio di una condizione di competizione. In circostanze
``fortunate'', questo particolare schedulamento dei due thread pu� non
verificarsi mai o la condizione di competizione pu� non mostrarsi mai. Solo in
diverse circostanze, probabilmente quando eseguito in un sistema molto caricato
(o in un nuovo server multiprocessore di un cliente importante) il bug si pu�
mostrare.

Per eliminare le condizioni di competizione � necessario un modo per fare
operazioni \textit{atomiche} (\textit{atomic}). Un'operazione atomica �
indivisibile e non interrompibile; una volta che l'operazione si avvia non pu�
essere messa in pausa o interrotta finch� non � stata completa, e nessun'altra 
operazione avr� luogo nel frattempo. In questo particolare esempio verificherai
\texttt{job\_queue}; se esso non � vuoto, rimuovi il primo lavoro, tutto come
una singola operazione atomica.

\subsection{Mutex \--- Mutua esclusione}\label{subsec:4.4.2} % 4.4.2
La soluzione al problema della coda di lavoro in condizione di competizione � di
permettere solo ad un thread per volta di accedere alla coda di lavoro. Una
volta che un thread inizia a guardare la coda nessun altro thread dovrebbe
essere in grado di accedervi finch� un thread non ha deciso se processare un
lavoro, e, se cos�, ha rimosso il lavoro dalla lista.

L'implementazione richiede il supporto da parte del sistema operativo. GNU/Linux
fornisce i \textit{mutexes}, abbreviazione di \textit{MUTual EXclusion locks}.
Un mutex � una speciale chiusura che un solo thread per volta ha il permesso di
chiudere. Se un thread chiude un mutex e quindi anche un secondo thread cerca di
chiudere lo stesso mutex, il secondo thread viene \textit{bloccato}, o messo in
attesa. Solo quando il primo thread sblocca il mutex il secondo thread viene
\textit{sbloccato} \--- gli viene permesso di riprendere l'esecuzione. GNU/Linux
garantisce che le condizioni di competizione non si verifichino tra thread che
cercano di bloccare un mutex; solo un thread avr� sempre la possibilit� di
ottenerne la chiusura e tutti gli altri thread verranno bloccati.

Immagina un mutex come la porta del bagno. Chiunque vi arriva prima, entra e
blocca la porta. Se qualcun altro cerca di entrare nel bagno mentre � occupato,
quella persona trover� la porta bloccata e sar� costretto ad aspettare fuori
finch� chi l'ha occupato non esce.

Per creare un mutex, crea una variabile di tipo \texttt{pthread\_mutex\_t} e
passa un puntatore ad essa a \texttt{pthread\_mutex\_init}. Il secondo argomento
a \texttt{pthread\_mutex\_init} � un puntatore ad un oggetto attributo mutex,
che specifica gli attributi del mutex. Come con \texttt{pthread\_create}, se
l'attributo puntatore � null, si assumono gli attributi di default. La variabile
mutex dovrebbe essere inizializzata una sola volta. Questo frammento di codice
dimostra la dichiarazione ed inizializzazione di una variabile mutex.
\begin{listcodeC}
pthread_mutex_t mutex;
pthread_mutex_init (&mutex, NULL);
\end{listcodeC}
Un altro semplice modo di creare un mutex con gli attributi di default � quello
di inizializzarlo con lo speciale valore \texttt{PTHREAD\_MUTEX\_INITIALIZER}.
Non � necessaria nessuna ulteriore chiamata a \texttt{pthread\_mutex\_init}. Ci�
� particolarmente conveniente per variabili globali (e, in C++, membri di dati
	statici). Il precedente frammento di codice pu� essere scritto
equivalentemente come questo:
\begin{listcodeC}
pthread_mutex_t mutex = PTHREAD_MUTEX_INITIALIZER;
\end{listcodeC}
Un thread pu� cercare di bloccare un mutex chiamando su questo
\texttt{pthread\_mutex\_lock}. Se il mutex non era bloccato, esso diventa
bloccato e la funzione ritorna immediatamente. Se il mutex era bloccato da un
altro thread, \texttt{pthread\_mutex\_lokc} blocca l'esecuzione e ritorna solo
eventualmente quando il mutex non � bloccato dall'altro thread. Pi� di un thread
per volta pu� essere bloccato su un mutex chiuso. Quando il mutex � sbloccato,
solo uno dei thread bloccati (scelto in modi imprevedibile) viene sbloccato e gli
� permesso di chiudere il mutex; gli altri thread stanno bloccati.

Una chiamata a \texttt{pthread\_mutex\_unlock} sblocca un mutex. Questa funzione
dovrebbe essere sempre chiamata dallo stesso thread che ha bloccato il mutex.
Il listato 4.11 mostra un'altra versione dell'esempio della coda di lavoro.
Adesso la coda � protetta da un mutex. Prima di accedere alla coda (sia in
	scrittura che in lettura), ogni thread blocca prima un mutex. Solo
quando l'intera sequenza di verifica della coda e rimozione del lavoro �
completa il mutex viene sbloccato. Ci� previene la condizione di competizione
precedentemente descritta.

\listfromfile{job-queue2.c}
	{Funzione Thread della coda di lavoro, protetta da un Mutex}
	{list:4.11}
	{ALP-listings/chapter-4/job-queue2.c}
Tutti accedono a \texttt{job\_queue}, il puntatore ai dati condiviso, viene tra
la chiamata a \texttt{pthread\_mutex\_lock} e la chiamata a
\texttt{pthread\_mutex\_unlock}. Ad un oggetto lavoro, memorizzato in
\texttt{next\_job}, si accede al di fuori di questa regione solo dopo che
quell'oggetto � stato rimosso dalla coda ed � quindi inaccessibile ad altri
thread.

Nota che se la coda � vuota (\texttt{job\_queue} � null), non usciamo
immediatamente dal loop perch� ci� lascerebbe il mutex permanentemente bloccato
ed eviterebbe ad ogni altro thread di accedere nuovamente alla coda di lavoro.
Piuttosto, ricordiamo questo fatto impostando \texttt{next\_job} a null ed
uscendo solo dopo aver sbloccato il mutex.

L'uso del mutex per bloccare \texttt{job\_queue} non � automatico; ti permette
di aggiungere il codice per bloccare il mutex prima di accedere a quella
variabile e quindi sbloccarlo in seguito. Per esempio, una funzione per
aggiungere un lavoro alla coda di lavoro potrebbe somigliare a questa:

\begin{listcodeC}
void enqueue_job (struct job* new_job)
{
  pthread_mutex_lock (&job_queue_mutex);
  new_job->next = job_queue;
  job_queue = new_job;
  pthread_mutex_unlock (&job_queue_mutex);
}
\end{listcodeC}

\subsection{Stallo del Mutex - Deadlocks}\label{subsec:4.4.3} % 4.4.3
I mutex forniscono un meccanismo per permettere ad un thread di bloccare
l'esecuzione di un altro. Ci� apre la possibilit� di una nuova classe di bug,
chiamati \textit{deadlocks}. Uno stallo si verifica quando uno o pi� thread sono
fermi in attesa di qualcosa che non si verifica mai.
Un semplice tipo di stallo pu� verificarsi quando lo stesso thread cerca di
bloccare un mutex due volte in una riga. Il comportamento in questo caso dipende
da quale tipo di mutex viene usato. Esistono tre tipi di mutex:
\begin{itemize}
\item Bloccare un \textit{mutex veloce} (\textit{fast mutex}) (il tipo di
	default) causer� il verificarsi di uno stallo. Un tentativo di chiudere
un mutex bloccher� finch� il mutex non sar� sbloccato. Ma poich� il thread che
ha chiuso il mutex � bloccato nello stesso mutex, la chiusura non pu� mai essere
rilasciata.
\item Bloccare un \textit{mutex ricorsivo} non causa lo stallo. Un mutex
ricorsivo pu� essere chiuso diverse volte senza problemi dallo stesso thread. Il
mutex ricorda quante volte \texttt{pthread\_mutex\_lock} � stato chiamato su di
esso dal thread che tiene la chiusura; quel thread deve fare lo stesso numero di
chiamate a \texttt{pthread\_mutex\_unlock} prima che il mutex venga attualmente
sbloccato ed un altro threadabbia il permesso di bloccarlo.
\item GNU/Linux trover� e segner� una doppia chiusura su un \textit{errore di
verifica del mutex} (\textit{error-checking mutex}) che potrebbe altrimente
causare uno stallo. La seconda chiamata consecutiva a
\texttt{pthread\_mutex\_lock} restituisce il codice d'errore \texttt{EDEADLK}.
\end{itemize}
Per default, un mutex GNU/Linux � del tipo veloce. Per creare un mutex di uno
degli altri due tipi, prima si crea un oggetto attributo mutex dichiarando una
variabile \texttt{pthread\_mutexattr\_t} e chiamando
\texttt{pthread\_mutexattr\_init} su un puntatore ad essa. Quindi impostare il
tipo di mutex chiamando \texttt{pthread\_mutexattr\_setkind\_np}; il primo
argomento � un puntatore all'oggetto attributo mutex, il secondo �
\texttt{PTHREAD\_MUTEX\_RECURSIVE\_NP} per un mutex ricorsivo, o
\texttt{PTHREAD\_MUTEX\_ERRORCHECK\_NP} per un mutex con controllo d'errore.
Passare un puntatorea questo oggetto attributo a
\texttt{pthread\_mutex\_destroy}.

La sequenza di codice illustra la creazione di un mutex con controllo d'errore,
per esempio:
\begin{listcodeC}
pthread_mutexattr_t attr;
pthread_mutex_t mutex;
pthread_mutexattr_init (&attr);
pthread_mutexattr_setkind_np (&attr, PTHREAD_MUTEX_ERRORCHECK_NP);
pthread_mutex_init (&mutex, &attr);
pthread_mutexattr_destroy (&attr);
\end{listcodeC}
Come suggerito dal suffisso "np", i tipi di mutex ricorsivo e con controllo
d'errore sono specifici per GNU/Linux e non sono portabili. Quindi, generalmente
non � consigliato usarli nei programmi (comunque, i mutex con controllo d'errore
	possono essere utili per debugging).

\subsection{Verifiche non bloccanti dei Mutex}\label{subsec:4.4.4} % 4.4.4
Occasionalmente, � utile verificare se un mutex � attualmente chiuso senza
bloccarsi su di esso. Per esempio, un thread pu� aver bisogno di chiudere un
mutex ma pu� avere altro lavoro da fare invece di bloccarsi se il mutex � gi�
chiuso. Poich� \texttt{pthread\_mutex\_lock} non ritorner� finch� il mutex non
diventa sbloccato � necessaria qualche altra funzione.

GNU/Linux fornisce \texttt{pthread\_mutex\_trylock} per questo scopo. Se chiami
\texttt{pthread\_mutex\_trylock} su un mutex non chiuso, chiuderai il mutex come
se avessi chiamato \texttt{pthread\_mutex\_lock} e
\texttt{pthread\_mutex\_trylock} ritorner� zero. Comunque, se il mutex � gi�
chiuso da un altro thread, \texttt{pthread\_mutex\_trylock} non si bloccher�.
Piuttosto, esso ritorner� immediatamente con il codice d'errore \texttt{EBUSY}.
La chiusur� del mutex tenuta dall'altro thread non sar� influenzata. Potrai
provare nuovamente dopo a chiudere il mutex.

\subsection{Semafori per i Thread}\label{subsec:4.4.5} % 4.4.5
Nell'esempio precedente, nel quale molti thread processano i lavori da una coda,
la funzione thread principale dei thread tira fuori il prossimo lavoro finch�
non ci sono pi� lavori rimasti e quindi esce dal thread. Questo schema funziona
se tutti i lavori sono accodati in anticipo o se i nuovi lavori sono accodati al
pi� tanto velocemente quanto vengono processati dai thread. Comunque, se i
thread lavorano troppo velocemente, la coda dei lavori verr� svuotata e i thread
usciranno. Se successivamente vengono accodati nuovi lavori, non pu� rimanere
nessun thread per precessarli. A noi piacerebbe piuttosto un meccanismo per
bloccare i thread quando la coda � vuota, finch� non tornano disponibili nuovi
lavori.

Un \textit{semaforo} fornisce un metodo conveniente per far ci�. Un semaforo �
un contatore che pu� essere utilizzato per sincronizzare thread multipli. Come
con un mutex, GNU/Linux garantisce che verificare o modificare il valore di un
semaforo possa essere fatto senza problemi, senza creare condizioni di
competizione.

Ogni semaforo ha un valore contatore, che � un intero non negativo. Un semaforo
supporta due operazioni di base:
\begin{itemize}
\item un'operazione (\textit{wait}) decrementa il valore del semaforo di 1. Se
il valore � gi� ero, l'operazione si blocca finch� il valore del semaforo non
diventa positivo (dovuto ad un'azione di qualche altro thread). Quando il valore
del semaforo diventa positivo, esso � decrementato di 1 e l'operazione wait
ritorna.
\item un'operazione \textit{post} incrementa il valore del semaforo di 1. Se il
semaforo era precedentemente zero e altri thread sono bloccati in un'operazione
wait su quel semaforo, uno di questi thread viene sbloccato e la sua operazione
wait si completa (ci� fa tornare il valore del semaforo nuovamente a zero)
\end{itemize}
Nota che GNU/Linux fornisce due implemntazioni di semaforo leggermente diverse.
Quella che noi descriviamo qui � l'implementazione del semaforo standard POSIX.
Usa questi semafori nella comunicazione tra thread. L'altra implementazione,
usata per la comunicazione tra i procesi, � descritta nella
\numnameref{sec:5.2} \marginpar{Sezione 5.2, Processare Semafori}. Se usi i
semafori, inlcudi \texttt{<semaphore.h>}.

Un semaforo � rappresentato da una variabile \texttt{sem\_t}. Prima di usarla,
devi inizializzarla usado la funzione \texttt{sem\_init}, passando un puntatore
alla variabile \texttt{sem\_t}. Il secondo parametro dovrebbe essere zero
\footnote{Un valore diverso da zero indicherebbe un semaforo che pu� essere
condiviso dai processi, che non � supportato da GNU/Linux per questo tipo di
semaforo}, e il terzo parametro � il valore iniziale del semaforo. Se non hai
pi� bisogno di un semaforo, � bene deallocarlo con \texttt{sem\_destroy}.

Per aspettare su un semaforo, usa \texttt{sem\_wait}. Per postare su un
semaforo, usa \texttt{sem\_post}. Viene fornita anche una funzione wait non
bloccante, \texttt{sem\_trywait}, � simile a \texttt{pthread\_mutex\_trylock}
\--- se l'attesa potrebbe essere stata bloccata poich� il valore del semaforo
era zero la funzione ritorna immediatamente, con un valore d'errore
\texttt{EAGAIN}, piuttosto che bloccarsi.

GNU/Linux fornisce anche una funzione per ottenere il valore corrente di un
semaforo, \texttt{sem\_getvalue}, che mette il valore nella variabile
\texttt{int} puntata dal suo secondo argomento. Non dovresti comunque usare il
valore di un semaforo che hai ottenuto da questa funzione per prendere una
decisione sul postare a aspettare in un semaforo. Far ci� potrebbe portare ad
una condizione di competizione: un altro thread potrebbe cambiare il valore del
semaforo tra la chiamata a \texttt{sem\_getvalue} e la chiamata ad un'altra
funzione semaforo. Usa piuttosto le funzioni atomiche post e wait.

Tornando al nostro esempio della coda di lavoro, possiamo usare un semaforo per
contare il numero di lavori in attesa sulla coda. Il listato 4.12 controlla la
coda con un semaforo. La funzione \texttt{enqueue\_job} aggiunge un nuovo lavoro
alla coda.

\listfromfile{job-queue3.c}
	{Coda di lavoro controllata da un semaforo}
	{list:4.12}
	{ALP-listings/chapter-4/job-queue3.c}
Prima di prendere un lavoro dall'inizio della coda, ogni thread prima aspetter�
al semaforo. Se il valore del semaforo � zero, che indica che la coda � vuota,
il thread semplicemente si bloccher� finch� il valore del semaforo non diventa
positivo, indicando che un lavoro � stato aggiunto alla coda.

La funzione \texttt{enqueue\_job} aggiunge un lavoro alla coda. Proprio come
\texttt{thread\_function}, esso ha bisogno di vedere il mutex coda prima di
modificare la coda \marginpar{verificare traduzione di vedere il mutex coda}.
Dopo aver aggiunto un lavoro alla coda, esso posta al semaforo, indicando che �
disponibile un nuovo lavoro. Nella versione mostrata nel listato 4.12, i thread
che processano i lavori non escono mai; se non ci sono lavori disponibili per un
po', tutti i thread semplicemente si bloccano in \texttt{sem\_wait}.

\subsection{Variabili condizione}\label{subsec:4.4.6} % 4.4.6
Abbiamo mostrato come usare un mutex per proteggere una variabile da accessi
simultanei da due thread e come usare i semafori per implementare un contatore
condiviso. Una \textit{variabile condizione} � un terzo dispositivo di
sincronizzazione che GNU/Linux fornisce; con esso, puoi implementare condizioni
pi� complesse sotto le quali eseguire i thread.

Supponi di aver scritto una funzione thread che esegue un loop all'infinito,
facendo qualche lavoro ad ogni iterazione. Il loop thread, comunque, ha bisogno
di essere controllato da un flag: il loop gira solo quando il flag � settato;
quando il flag non � settato, il loop va in pausa.

Il listato 4.13 mostra come puoi implementare ci� girando in un loop. Durante
ogni iterazione del loop, la funzione thread verifica che il flag sia impostato.
Poich� al flag accedono pi� thread, esso � protetto da un mutex. Questa
implementazione pu� essere corretta, ma non � efficiente. La funzione thread
spender� molta CPU ogni volta che il flag non � impostato, verificando e
riverificando il flag, ogni volta chiudendo e aprendo il mutex. Ci� che
realmente vuoi � un modo per far dormire il thread quando il flag non �
impostato, finch� il cambiamento di qualche circostanza non fa diventare il flag
impostato.

\listfromfile{spin-condvar.c}{Una semplice implementazione della variabile condizione}
	{list:4.13}
	{ALP-listings/chapter-4/spin-condvar.c}
Una variabile condizione ti permette di implementare una condizione sotto la
quale un thread si esegue e, al contrario, la condizione sotto la quale il
thread � bloccato.
Finch� ogni thread che potenzialmente cambia il senso della condizione usa la
variabile condizione in maniera appropriata, Linux garantisce che i thread
bloccati nella condizione verranno sbloccati quando la condizione cambia.

Come con un semaforo, un thread pu� \textit{aspettare} su una variabile
condizione. Se il thread A aspetta su una variabile condizione, esso � bloccato
finch� un altro thread, thread B, segnala la stessa variabile codizione.
Diversamente da un semaforo, una variabile condizione non ha un contatore o
memoria; il thread A deve restare in attesa sulla variabile condizione
\textit{prima} che il thread B la segnali. Se il thread B segnala la variabile
condizione prima che il thread A aspetti su essa, il segnale � perso,
\marginpar{verificare traduzione} e il thread A si blocca finch� qualche altro
thread non segnala la variabile condizione nuovamente.

Ci� � come dovresti usare una variabile condizione per rendere l'esempio
precedente pi� efficiente:
\begin{itemize}
\item Il loop nella \texttt{thread\_function} verifica il flag. Se il flag non �
impostato, il thread aspetta sulla variabile condizione.
\item La funzione \texttt{set\_thread\_flag} segnala la variabile condizione
dopo aver cambiato il valore del flag. In questo modo, se
\texttt{thread\_function} � bloccata sulla variabile condizione, esso verr�
sbloccato e la condizione verificata nuovamente.
\end{itemize}
In questo c'� un problema: c'� una condizione di competizione tra il verificare
il valore del flag e segnalare o aspettare sulla variabile condizione. Supponi
che \texttt{thread\_function} abbia verificato il flag e visto che non era
impostato. In quel momento, lo scheduler di Linux abbia messo in pausa quel
thread e ripreso quello principale. Per qualche coincidenza, il thread
principale � in \texttt{set\_thread\_flag}. Esso imposta il flag e quindi
segnala la variabile condizione. Poich� nessun thread sta aspettando sulla
variabile condizione in quel momnento (ricorda che \texttt{thread\_function}
	era stata messa in pausa prima che potesse aspettare sulla variabile
	condizione), il segnale � perso. Adesso, quando Linux rischedula l'altro
thread, inizia aspettando sulla variabile condizione e pu� finire per restare
bloccato per sempre.

Per risolvere questo problema, abbiamo bisogno di un modo per bloccare il flag e
la variabile condizione assieme con un singolo mutex. Fortunatamente, GNU/Linux
fornisce esattamente questo meccanismo. Ogni variabile condizione deve essere
usata in congiunzione con un mutex, per evitare questo tipo di condizione di
competizione. Usando questo schema, la funzione thread segue questi passi:
\begin{enumerate}
\item Il loop nella \texttt{thread\_function} chiude il mutex e legge il valore
	del flag.
\item Se il flag � impostato, esso sblocca il mutex ed esegue la funzione
	lavoro.
\item Se il flag non � impostato, esso automaticamnete sblocca il mutex ed
	aspetta sulla variabile condizione
\end{enumerate}
La caratteristica criticq qui � nel passo 3, nel quale GNU/Linux ti permette di
sbloccare il mutex e aspettare sulla variabile condizione automaticamente, senza
la possibilit� che un altro thread intervenga. Questo elimina la possibilit� che
un altro thread possa cambiare il valore del flag e segnalare la variabile
condizione tra la verifica del valore del flag di \texttt{thread\_function} e
l'attesa sulla variabile condizione.

Una variabile condizione � rappresentata da un'istanza di
\texttt{pthread\_cond\_t}. Ricorda che ogni variabile condizione dovrebbe essere
accompagnata da un mutex. Queste sono le funzioni che gestiscono le variabili
condizione:
\begin{itemize}
\item \texttt{pthread\_cond\_init} inizializza una variabile condizione. Il
	primo argomento � un puntatore a un'istanza di
	\texttt{pthread\_cond\_t}. Il secondo argomento, un puntatore ad un 
	oggetto attributo di una variabile condizione, � ignorato sotto
	GNU/Linux.\\
	Il mutex deve essere inizializzato separatamente, come descritto
	nella \numnameref{subsec:4.4.2}.
\item \texttt{pthread\_cond\_signal} segnala una variabile condizione. Un
	singolo thread che � stato bloccato sulla variabile condizione verr�
	sbloccato. Se nessun altro thread � bloccato sulla variabile condizione,
	il segnale � ignorato. L'argomento � un puntatore all'istanza di
	\texttt{pthread\_cond\_t}\\
	Una chiamata simile, \texttt{pthread\_cond\_broadcast}, sblocca
	\textit{tutti} i thread che sono bloccati sulla variabile condizione,
	invece di uno solo.
\item \texttt{pthread\_cond\_wait} blocca il thread chiamante finch� la
	variabile condizione non � segnalata. L'argomento � un puntatore
	all'istanza di \texttt{pthread\_cond\_t}. Il secondo argomento � un
	puntatore all'istanza di \texttt{pthread\_mutex\_t}.\\
	Quando \texttt{pthread\_cond\_wait} � chiamato, il mutex deve essere gi�
	chiuso dal thread chiamante. La funzione apre automaticamente il mutex e
	blocca sulla variabile condizione. Quando la variabile condizione �
	segnalata e il thread chiamante sbloccato, \texttt{pthread\_cond\_wait}
	riacquisisce automaticamente la chiusura sul mutex.
\end{itemize}
Ogni volta che il tuo programma esegue un'azione che pu� cambiare il senso della
condizione che stai proteggendo con la variabile condizione, esso dovrebbe
eseguire questi passi. (Nel nostro esempio, la condizione � lo stato del flag
thread, cos� questi passi devono essere fatti ogni volta che il flag � cambiato).
\begin{enumerate}
\item Chiudere il mutex assieme alla variabile condizione.
\item Eseguire l'azione che pu� cambiare il senso della condizione (nel nostro
	esempio, impostare il flag).
\item segnalare o broadcast la variabile condizione, dipendentemente dal
	comportamento desiderato.
\item Aprire il mutex assieme alla variabile condizione.
\end{enumerate}
Il listato 4.14 mostre nuovamente l'esempio precedente, adesso usando una
variabile condizione per proteggere il flag thread. Nota che nella
\texttt{thread\_function}, viene tenuta una chiusura sul mutex prima di
verificare il valore di \texttt{thread\_flag}. Questa chiusura � rilasciata
automaticamente da \texttt{pthread\_cond\_wait} prima di bloccare ed �
automaticamente riacquisita dopo. Nota anche che \texttt{set\_thread\_flag}
chiude il mutex prima di ipostare il valore di \texttt{thread\_flag} e segnalare
il mutex.

\listfromfile{condvar.c}{Controllare un thread usando una variabile condizione}
	{list:4.14}
	{ALP-listings/chapter-4/condvar.c}
La condizione protetta da una variabile condizione pu� essere arbitrariamente
complessa. Comunque, prima di eseguire ogni operazione che pu� cambiare il senso
della condizione, dovrebbe essere richiesta una chiusura del mutex, e la
variabile condizione dovrebbe essere segnalata successivamente.

Una variabile condizione pu� anche essere usata senza una condizione,
semplicemente come un meccanismo per bloccare un thread finch� un altro tread
non ``lo sveglia''. Pu� anche essere usato un semaforo per questo scopo. La
differenza principale � che un semaforo ``ricorda'' la chiamata wake-up
(risveglio) anche se nessun thread era bloccato su di esso in quel momento,
mentre una variabile condizione scarta la chiamata wake-up finch� qualche thread
� attualmente bloccato su di esso al momento. Inoltre, un semaforo smista solo
un singolo wake-up per post; con \texttt{pthread\_cond\_broadcast}, pu� essere
risvegliato un numero sconosciuto e arbitrario di thread allo stesso tempo.

\subsection{Deadlocks (stalli) con due o pi� thread}\label{subsec:4.4.7} % 4.4.7
Gli stalli possono verificarsi quando due o pi� thread sono bloccati,
nell'attesa che si verifichi una condizione che solo l'altro pu� causare. Per
esempio, se il thread A � bloccato su una variabile condizione in attesa che il
thread B la segnali, e il thread B � bloccato su una variabile condizione in
attesa che il thread A la segnali, si � verificato un deaklock perch� nessun
thread segnaler� mai l'altro. Dovresti fare attenzione ed evitare la possibilit�
di situazioni del genere perch� sono un po' difficili da individuare.

Un errore comune che pu� causare un deadlock riguarda un problema nel quale pi�
di un thread cerca di bloccare lo stesso insieme di oggetti. Per esempio,
considera un programma nel quale due diversi thread, che eseguono due diverse
funzioni thread, hanno bisogno di chiudere gli stessi due mutex. Supponi che il
thread A chiuda il mutex 1 e quindi il mutex 2, e che sia successo che il thread
B abbia chiuso il mutex 2 prima del mutex 1. In uno scenario di scheduling
sufficientemente sfortunato, Linux pu� schedulare A abbastanza a lungo per
chiudere il mutex 1 e quindi schedulare il thread B che prontamente chiude il
mutex 2. Adesso nessun thread pu� andare avanti erch� ognuno � bloccato in un
mutex che l'altro thread tiene chiuso.

Questo � un esempio di un problema di deadlock pi� generale, che pu� includere
non solo la sincronizzazione di oggetti come mutex, ma anche altre risorse, come
blocco di files o dispositivi. Il problema si verifica quando thread multipli
cercano di chiudere lo stesso insieme di risorse in ordini diversi. La soluzione
� quella di assicurarsi che tutti i thread che chiudono pi� di una risorsa le
chiudano nello stesso ordine.

\section{Implementazione dei Thread in GNU/Linux}\label{sec:4.5} % 4.5
L'implementazione dei thread POSIX su GNU/Linux differisce dall'implementazione
dei thread su molti altri sistemi UNIX-like in maniera importante: su GNU/Linux,
i thread sono implementati come processi. Ogni volta che chiami 
\texttt{pthread\_create} per creare un nuovo thread, Linux crea un nuovo
processo che esegue quel thread. Comunque, questo processo non � lo stesso di
un processo che creeresti con \texttt{fork}; in particolare, esso condivide lo
stesso spazio di indirizzi e le stesse risorse come il processo originale
piuttosto che riceverne delle copie.

Il programma \texttt{thread-pid} mostrato nel listato 4.15 lo dimostra. Il
programma crea un threa; sia il thread originale che quello nuovo chiamano la
funzione \texttt{getpid} e stampano i loro rispettivi ID di processo e quindi
girano all'infinito.

\listfromfile{thread-pid}{Stampa l'ID di processo per i Thread}
	{list:4.15}
	{ALP-listings/chapter-4/thread-pid.c}
Esegui il programma in background e invoca quindi \texttt{ps x} per visualizzare
i tuoi processi in esecuzione. Non dimenticare di uccidere successivamente il
programma \texttt{thread-pid} \--- esso consuma molta CPU e non fa nulla. Ecco a
cosa dovrebbe somigliare l'output
\begin{listcodeBash}
% cc thread-pid.c -o thread-pid -lpthread
% ./thread-pid &
[1] 14608
main thread pid is 14608
child thread pid is 14610
% ps x
  PID  TTY    STAT TIME COMMAND
14042 pts/9   S    0:00 bash
14608 pts/9   R    0:01 ./thread-pid
14609 pts/9   S    0:00 ./thread-pid
14610 pts/9   R    0:01 ./thread-pid
14611 pts/9   R    0:00 ps x
% kill 14608
[1]+ Terminated ./thread-pid
\end{listcodeBash}

\begin{quote}
{\large\textbf{Notifica del controllo dei lavori nella Shell}}\\
Le righe che iniziano con \texttt{[1]} sono della shelle. Quando esegui un
programma in background, la shell gli assegna un numero di lavoro \--- in questo
caso, 1 \--- e  stampa il pid del programma. Se un lavoro in background termina,
la shell riporta questo fatto la prossima volta che invochi un comando.
\end{quote}
Nota che ci sono tre processi che stanno eseguendo il programma
\texttt{thread-pid}. Il primo di questi, con pid 14608, � il thread principale
del programma; il terzo, con pid 14610, � il thread che abbiamo creato per
eseguire \texttt{thread\_function}.

Riguardo al secondo thread, con pid 14609? Questo � il ``thread manager'', che �
parte dell'implementazione interna dei thread di GNU/Linux. Il thread manager �
creato la prima volta che un programma chiama \texttt{pthread\_create} per
creare un nuovo thread.

\subsection{Gestione dei segnali}\label{subsec:4.5.1} % 4.5.1
Supponi che un programma multithread riceva un segnale. In quale thread �
invocato il gestore del segnale? Il comportamento dell'interazione tra segnali e
trhead varia da un sistema UNIX-like a un'altro. In GNU/Linux, il comportamento
� dettato dal fatto che i thread sono implementati come processi.

Poich� ogni thread � un processo separato e poich� un segnale � inviato ad un
particolare processo, non c'� ambiguit� su quale thread riceve il segnale.
Tipicamente, i segnali inviati dall'esterno del programma sono inviati al
processo corrispondente al thread principale del programma. Per esempio, se un
programma fa il fork ed il processo figlio esegue un programma multithread, il
processo padre terr� l'id di processo del thread principale del programma del
processo figlio e user� quell'id di processo per inviare segnali al suo figlio.
Questa � generalmente una buona convenzione che dovresti seguire tu stesso
quando invii segnali ad un programma multithread.

Nota che quest'aspetto dell'implementazione GNU/Linux de pthread � una variante
con il thread standard POSIX. Non fare affidamento a questo comportamento in
programmi che si intendono essere portabili.

All'interno di un programma multithread, � possibile per un thread inviare un
segnale specificatamente ad un altro thread. Usa la funzione
\texttt{pthread\_kill} per farlo. Il suo primo parametro � un ID di thread e il
suo secondo parametro � un numero di segnale.

\subsection{La chiamata di sistema \textit{clone}}\label{subsec:4.5.2} % 4.5.2
Sebbene i thread GNU/Linux creati nello stesso programma sono implementati come
processi separati, essi condividono il loro spazio virtuale di memoria ed altre
risorse. Un processo figlio creato con \texttt{fork}, comunque, ottiene copie di
questi elementi. Com'� creato il tipo di forma dei processi?
La chiamata di sistema Linux \texttt{clone} � una forma generalizzata di
\texttt{fork} e \texttt{pthread\_create} che permette al chiamante di
specificare quali risorse sono condivise tra il processo chiamante e il nuovo
processo creato. Inoltre, \texttt{clone} richiede che venga specificata l'area
di memoria per l'esecuzione dello stack che i nuovi processi useranno. Comunque,
menzioniamo \texttt{clone} qui per soddisfare la curiosit� del lettore. Questa
chiamata di sistema non dovrebbe essere ordinariamente usata nei programmi. Usa
\texttt{fork} per creare nuovi processi o \texttt{pthread\_create} per creare
thread.

\section{Processi Vs. Thread}\label{sec:4.6} % 4.6
Per alcuni programmi che traggono benefici dalla concorrenza, la decisione se
usare i processi o i thread pu� essere difficile. Qui ci sono alcune linee guida
per aiutarti a decidere quale modello di concorrenza si addice meglio al tuo
programma:
\begin{itemize}
\item Tutti i thread in un programma devono eseguire lo stesso eseguibile. Un
	processo figlio, d'altro canto, pu� eseguire un diverso eseguibile 
	chiamando la funzione \texttt{exec}.
\item Un thread che sbaglia pu� danneggiare altri thread nello stesso processo
	poich� i thread condividono lo stesso spazio di memoria ed altre
	risorse. Per esempio, una scrittura in un'area di memoria tramite un
	puntatore non inizializzato in un thread pu� corrompere la memoria
	visibile ad un altro thread.\\
	Un processo che sbaglia, d'altro canto, non pu� farlo perch� ogni
	processo ha una copia dello spazio di memoria del programma.
\item Copiare la memoria per un nuovo processo aggiunge sovraccarico addizionale
	alle prestazioni, relativo al creare un nuovo thread. Comunque, la
	compia � fatta solo quando la memoria viene modificata, cos� la penalit�
	� minima se il processo figlio soltanto legge la memoria.
\item I thread dovrebbero essere usati per programmi che hanno bisogno di
	parallelismo \textit{a grana fine}. Per esempio, se un problema pu�
	essere rotto in pi� thread con compiti pressoch� identici, potrebbe
	essere una buona scelta. I processi dovrebbero essere usati per
	programmi che hanno bisogno di un parallelismo meno preciso.
\item Condividere i dati tra i thread � inutile perch� i thread condividono la
	stessa memoria. (Comunque, si deve dare pi� attenzione per evitare le
	condizioni di competizione, come descritto precedentemente). Condividere
	i dati tra i processi richiede l'uso di meccanismi IPC, come descritto
	nel capitolo 5. Questo pu� essere pi� lento ma permette a pi� processi
	di soffrire di meno del bug della concorrenza.
\end{itemize}



























\chapter{Comunicazione tra processi}\label{cap:5}

Il \numnameref{cap:3}, ha parlato della creazione dei processi e mostrato come un processo pu� ottenere lo stato di uscita di un processo figlio. Questa � la forma pi� semplice di comunicazione tra due processi, ma non � la pi� potente. Il meccanismo del capitolo 3 non fornisce nessun modo al padre di comunicare con il figlio, tranne che per mezzo degli argomenti della linea di comando e le variabili d'ambiente, nessun altro modo per il figlio di comunicare con il padre, ad eccezione dello stato di uscita del figlio. Nessuno di questi meccanismi fornisce una maniera per comunicare con il processo figlio mentre questo � in esecuzione, questi meccanismi non permettono neppure la comunicazione con un processo al di fuori della relazione padre-figlio.

Questo capitolo descrive il significato della comunicazione tra processi che aggira queste limitazioni. Presenteremo vari modi per comunicare tra padri e figli, tra processi ``non collegati'' e anche tra processi su macchine diverse.

La comunicazione tra processi \--- \textit{Interprocess communication (IPC)} \--- � il trasferimento di dati tra processi. Per esempio, un browser web pu� richiedere una pagina web da un server web, che quindi invia i dati HTML. Questo trasferimento di dati di solito usa i socket in una connessione telefonica. In un altro esempio, pu�i voler stampare i nomi dei files di una directory usando un comando come \texttt{ls | lpr}. La shell crea un processo \texttt{ls} ed un processo separato \texttt{lpr}, collegando i due con una pipe, rappresentata dal simbolo ``\texttt{|}''. Una pipe permette una via di comunicazione tra due processi correlati. Il processo \texttt{ls} scrive dati nella pipe e il processo \texttt{lpr} legge i dati dalla pipe.

In questo capitolo, discutiamo cinque tipi di comunicazione tra processi:
\begin{itemize}
\item La memoria condivisa permette ai processi di comunicare semplicemente leggendo e scrivendo su una specifica locazione di memoria.
\item La memoria mappata � simile alla memoria condivisa, ad eccezione che essa � associata ad un file nel filesystem.
\item Le pipe permettono la comunicazione sequenziale da un processo ad un processo correlato.
\item Le FIFO sono simili alle pipe, ad eccezione che i processi non correlati possono comunicare perch� alla pipe viene dato un nome nel file system.
\item I socket supportano la comunicazioni tra processi non correlati anche in computer diversi.
\end{itemize}
Questi tipi di IPC differiscono per i seguenti criteri:
\begin{itemize}
\item Se essi estendono la comunicazione a processi correlati (processi con un antenato comune), a processi non correlati condividendo lo stesso filesystem o a ogni computer connesso ad una rete.
\item Se un processo comunicante � limitato solamente a scrivere dati o a leggere dati.
\item Il numero di processi a cui � permesso di comunicare.
\item se i processi comunicanti sono sincronizzati da IPC \--- per esempio, un processo che legge si ferma finch� non sono disponibili dati da leggere. 
\end{itemize}
In questo capitolo, omettiamo la discussione sugli IPC che permettono la comunicazione solo un numero limitato di volte, come la comunicazione tramite il valore di uscita del figlio.

\section{Memoria condivisa}\label{sec:5.1} % 5.1
Uno dei metodi pi� semplici per la comunicazione tra processi � usando la memoria condivisa. La memoria condivisa permette a due o pi� processi di accedere alla stessa memoria come se avessero tutti chiamato \texttt{malloc} e ottenuto puntatori alla stessa memoria attuale. Quando un processo cambia la memoria, tutti gli altri processi vedono la modifica.

\subsection{Comunicazione locale veloce}\label{subsec:5.1.1}
La memoria condivisa � la forma pi� veloce di comunicazione tra processi perch� tutti i processi condividono lo stesso pezzo di memoria. Accedere a questa memoria condivisa � tanto veloce quanto accedere alla memoria non condivisa del processo, e non richiede chiamate di sistema o voci del khernel. Evita anche di copiare dati non necessariamente.

Poich� il kernel non sincronizza gli accessi alla memoria condivisa, devi fornire la tua sincronizzazione. Per esempio, un processo non dovrebbe leggere dalla memoria fin dopo che i dati non vi siano stati scritti, e due processi non dovrebbero scrivere sulla stessa locazione di memoria allo stesso tempo. Una strategia comune per evitare queste condizioni di competizione � quella di usare i semafori, che sono discussi nella prossima sezione. I nostri programmi illustrativi, comunque, mostrano giusto un singolo processo che accede alla memoria, per focalizzare sui meccanismi della memoria condivisa ed evitrare di confondere il codice di esempio con la logica di sincronizzazione.

\subsection{Il modello della memoria}\label{subsec:5.1.2}
Per usare un segmento di memoria condiviso, un processo deve allocare il segmento. Quindi ogni processo che desidera accedere al segmento deve attaccarsi al segmento. Dopo aver finito il loro uso del segmento, ogni processo si stacca dal segmento. A questo punto, un processo deve deallocare il segmento.

Comprendere il modello della memoria di Linux aiuta a spiegare l'allocazione e l'attaccamento dei processi. Su Linux, ogni memoria virtuale del processo � divisa in pagine. Ogni processo mantiene una mappa dai suoi indirizzi di memoria a queste pagine di memoria virtuali, che attualmente contengono i dati. Anche se ogni processo ha i suoi indirizzi propri, la mappatura di pi� processi pu� puntare alla stessa pagina, permettendo di condividere la memoria. Delle pagine di memoria si parla nella \numnameref{sec:8.8} del \numnameref{cap:8}. \marginpar{Sezione 8.8, titolo... del capitolo 8, titolo...}
Allocare un nuovo segmento di memoria condivisa causa la creazione di una pagina di memoria virtuale. Poich� tutti i processi desiderano accedere allo stesso segmento di memoria condivisa, solo un processo dovrebbe allocare un nuovo segmento condiviso. Allocare un segmento esistente non crea nuove pagine, ma restituisce un identificatore per la pagina esistente. Per permettere ad un processo di usare un segmento di memoria condivisa, un processo lo collega, che aggiunge voci di mappatura dalla sua memoria virtuale alla pagina del segmento condiviso. \marginpar{verificare traduzione} Quando finito con il segmento, queste voci di mappatura vengono rimosse. Quando non ci sono pi� processi che vogliono accedere a questi segmenti di memoria condivisa, esattamente un processo deve deallocare le pagine di memoria virtuale.
Tutti i segmenti di memoria condivisa sono allocati come multipli interi della \textit{dimensione di pagina} del sistema, che � il numero di byte in una pagina di memoria. Su sistemi Linux, la dimensione della pagina � di 4KB, ma si pu� ottenere questo valore chiamando la funzione \texttt{getpagesize}

\subsection{Allocazione}\label{subsec:5.1.3}
Un processo alloca un segmento di memoria condivisa usando \texttt{shmget} (``SHared Memory GET''). Il suo primo parametro � una chiave intera che specifica quale segmento creare. Processi non correlati possono accedere allo stesso segmento condiviso specificando lo stesso valore chiave. Sfortunatamente, altri processi possono anche avere scelto la stessa chiave fissata, che pu� andare in conflitto. Usando la speciale costante \texttt{IPC\_PRIVATE} come valore chiave garantische che venga creato un nuovo segmento di memoria.

Il suo secondo parametro specifica il numero di byte nel segmento. Poich� i segmenti sono allocati usando le pagine, il numero di byte attualmente allocati � arrotondato superiormente all'intero multiplo della dimensione della pagina.
Il terzo parametro � l'or bit a bit dei valori flag che specificano le opzioni di \texttt{shmget}. I valori flag includono questi:
\begin{itemize}
\item \texttt{IPC\_CREAT} \--- Questo flag indica che dovrebbe essere creato un nuovo segmento. Ci� permette di chreare un nuovo segmento quando si specifica un valore chiave.
\item \texttt{IPC\_EXCL} \--- Questo flag, che � sempre usato con \texttt{IPC\_CREAT}, causa a \texttt{shmget} di fallire se � specificata una chiave di segmento che gi� esiste. Quindi, questo permette al processo chiamante di avere un segmento ``esclusivo''. Se questo flag non � stato dato e la chiave di un segmento esistente � usata, \texttt{shmget} ritorna il segmento esistente invece di crearne uno nuovo.
\item Flag modo \--- Questo valore � fatto da 9 biti, indica il permesso dato al proprietario, gruppo e il resto di controllare l'accesso al segmento. I bit di esecuzione sono ignorati. Un modo semplice per specificare i pemessi � di usare le costanti definite in \texttt{<sys/stat.h>} e documentati nella sezione \texttt{2 stat} delle pagine man.\footnote{Questi bit di permesso sono gli stessi di quelli usati per i files. Essi sono descritti nella \numnameref{sec:10.3}}. Per esempio, \texttt{S\_IRUSR} e \texttt{S\_IWUSR} specificano permessi di lettura e scrittura per il proprietario del segmento di memoria condivisa, e \texttt{S\_IROTH} e \texttt{S\_IWOTH} specificano permessi di lettura e scrittura per gli altri.
\end{itemize}
Per esempio, questa chiamata di \texttt{shmget} crea un nuovo segmento di memoria condiviso (o accede ad uno esistente, se \texttt{shm\_key} � gi� usato) che � leggibile e scrivibile per il proprietario ma non per altri utenti.
\begin{listcodeC}
int segment_id = shmget (shm_key, getpagesize (),
			IPC_CREAT | S_IRUSR | S_IWUSER);
\end{listcodeC}
Se la chiamata ha successo, \texttt{shmget} ritorna un identificatore di segmento. Se il segmento di memoria esiste gi�, vengono verificati i permessi di accesso e viene fatta una verifica per assicurarsi che il segmento non sia marcato per la distruzione.

\subsection{Attacco e distacco}\label{subsec:5.1.4}
Per rendere il segmento di memoria condivisa disponibile, un processo deve usare \texttt{shmat}, ``SHared Memory ATtach''. Passargli l'identificatore del segmento di memoria condivisa \texttt{SHMID} restituito da \texttt{shmget}. Il secondo argomento � un puntatore che specifica dove nel tuo spazio di indirizzi del processo vuoi mappare la memoria condivisa; se specifichi NULL, Linux sceglier� un indirizzo disponibile. Il terzo argomento � un flag, che pu� includere il seguente:
\begin{itemize}
\item \texttt{SHM\_RND} indica che l'indirizzo specificato per il secondo parametro dovrebbe essere arrotondato inferiormente ad un multiplo della dimensione della pagina. Se non specifichi questo flag, dovrai allineare tu stesso il secondo argomento di \texttt{shmat} alla dimensione della pagina.
\item \texttt{SHM\_RDONLY} indica che il segmento sar� solo letto, non scritto
\end{itemize}
Se la chiamata ha successo, essa restituisce l'indirizzo del segmento condiviso attaccato. I figli creati dalla chiamata \texttt{fork} ereditano i segmenti attaccati; essi possono staccare il segmento di memoria condivisa, se vogliono.

Quando hai finito con un segmento di memoria condivisa, il segmento dovrebbe essere staccato usando \texttt{shmdt} (``SHared Memory DeTach''). Passagli l'indirizzo restituito da \texttt{shmat}. Se il segmento � stato deallocato e questo era l'ultimo processo che lo stava usando, viene rimosso. Le chiamate a \texttt{exit} ed ogni altra della famiglia \texttt{exec} staccano automaticamente i segmenti.

\subsection{Controllare e deallocare la memoria condivisa}\label{subsec:5.1.5}
La chiamata \texttt{shmctl} (``SHared Memory ConTroL'') restituisce informazioni su un segmento di memoria condiviso e pu� modificarlo. Il primo parametro � un identificatore di segmento di memoria condivisa.

Per ottenere informazioni su un segmento di memoria condivisa, passa \texttt{IPC\_STAT} come secondo argomento ed un puntatore ad una \texttt{struct shmid\_ds}.

Per rimuovere un segmento, passa \texttt{IPC\_RMID} come secondo argomento, e passa NULL come terzo argomento. Il segmento viene rimosso quando l'ultimo processo che vi � attaccato lo ha infine staccato.

Ogni segmento di memoria condivisa dovrebbe essere deallocato esplicitamente usando \texttt{shmclt} quando hai finito con questo, Per evitare la violazione di limiti di tutto il sistema sul numero totale di segmenti di memoria condivisa. Invocando \texttt{exit} e \texttt{exec} vengono staccati i segmenti di memoria, ma non vengono deallocati.

Vedi la pagina man di \texttt{shmctl} per una descrizione di altre operazione che puoi fare sui segmenti di memoria condivisa.

\subsection{Un programma di esempio}\label{subsec:5.1.6}
Il programma nel listato 5.1 illustra l'uso della memoria condivisa

\listfromfile{shm.c}{Esercizio Memoria condivisa}{list:5.1}
		{ALP-listings/chapter-5/shm.c}

\subsection{Debugging}
Il comando \texttt{ipcs} fornisce informazioni sulle possibilit� di comunicazione tra processi, inclusi i segmenti condivisi. Usa il flag \texttt{-m} per ottenere informazioni sulla memoria condivisa. Per esempio, questo codice mostra che � in uso un segmento di memoria condivisa, numero 1627649:
\begin{listcodeBash}
% ipcs -m
------ Shared Memory Segments --------
key		shmid	owner	perms	bytes	nattch	status
0x00000000	1627649	user	640	25600	0
\end{listcodeBash}
Se un segmento di memoria � stato erroneamente lasciato da un programma, puoi usare il comando \texttt{ipcrm} per rimuoverlo.
\begin{listcodeBash}
% ipcrm shm 1627649
\end{listcodeBash}

\subsection{Pro e contro}\label{subsec:5.1.8}
I segmenti di memoria condivisa permettono una comunicazione bidirezionale veloce tra qualsiasi numero di processi. Ogni utente pu� sia leggere che scrivere, ma un programma deve stabilire e seguire qualche protocollo per prevenire le condizioni di competizione come la sovrascrittura di informazioni prima che queste vengano lette. Sfortunatamente, Linux non garantisce strettamente l'accesso esclusivo anche se crei un nuovo segmento condiviso con \texttt{IPC\_PRIVATE}.

Inoltre, affinch� pi� processi possano usare un segmento condiviso, essi devono cercare di arrangiarsi ad usare la stessa chiave.








\end{document}
